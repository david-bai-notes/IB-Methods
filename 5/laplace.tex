\section{The Laplace Equation}
We already encountered Laplace equation $\Delta^2\phi=0$ several times.
It has very wide applications in mathematical physics, applied mathematics and pure mathematics.
In physics, it often describes some physical systems (e.g. heat flow) that is in stationary state as it does not depend on time.
We can also see this (possibly with a forcing term) in potential theory.
For example, Laplace used it to describe gravitational systems.
It also appears in the study of incompressible fluid flow.\\
We often want to solve Laplace's equation in a domain $D$ subject to boundary conditions.
The most common ones are the dirichlet conditions where we are given the value of $\phi$ on $\partial D$ and the Neumann conditions where we specify $\underline{\hat{n}}\cdot\nabla\phi$ on $\partial D$.
\subsection{3D Cartesian Coordinates}
In 3D Cartesian coordinates, the equation becomes
$$\frac{\partial^2\phi}{\partial x^2}+\frac{\partial^2\phi}{\partial y^2}+\frac{\partial^2\phi}{\partial z^2}=0$$
The seperation of variables $\phi(x,y,z)=X(x)Y(y)Z(z)$ gives the systems
$$X^{\prime\prime}=-\lambda_lX,Y^{\prime\prime}=-\lambda_mY,Z^{\prime\prime}=-\lambda_nZ=(\lambda_l+\lambda_m)Z$$
where $\lambda_l,\lambda_m$ are seperation constants.
Therefore the general solution arising from this way is
$$\phi(x,y,z)=\sum_{l,m,n}a_{l,m,n}X_l(x)Y_m(y)Z_n(z)$$
\begin{example}[Steady Heat Conduction]
    Consider a semi-infinite rectangular bar $[0,a]\times [0,b]\times [0,\infty]$ as the domain with boundary conditions $\phi=0$ at $x=0,a$ and $y=0,b$, $\phi=1$ at $z=0$ and $\phi\to 0$ as $z\to\infty$.
    We shall try to find the eigenmodes.
    For $X^{\prime\prime}=-\lambda_lX$ with $X(0)=X(a)=0$ we get $\lambda_l=l^2\pi^2/a^2$ and $X_l(x)=\sin(l\pi x/a)$ for $l=1,2,3,\ldots$.
    For $Y^{\prime\prime}=-\lambda_mY$ we have $\lambda_m=m^2\pi^2/b^2$ and $Y_m(y)=\sin(m\pi y/b)$ again for $m=1,2,3,\ldots$.
    For $Z$, the equation would be
    $$Z^{\prime\prime}=-\lambda_nZ=(\lambda_l+\lambda_m)Z=\pi^2\left( \frac{l^2}{a^2}+\frac{m^2}{b^2} \right)Z$$
    which has exponential solutions.
    But $Z$ is bounded at infinity, therefore necessarily
    $$Z_n=Z_{l,m}=\exp\left( -\sqrt{\frac{l^2}{a^2}+\frac{m^2}{b^2}}\pi z \right)$$
    which gives the general solution
    $$\phi(x,y,z)=\sum_{l,m}a_{l,m}\sin\frac{l\pi x}{a}\sin\frac{m\pi y}{b}\exp\left( -\sqrt{\frac{l^2}{a^2}+\frac{m^2}{b^2}}\pi z \right)$$
    Now the condition $\phi(x,y,0)=1$ gives
    $$a_{l,m}=\frac{2}{b}\int_0^b\frac{2}{a}\int_0^a\sin\frac{l\pi x}{a}\sin\frac{m\pi y}{b}\,\mathrm dx\mathrm dy=\frac{16}{\pi^2lm}$$
    for odd $l,m$ and $0$ if any of them is even.
    Therefore the heat flow solution is
    $$\phi(x,y,z)=\sum_{l,m\text{ odd}}\frac{16}{\pi^2lm}\sin\frac{l\pi x}{a}\sin\frac{m\pi y}{b}\exp\left( -\sqrt{\frac{l^2}{a^2}+\frac{m^2}{b^2}}\pi z \right)$$
    This may look complicated, and yes it is complicated.
    However, for large $l,m$ (and large $z$), the exponential term would be very much close to $0$.
    This allows us to get a very nice approximation by considering just lower order terms.
\end{example}
\subsection{2D Plane Polar Coordinates}
In plane polar, Laplace's equation translates to
$$0=\nabla^2\phi=\frac{1}{r}\frac{\partial}{\partial r}\left( r\frac{\partial\phi}{\partial r} \right)+\frac{1}{r^2}\frac{\partial^2\phi}{\partial\theta^2}$$
Again we do a seperation of variables $\phi(r,\theta)=R(r)\Theta(\theta)$ to get
$$\begin{cases}
    \Theta^{\prime\prime}+\mu\Theta=0\\
    r(rR^\prime)^\prime-\mu R=0
\end{cases}$$
where $\mu$ is the seperation constant.
Assuming periodic boundary conditions, then the polar equation yields $\mu=m^2$ and $\Theta_m(\theta)$ is a superposition of $\cos(m\theta)$ and $\sin(m\theta)$.
So the radial equation becomes $r(rR^\prime)^\prime-m^2R=0$.
For $m\neq 0$, trying $R=\alpha r^\beta$ shows that $\beta=\pm m$ works, so $R_m$ is composed of $r^m$ and $r^{-m}$.
If $m=0$, $R_0$ is a linear combination of constant and $\log r$ by just integrating.
So the general solution is just
\begin{align*}
    \phi(r,\theta)&=\frac{a_0}{2}+c_0\log r\\
    &\quad+\sum_{m=1}^\infty(a_m\cos(m\theta)+b_n\sin(m\theta))r^m\\
    &\quad+\sum_{m=1}^\infty(c_m\cos(m\theta)+d_m\sin(m\theta))r^{-m}
\end{align*}
For constants $a_m,b_m,c_m,d_m$.
\begin{example}[Soap Film on a Unit Disk]
    We want to solve Laplace's equation on the unit disk, where the boundary condition is given a distorted circular wire $\phi(1,\theta)=f(\theta)$.
    Of course we want our solution to be continuous in the inside of the disk, in particular at $0$, therefore $c_m=d_m=0$ for all $m$.
    Therefore we just got
    $$\phi(r,\theta)=\frac{a_0}{2}+\sum_{m=1}^\infty(a_m\cos(m\theta)+b_m\sin(m\theta))r^m$$
    left.
    But then $f(\theta)=\phi(1,\theta)$ gives a Fourier series (again!), so
    $$a_m=\frac{1}{\pi}\int_0^{2\pi}f(\theta)\cos(m\theta)\,\mathrm d\theta,b_m=\frac{1}{\pi}\int_0^{2\pi}f(\theta)\sin(m\theta)\,\mathrm d\theta$$
    For a nontrivial distortion, the term $r^m$ then tells us that the high harmonics are concentrated near the edge of the wire.
\end{example}
\subsection{3D Cylindrical Polar Coordinates}
Here Laplace's equation become
$$0=\nabla^2\phi=\frac{1}{r}\frac{\partial}{\partial r}\left( r\frac{\partial\phi}{\partial r} \right)+\frac{1}{r^2}\frac{\partial^2\phi}{\partial\theta^2}+\frac{\partial^2\phi}{\partial z^2}$$
Seperation of variables $\phi(r,\theta,z)=R(r)\Theta(\theta)Z(z)$ gives
$$\begin{cases}
    \Theta^{\prime\prime}=-\mu\Theta\\
    Z^{\prime\prime}=\lambda Z\\
    r(rR^\prime)^\prime+(\lambda r^2-\mu)R=0
\end{cases}$$
where $\mu,\lambda$ are the seperation constants.
For the polar equation, periodic boundary conditions give $\mu_m=m^2$ and $\Theta_m(\theta)$ is a superposition of $\sin(m\theta)$ and $\cos(m\theta)$.
The radial equation is Bessel's equation (surprise?) with eigenfunctions $R_{mn}=J_m(j_{mn}r/a)$ under boundary condition $R(a)=0$ and the requirement of it not being singular (so we can exclude the Neumann functions).
The $Z$ equation then becomes $Z^{\prime\prime}=kZ$ where $k=j_{mn}/a$ which gives $Z=e^{-kz}$ (the $e^{kz}$ solution is eliminated by the boundary condition $Z\to 0$ as $z\to\infty$).
So the general solution is
$$\phi(r,\theta,z)=\sum_{m=0}^\infty\sum_{n=1}^\infty(a_{mn}\cos(m\theta)+b_{mn}\sin(m\theta))J_m(j_{mn}r/a)\exp(-j_{mn}r/a)$$
\begin{example}
    The boundary condition $\phi=0$ at $r=a$, $\phi=T_0$ at $z=0$ and $\phi\to 0$ as $z\to\infty$ gives the solution
    $$\phi(r,\theta,z)=\sum_{n=1}^\infty\frac{2T_0}{j_{0n}J_1(j_{0n})}J_0(j_{0n}r/a)\exp(-j_{0n}z/a)$$
\end{example}
\subsection{3D Spherical Polar Coordinates}
Recall that the spherical polar coordinate transforms from Cartesian coordinates by
$$x=r\sin\theta\cos\phi,y=r\sin\theta\sin\phi,z=r\cos\theta$$
for $r\in\mathbb R_{\ge 0},\theta\in [0,\pi],\phi\in[0,2\pi]$ where we have $\mathrm dV=r^2\sin\theta\,\mathrm dr\mathrm d\theta\mathrm d\phi$.
Laplace's equation transforms into
$$0=\nabla^2\Phi=\frac{1}{r^2}\frac{\partial}{\partial r}\left( r^2\frac{\partial\Phi}{\partial r} \right)+\frac{1}{r^2\sin\theta}\frac{\partial}{\partial\theta}\left( \sin\theta\frac{\partial\Phi}{\partial\theta} \right)+\frac{1}{r^2\sin^2\theta}\frac{\partial^2\Phi}{\partial\phi^2}$$
We only consider the axis-symmetric case where $\partial\Phi/\partial\phi=0$.
Again seperate the variables $\Phi(r,\theta,\phi)=R(r)\Theta(\theta)$ gives
$$\begin{cases}
    ((\sin\theta)\Theta^\prime)^\prime+\lambda(\sin\theta)\Theta=0\\
    (r^2R^\prime)^\prime-\lambda R=0
\end{cases}$$
where $\lambda$ is the seperation constant.
The substitution $x=\cos\theta$ transforms the polar equation into
$$\frac{\mathrm d}{\mathrm dx}\left( (1-x^2)\frac{\mathrm d\Theta}{\mathrm dx} \right)+\lambda\Theta=0$$
which is exactly Legendre's equation.
So we obtain the eigenvalues $\lambda_l=l(l+1)$ with eigenfunctions $\Theta_l(\theta)=P_l(x)=P_l(\cos\theta)$ where $P_l$ is the $l^{th}$ Legendre polynomial.
Putting it into the radial equation gives $(r^2R^\prime)^\prime-l(l+1)R=0$, which gives (by educated guess) the solution $R_l$ being a superposition of $r^l$ and $r^{-l-1}$.
The general axis-symmetric solution is then
$$\Phi=\sum_{l=0}^\infty(a_lr^l+b_lr^{-l-1})P_l(\cos\theta)$$
where $a_l,b_l$ can be determined by boundary conditions.
\begin{example}
    Consider the boundary condition $\Phi(1,\theta,\phi)=f(\theta)$ for some $f$.
    Regularity implies $b_l=0$ for any $l$.
    So we have
    $$f(\theta)=\sum_{l=0}^\infty a_lP_l(\cos\theta)\implies F(x)=\sum_{l=0}^\infty a_lP_l(x)$$
    with $f(\theta)=F(\cos\theta)$.
    This gives
    $$a_l=\frac{2l+1}{2}\int_{-1}^1F(x)P_l(x)\,\mathrm dx$$
    in the special case where $f(\theta)=\sin^2\theta$ we have $\Phi=2(1-P_2(\cos\theta)r^2)/3$.
\end{example}
Consider a charge on $z$-axis at $\underline{r}_0=(0,0,1)$ and the potential at $P$ is defined by
$$\Phi(\underline{r})=\frac{1}{|\underline{r}-\underline{r}_0|}=\frac{1}{\sqrt{r^2-2r\cos\theta+1}}=\frac{1}{\sqrt{r^2-2rx+1}}$$
where $x=\cos\theta$.
It is easy to see that $\Phi$ satisfies $\nabla^2\Phi=0$ in $\mathbb R^3\setminus\{\underline{r}_0\}$.
Therefore there is some $a_l$ such that
$$\frac{1}{\sqrt{r^2-2rx+1}}=\sum_{l=0}^\infty a_lP_l(x)r^l$$
We have $P_l(1)=1$ at $x=1$, therefore plugging in $x=1$ gives $a_l=1$ for any $l$, therefore
$$\frac{1}{\sqrt{r^2-2rx+1}}=\sum_{l=0}^\infty P_l(x)r^l$$
is the generating function of the Legendre polynomials.
\begin{example}[Electric Multipoles]
    Consider the case where we put charges along $z$-axis at $z=\pm a,0$ viewed from a large distance $r>>a$ with $\Phi\to 0$ as $r\to\infty$.
    Therefore $a_n=0$ for all $n$.
    When $l=0$, we just get a point charge and thus $\Phi\propto 1/r$
    This is called the monopole field of the point charge $q$.
    When $l=1$, we get the dipole (i.e. opposite charges sitting opposite each other) $\Phi\propto(\cos\theta)/r^2$ for two opposite charges.
    When $l=2$, it is like putting a charge $2q$ at the origin and $-q$ at opposite position across the origin, in which case $\Phi\propto(3\cos^2\theta-1)/(2r^3)$ gives the quadruple field.
\end{example}