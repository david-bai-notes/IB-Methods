\section{Green's Function}
\subsection{Physical Motivation}
Consider the static force on a string which can be caused by gravity acting on the mass of the string.
Let the tension be $T$ and linear mass density be $\mu$.
Suppose the string is suspended between fixed ends $y(0)=y(1)=0$.
The static force is then $-\mu\delta xg$ on the $y$ direction on the piece $\delta x$.
By resolving forces, we get the equation $-y^{\prime\prime}=f(x)$ where $f(x)=-\mu g/T$.
We can solve it via direct integration.
For uniform mass density (i.e. $\mu$ is constant), we have
$$-y=-\frac{\mu g}{2T}x^2+k_1x+k_2\implies y(x)=\left( -\frac{\mu g}{T} \right)\frac{1}{2}x(1-x)$$
by the boundary conditions.
This is a parabolic curve.
\footnote{You might be expecting the catenary curve instead -- but not really, since in that problem we require the string to be non-elastic (i.e. has a fixed length) which makes it a completely different situation.}
Another way to solve this problem is to disassemble the force on the string as the sum of infinitesimal parts, and consider their superposition.
Assume the string is massless but with a point mass $\delta m$ suspended at $x=\xi_i$.
Obviously the solution is simply two line segments meeting at some points $(\xi_i,y_i)$.
Suppose the segment closer to $0$ makes an angle $\theta_1$ with the horizontal and the other segment makes an angle $\theta_2$, then resolving in $y$-direction gives
$$0=T(\sin\theta_1+\sin\theta_2)-\delta mg=T\left( \frac{-y_i}{\xi_i}+\frac{-y_i}{1-\xi_i}-\delta mg \right)$$
Solving this gives us $y_i=-\delta mg\xi_i(1-\xi_i)/T$.
So the solution is
$$y_i(x)=\frac{-\delta mg}{T}\begin{cases}
    x(1-\xi_i)\text{, for $x<\xi_i$}\\
    \xi(1-x)\text{., for $x>\xi_i$}
\end{cases}=f_iG(x,\xi),f_i=\frac{-\delta mg}{T}$$
where $f_i$ is interpreted as the source $f$ around a infinitesimal neighbourhood of $\xi_i$.
Now the superposition of the solution for $N$ point masses $\delta m$ at $x=\{\xi_i\}$ gives
$$y(x)=\sum_{i=1}^Nf_iG(x,\xi_i)$$
If we take the continuum limit
$$f_i=-\frac{\delta mg}{T}=-\frac{\mu\delta x g}{T}=f(x)\,\mathrm dx$$
we have
$$y(x)=\int_0^1f(\xi)G(x,\xi)\,\mathrm d\xi=\left( -\frac{\mu g}{T} \right)\frac{1}{2}x(1-x)$$
\subsection{Definitions of Green's Function}
We wish to solve the inhomogeneous ODE $\mathcal Ly=f(x)$ where $\mathcal L=\alpha y^{\prime\prime}+\beta y^\prime+\gamma y$ on $[a,b]$ subject to boundary conditions $y(a)=y(b)=0$.
We require $\alpha\neq 0$ over $[a,b]$ and $\alpha,\beta,\gamma$ all continuous and bounded.
\begin{definition}
    The Green's function $G$ for the operator $\mathcal L$ is the solution to $\mathcal LG(x,\xi)=\delta(x-\xi)$ subject to homogeneous boundary conditions $G(a,\xi)=G(b,\xi)=0$ for all $\xi$.
\end{definition}
So by linearily, if such $G$ does exist, then we have
$$y(x)=\int_a^bG(x,\xi)f(\xi)\,\mathrm d\xi$$
Indeed,
$$\mathcal Ly=\int_a^b\mathcal LG(x,\xi)f(\xi)\,\mathrm d\xi=\int_a^b\delta(x-\xi)f(\xi)\,\mathrm d\xi=f(x)$$
Loosely speaking, we can write $y=\mathcal L^{-1}f$ where
$$\mathcal L^{-1}=\int_a^b\mathrm d\xi\, G(x,\xi)$$
Now, by our established study of these sort of ODEs, the Green's function splits into two smooth enough parts
$$G(x,\xi)=\begin{cases}
    G_1(x,\xi)\text{, for $x\in [a,\xi)$}\\
    G_2(x,\xi)\text{, for $x\in (\xi,b)$}
\end{cases}$$
such that the following conditions hold:\\
1. $\mathcal LG_1=\mathcal LG_2=0$ at any $x\neq\xi$.\\
2. $G_1(a,\xi)=G_2(b,\xi)=0$ for any $\xi$.\\
3. $G$ is continuous, so $G_1(\xi,\xi)=G_2(\xi,\xi)$.\\
4. The jump condition of $G^\prime$, that is
$$[G^\prime(\cdot,\xi)]^{\xi_+}_{\xi_-}=G_2^\prime(\xi_+,\xi)-G_1^\prime(\xi_-,\xi)=\frac{1}{\alpha(\xi)}$$
How to construct $G$?
Note that $\mathcal L$ is a second order differential opertaor, so for $x\in[a,\xi)$, $G_1(x,\xi)=A(\xi)y_1(x)+B(\xi)y_2(x)$ for linearly independent $y_1,y_2$.
The boundary condition $G_1(a,\xi)=0$ gives $G_1(x,\xi)=C(\xi)y_-(x)$ where $y_-(a)=0$.
Similarly for $x\in(\xi,b]$ we have $G_2(x,\xi)=D(\xi)y_+(x)$ with $y_+(b)=0$.
Then continuity condition gives $C(\xi)y_-(\xi)=D(\xi)y_+(\xi)$ and $D(\xi)y_+^\prime(\xi_+)-C(\xi)y_-^\prime(\xi_-)=\alpha(\xi)^{-1}$.
In other words we have the system
$$\begin{pmatrix}
    y_-(\xi)&y_+(\xi)\\
    -y_-^\prime(\xi_-)&y_+^\prime(\xi_+)
\end{pmatrix}\begin{pmatrix}
    C(\xi)\\
    D(\xi)
\end{pmatrix}=\begin{pmatrix}
    0\\
    1/\alpha(\xi)
\end{pmatrix}$$
which has the solution (after extending everything continuously to $x=\xi$)
$$C(\xi)=\frac{y_+(\xi)}{\alpha(\xi)W(\xi)},D(\xi)=\frac{y_-(\xi)}{\alpha(\xi)W(\xi)}$$
where $W(\xi)=y_-(\xi)y_+^\prime(\xi)-y_+(\xi)y_-^\prime(\xi)$ is the Wronskian which is nonzero if $y_+,y_-$ are linearly independent.
So the final Green's function is
$$G(x,\xi)=\begin{cases}
    y_-(x)y_+(\xi)/(\alpha(\xi)W(\xi))\text{, for $x\in[a,\xi]$}\\
    y_+(x)y_-(\xi)/(\alpha(\xi)W(\xi))\text{, for $x\in[\xi,b]$}
\end{cases}$$
Therefore the solution to the original boundary value problem $\mathcal Ly=f$ for $y=0$ at $a,b$ is
\begin{align*}
    y(x)&=\int_a^bG(x,\xi)f(\xi)\,\mathrm d\xi\\
    &=\int_a^xG_2(x,\xi)f(\xi)\,\mathrm d\xi+\int_x^bG_1(x,\xi)f(\xi)\,\mathrm d\xi\\
    &=y_+(x)\int_a^x\frac{y_-(\xi)f(\xi)}{\alpha(\xi)W(\xi)}\,\mathrm d\xi+y_-(x)\int_x^b\frac{y_+(\xi)f(\xi)}{\alpha(\xi)W(\xi)}\,\mathrm d\xi
\end{align*}
\begin{note}
    1. If $\mathcal L$ is in Sturm-Liouville form, then $\beta=\alpha^\prime$, then $\alpha(\xi)W(\xi)$ is a constant and hence $G$ has to be symmetric.\\
    2. Often we take $\alpha=1$.\\
    3. The indefinite integrals are the particular integrals in the particular integral in the Sturm-Liouville solution.
\end{note}
\begin{example}
    For $y^{\prime\prime}-y=f(x),y(0)=y(1)=0$, the homogeneous solutions are $y_1=e^x$ and $y_2=e^{-x}$.
    Imposing the homogeneous boundary conditions reveals that $y_-(x)=\sinh(x)$ and $y_+(x)=\sinh(1-x)$.
    The countinuity condition of $G$ gives $C=D\sinh(1-\xi)/(\sinh\xi)$.
    The jump condition of $G^\prime$ then gives
    $$D=-\frac{\sinh\xi}{\sinh 1},C=-\frac{\sinh(1-\xi)}{\sinh 1}$$
    So
    $$y(x)=-\frac{\sinh(1-x)}{\sinh 1}\int_0^x\sinh(\xi)f(\xi)\,\mathrm d\xi-\frac{\sinh x}{\sinh 1}\int_x^1\sinh(1-\xi)f(\xi)\,\mathrm d\xi$$
\end{example}
For inhomogenous boundary conditions, we simply need to find a solution to $\mathcal Ly_p=0$ satisfying them and solve for $\mathcal Ly_g=f$ under homogeneous boundary conditions by Green's functions.
Adding them up gives the particular solution $y=y_p+y_g$.
\begin{example}
    For $y^{\prime\prime}-y=f(x)$ with $y(0)=0,y(1)=1$, we have the solution $y_p(x)=\sinh x/\sinh 1$ to $y^{\prime\prime}-y=0$ under the same boundary conditions, so the particular solution would be
    \begin{align*}
        y(x)&=y_p(x)+y_g(x)\\
        &=\frac{\sinh x}{\sinh 1}-\frac{\sinh(1-x)}{\sinh 1}\int_0^x\sinh(\xi)f(\xi)\,\mathrm d\xi-\frac{\sinh x}{\sinh 1}\int_x^1\sinh(1-\xi)f(\xi)\,\mathrm d\xi
    \end{align*}
\end{example}
How about high-order ODEs?
Suppose we have $\mathcal Ly=f(x)$ with the highest order term $\alpha(x)y^{(n)}(x)$ in $\mathcal Ly$ with $\alpha\neq 0$ everywhere, then $\mathcal LG(x,\xi)=\delta(x-\xi)$ has the properties:\\
1. $G_1,G_2$ are solutions to $\mathcal LG=0$.\\
2. $G_1,G_2$ satisfy the homogeneous boundary conditions.\\
3. Continuity condition of $G_1^{(i)}(\xi,\xi)=G_2^{(i)}(\xi,\xi)$ for all $i=1,\ldots,n-2$.\\
4. Jump condition of $[G^{(n-1)}(\cdot,\xi)]_{\xi_-}^{\xi_+}=G_2^{(n-1)}(\xi_+,\xi)-G_1^{(n-1)}(\xi_-,\xi)=1/\alpha(\xi)$.\\
We want to take a look at the eigenfunction expansion of $G$.
Suppose $\mathcal L$ is in Sturm-Liouville form with eigenfunctions $y_n(x)$ with eigenvalues $\lambda_n$.
We seek an expansion
$$G(x,xi)\sum_{n=1}^\infty A_n(\xi)y_n(x)$$
satisfying $\mathcal LG=\delta(x-\xi)$.
Now suppose such an expansion exists, then
\begin{align*}
    \sum_{n=1}^\infty A_n(\xi)\lambda_nw(x)y_n(x)&=\sum_{n=1}^\infty A_n(\xi)\mathcal Ly_n(x)=\mathcal LG\\
    &=\delta(x-\xi)=w(x)\sum_{n=1}^\infty y_n(\xi)\frac{y_n(x)}{N_n}
\end{align*}
where $N_n=\langle y_n,y_n\rangle_w$ is the normalisation constant.
Consequently $A_n(\xi)=y_n(\xi)/(\lambda_nN_n)$, therefore
$$G(x,\xi)=\sum_{n=1}^\infty\frac{y_n(\xi)y_n(x)}{\lambda_nN_n}=\sum_{n=1}^\infty\frac{Y_n(\xi)Y_n(x)}{\lambda_n}$$
\subsection{Construction of Green's Function from Initial Values}
We want to solve $\mathcal Ly(t)=f(t)$ for $t\ge a$ with $y(a)=y^\prime(a)=0$.
The Green's function then should satisfy $\mathcal LG=\delta(t-\tau)$ with $G(a)=G^\prime(a)=0$.
For $t<\tau$, $G=G_1$ satisfies $\mathcal LG_1=0$.
So $G_1=Ay_1+By_2$ for $A,B$ constants and $y_1,y_2$ linearly independent solutions to $\mathcal Ly=0$.
The initial conditions then give
$$\begin{pmatrix}
    y_1(a)&y_2(a)\\
    y_1^\prime(a)&y_2^\prime(a)
\end{pmatrix}\begin{pmatrix}
    A\\
    B
\end{pmatrix}=\begin{pmatrix}
    0\\
    0
\end{pmatrix}$$
But $y_1,y_2$ are independent, so the Wronskian is nowhere zero, hence nonzero at $a$.
Therefore necessarily $A=B=0$, so $G_1(t,\tau)=0$ for $a\le t<\tau$.\\
For $t>\tau$, we have $G=G_2$ for $\mathcal LG_2=0$ and $G_2(\tau,\tau)=0$ by continuity.
So we can choose solution $y_+$ to $\mathcal Ly=0$ such that $G_2(t,\tau)=D(\tau)y_+(t)$.
But we must have (extending everything continuously to $\tau$)
$$\frac{1}{\alpha(\tau)}=G_2^\prime(\tau,\tau)-G_1^\prime(\tau,\tau)$$
which gives $D(\tau)=1/(\alpha(\tau)y_+^\prime(\tau))$, hence
$$G(t,\tau)=\begin{cases}
    0\text{, for $t\le\tau$}\\
    y_+(t)/(\alpha(\tau)y_+^\prime(\tau))\text{, for $t\ge\tau$}
\end{cases}$$
So we get the solution
$$y(t)=\int_a^tG_2(t,\tau)f(\tau)\,\mathrm d\tau=\int_a^t\frac{y_+(t)f(\tau)}{\alpha(\tau)y_+^\prime(\tau)}\,\mathrm d\tau$$
which looks simpler as we built in the causality with the initial conditions.
\begin{example}
    For $y^{\prime\prime}-y=f(t)$ with $y(0)=y^\prime(0)=0$.
    We then obtain $G_2(t,\tau)=\sinh(t-\tau)$.
    Therefore
    $$y(t)=\int_0^tf(\tau)\sinh(t-\tau)\,\mathrm d\tau$$
\end{example}