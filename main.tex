\documentclass[a4paper]{article}

\usepackage{hyperref}

\newcommand{\triposcourse}{Methods}
\newcommand{\triposterm}{Michaelmas 2020}
\newcommand{\triposlecturer}{Prof. E. P. S. Shellard}
\newcommand{\tripospart}{IB}

\usepackage{amsmath}
\usepackage{amssymb}
\usepackage{amsthm}
\usepackage{mathrsfs}

\usepackage{tikz-cd}

\theoremstyle{plain}
\newtheorem{theorem}{Theorem}[section]
\newtheorem{lemma}[theorem]{Lemma}
\newtheorem{proposition}[theorem]{Proposition}
\newtheorem{corollary}[theorem]{Corollary}
\newtheorem{problem}[theorem]{Problem}
\newtheorem*{claim}{Claim}

\theoremstyle{definition}
\newtheorem{definition}{Definition}[section]
\newtheorem{conjecture}{Conjecture}[section]
\newtheorem{example}{Example}[section]

\theoremstyle{remark}
\newtheorem*{remark}{Remark}
\newtheorem*{note}{Note}

\title{\triposcourse{}
\thanks{Based on the lectures under the same name taught by \triposlecturer{} in \triposterm{}.}}
\author{Zhiyuan Bai}
\date{Compiled on \today}

%\setcounter{section}{-1}

\begin{document}
    \maketitle
    This document serves as a set of revision materials for the Cambridge Mathematical Tripos Part \tripospart{} course \textit{\triposcourse{}} in \triposterm{}.
    However, despite its primary focus, readers should note that it is NOT a verbatim recall of the lectures, since the author might have made further amendments in the content.
    Therefore, there should always be provisions for errors and typos while this material is being used.
    \tableofcontents
    \section{Fourier Series}
\subsection{Periodic Functions}
\begin{definition}
    A function $f$ is periodic with period $T$ is $f(x+T)=f(x)$ for any $x$.
\end{definition}
\begin{example}
    The physical quantities in a simple harmonic motion are periodic in time $t$.
    For example, if we take a simple pendulum, then the height of the pendulum bulb can be described
    \footnote{Approximated.}
    by $y=A\sin(\omega t)$, so $y$ is periodic (in $t$) with period $T=2\pi/\omega$.
    So it has angular frequency $\omega$ and frequency $f=1/T$.\\
    In space, the wavelength is $\lambda=2\pi/k$ and the (angular) wavenumber is $k=2\pi/\lambda$.
\end{example}
Consider the set of functions $g_n(x)=\cos(n\pi x/L)$ and $h_n(x)=\sin(n\pi x/L)$ where $n$ is taken as positive integer.
They are obviously all periodic with period $2L$.
\begin{definition}
    For (sufficiently nice) $f,g:[0,2L)\to\mathbb R$, we define their inner product to be
    $$\langle f,g\rangle =\int_0^{2L}f(x)g(x)\,\mathrm dx$$
\end{definition}
\begin{proposition}
    The functions $g_n,h_n$ are mutually orthogonal on the inteval $[0,2L]$ with respect to the inner product above.
\end{proposition}
\begin{proof}
    Recall the identites we learnt half an eternity ago:
    \begin{align*}
        \cos A\cos B&=\frac{\cos(A-B)+\cos(A+B)}{2}\\
        \sin A\sin B&=\frac{\cos(A-B)-\cos(A+B)}{2}\\
        \sin A\cos B&=\frac{\sin(A-B)+\sin(A+B)}{2}
    \end{align*}
    We can obtain, by simply integrating, that $\langle h_n,g_m\rangle=0$ for any $m,n$.
    Similarly, for any $m\neq n$, $\langle h_n,h_m\rangle=\langle g_n,g_m\rangle=0$.
    So they are orthogonal.
\end{proof}
What if $m=n$?
By integrating again, we can get
$$\langle g_n,g_n\rangle=\begin{cases}
    L\text{, if $n\neq 0$}\\
    2L\text{, if $n=0$}
\end{cases},\langle h_n,h_n\rangle=\begin{cases}
    L\text{, if $n\neq 0$}\\
    0\text{, if $n=0$}
\end{cases}$$
This shows that $g_n$ and $h_n$ form a linearly independent set.
We decree that this set actually ``spans'' the space of ``well-behaved'' periodic functions with period $2L$.
We will get to the intuitive reason why we make such an assertion (and what does it actually mean) in a moment.\\
In finite dimensional vector spaces like $\mathbb R^3$, we have the standard basis which forms a orthonormal basis.
We can make the analogy to the space of nice enough functions mentioned above so that we can say this set of trigonometric functions form a ``basis'' there given that we can indeed represent every (nice) function with a (possibly infinite) series of linear combinations of $f_n,g_n$.
\footnote{As you expect, something not rigourous shall start to happen.}
\subsection{Definition of a Fourier Series}
We assert that we can represent any ``well-behaved'' periodic functions $f$ with period $2L$ in the form
$$\frac{1}{2}a_0+\sum_{n=1}^\infty a_n\cos\frac{n\pi x}{L}+\sum_{n=1}^\infty b_n\sin\frac{n\pi x}{L}$$
We sure will want this series to converge to $f$ wherever $f$ is continuous.
As a jump discontinuity, we would want this series to converge to the average value of the upper and lower limits of $f$ at that point.\\
If these conditions are satisfied and we are allowed to exchange limiting operations, previous discussions then yield
$$\langle h_n,f\rangle=Lb_n\implies b_n=\frac{1}{L}\langle h_n,f\rangle=\frac{1}{L}\int_0^{2L}f(x)\sin\frac{n\pi x}{L}\,\mathrm dx$$
Similarly, for any $n$ we have
$$a_n=\frac{1}{L}\langle g_n,f\rangle=\frac{1}{L}\int_0^{2L}f(x)\cos\frac{n\pi x}{L}\,\mathrm dx$$
\begin{note}
    1. The coefficient $1/2$ in front of $a_0$ helped here as it makes the above formula work for $n=0$ too.
    Actually $a_0/2$ is the average value of $f$ over the interval $[0,2L)$.\\
    2. The range of integration actually does not matter much as long as its length is $2L$.
    E.g. we can replace it by $[-L,L)$ as well.\\
    3. We can think of the Fourier series of a function as decomposing the function into harmonics.
\end{note}
\begin{example}[Sawtooth wave]
    Consider a $2L$-periodic function where $f(x)=x$ for $x\in [-L,L)$.
    Then obviously $a_n=0$ for any $n$ as $f$ is odd.
    Whereas integration by part reveals that $b_n=2L(-1)^{n+1}/(n\pi)$.
    So the Fourier series has the form
    $$2L\sum_{n=1}^\infty\frac{(-1)^{n+1}}{n\pi}\sin\frac{n\pi x}{L}$$
    We know this is (slowly) convergent by the alternating series test.
    A plot of the truncated series seems to show that it does converge to what we want.
\end{example}
\begin{note}
    In the above example, as $n\to\infty$, the Fourier series approximation improves and convergent where the original function is continuous.
    Also, at the discontinuity, it does converge to the average value of the limits from two sides.
    So this particular Fourier series behaves as advertised.\\
    One should also observe that this Fourier series has a persistent ``overshoot'' near the discontinuity which is approximately $9\%$.
    This is known as the Gibbs' Phenomenon.
\end{note}
\subsection{The Dirichlet Conditions and Fourier's Theorem}
A natural question is then which functions are allowed to have a proper Fourier series.
Surprisingly, a big, yet hard to precisely characterise, class of functions has a convergent Fourier series that has the desired properties.
This class even includes some classical counterexamples in analysis.
As an applied course, we will just look at some of the sufficient conditions.
\begin{theorem}[Fourier's Theorem]
    If $f$ is a bounded periodic function with period $2L$ with a finite number of minima, maxima, and discontinuities in $[0,2L)$, then its Fourier series converges to $f$ where it is continuous and converges to the average of the two side limits.
\end{theorem}
The conditions in this theorem is known as the Dirichlet conditions.
\begin{note}
    1. These conditions are hella weak compared to our conditions for a function to have e.g. a Taylor series.
    However, pathological functions like $1/x,\sin(1/x),1_{\mathbb R\setminus\mathbb Q}(x)$ are excluded from these conditions.\\
    2. The converse is not true, as $\sin(1/x)$ has a Fourier series we desire.
\end{note}
\begin{proof}
    You don't really expect to see an actual proof here, do you?
\end{proof}
Another subject of interest is the rate of convergence of a Fourier series.
Perhaps unsurprisingly, it depends on the smoothness of the function.
\begin{theorem}
    If $f(x)$ is $p^{th}$ differentiable but $f^{(p)}$ is not continuous, then its Fourier series converges as $O(n^{-(p+1)})$ as $n\to\infty$.
\end{theorem}
\begin{proof}
    Ditto.
\end{proof}
\begin{example}
    1. Consider the square wave
    $$f(x)=\begin{cases}
        1\text{, for $0\le x<1$}\\
        -1\text{, for $-1\le x<0$}
    \end{cases}$$
    That extends periodically with period $2$.
    Then it has a Fourier series
    $$4\sum_{m=1}^\infty\frac{\sin[(2m-1)\pi x]}{(2m-1)\pi}$$
    which, as one can see both from the preceding theorem (with $p=0$) and observation, converges slowly.\\
    2. Consider the general ``see-saw'' wave
    $$f(x)=\begin{cases}
        x(1-\xi)\text{, for $x\in[0,\xi)$}\\
        \xi(1-x)\text{, for $x\in[\xi,1)$}
    \end{cases}$$
    which extends as an odd periodic function with period $2$.
    This has Fourier series
    $$2\sum_{m=1}^\infty\frac{\sin (n\pi\xi)\sin (n\pi x)}{(n\pi)^2}$$
    which converges with $p=1$ in the preceding theorem.
    In particular, $\xi=1/2$ gives
    $$2\sum_{m=1}^\infty(-1)^{m+1}\frac{\sin[(2m-1)\pi x]}{[(2m-1)\pi]^2}$$
    which can be seen, immediately, that it converges faster than the series in the previous example.\\
    3. Take $f(x)=x(1-x)/2$ for $x\in[0,1)$ that extends as an odd periodic function with period $2$.
    Then its Fourier series is
    $$4\sum_{m=1}^\infty\frac{\sin[(2m-1)\pi x]}{[(2m-1)\pi]^3}$$
    which has $p=2$.\\
    4. Take $f(x)=(1-x^2)^2$, then $a_n=O(n^{-4})$.
\end{example}
Of course, we want to integrate and differentiate a Fourier series term-by-term.
Integration, as one expect, seldom yields problems as it imposed very few restrictions on the function.
And indeed, we are just going to assume we can integrate any Fourier series term-by-term and they guarantee to yield a smoother function, which satisfies the Dirichlet conditions if the original function does.\\
Differentiation is more problematic when doing it term-by-term.
\begin{example}
    Take the square wave again which is known to have Fourier series
    $$4\sum_{m=1}^\infty\frac{\sin[(2m-1)\pi x]}{(2m-1)\pi}$$
    which, after term-by-term differentiation, yields
    $$4\sum_{m=1}^\infty\cos[(2m-1)\pi x]$$
    which is clearly divergent.
    This is perhaps unsurprising as the original function is not even continuous.
\end{example}
\begin{theorem}
    If $f(x)$ is differentiable and both $f,f^\prime$ satisfy Dirichlet conditions, then we can differentiate the Fourier series of $f$ term-by-term to get the Fourier series of $f^\prime$.
\end{theorem}
\begin{proof}
    Haha.
\end{proof}
\begin{example}
    If we differentiate the see-saw curve with $\xi=1/2$, then we will get an offset of the Fourier series of the square wave.
\end{example}
\subsection{Parseval's Theorem}
There is some interesting relation between the integral of the square of a function and the square of the Fourier coefficients of that function.
If the function is nice enough to have a nice enough Fourier series, then by orthogonality,
\begin{align*}
    \int_0^{2L}f(x)^2\,\mathrm dx&=\int_0^{2L}\left( \frac{a_0}{2}+\sum_{n\ge 1}a_n\cos\frac{n\pi x}{2}+\sum_{n\ge 1}b_n\sin\frac{n\pi x}{2} \right)^2\,\mathrm dx\\
    &=\int_0^{2L}\left( \frac{a_0^2}{4}+\sum_{n\ge 1}a_n^2\cos^2\frac{n\pi x}{2}+\sum_{n\ge 1}b_n^2\sin^2\frac{n\pi x}{2} \right)\,\mathrm dx\\
    &=L\left( \frac{a_0^2}{2}+\sum_{n\ge 1}(a_n^2+b_n^2) \right)
\end{align*}
This is also called the completeness relation as the left hand side would be greater than or equal to the right hand side if any basis functions are missing from the series.
This is known as Parseval's Theorem.
\begin{theorem}[Parseval's Theorem]\label{parseval}
    For a nice enough function $f$ with Fourier coefficients $a_n,b_n$, we have
    $$\int_0^{2L}f(x)^2\,\mathrm dx=L\left( \frac{a_0^2}{2}+\sum_{n\ge 1}(a_n^2+b_n^2) \right)$$
\end{theorem}
\begin{proof}
    Above.
\end{proof}
\begin{example}
    Consider the sawtooth curve with $f(x)=x,x\in[-L,L)$ with period $2L$.
    Then Parseval's Theorem reveals that
    $$\frac{2}{3}L^3=\int_{-L}^Lx^2\,\mathrm dx=L\sum_{n=1}^\infty\frac{4L^2}{n^2\pi^2}=\frac{4L^3}{\pi^2}\sum_{n=1}^\infty\frac{1}{n^2}\implies \sum_{n=1}^\infty\frac{1}{n^2}=\frac{\pi^2}{6}$$
\end{example}
\begin{remark}
    If we think of the integral of the square as the inner product of a function with itself, then Parseval's Theorem can be thought of an analog of Pythagoras' Theorem in this space of functions.
\end{remark}
\subsection{Alternative Fourier Series}
Consider a function $f:[0,L)\to\mathbb R$.
We can extend $f$ to a periodic function of period $2L$ in two ways:\\
1. We can require the function to be odd, then $a_n=0$ for all $n$ and
$$b_n=\frac{2}{L}\int_0^Lf(x)\sin\frac{n\pi x}{L}\,\mathrm dx$$
and the Fourier series would be $\sum_{n\ge 1}b_n\sin(n\pi x/L)$, which is called a Fourier sine series.
The sawtooth function is an example of this.\\
2. We can require the function to be even, then $b_n=0$ for all $n$ and
$$a_n=\frac{2}{L}\int_0^Lf(x)\cos\frac{n\pi x}{L}\,\mathrm dx$$
So the Fourier series is $a_0/2+\sum_{n\ge 1}a_n\cos(n\pi x/L)$.
This is called a Fourier cosine series.
$f(x)=(1-x^2)^2$ is an example (where $L=1$).\\
The actual thing we want is tp represent the Fourier series more neatly in terms of exponentials.
We know that
$$\cos\frac{n\pi x}{L}=\frac{e^{in\pi x/L}+e^{-in\pi x/L}}{2},\sin\frac{n\pi x}{L}=\frac{e^{in\pi x/L}-e^{-in\pi x/L}}{2i}$$
So by writing $c_0=a_0/2$ and
$$c_m=\begin{cases}
    (a_m-ib_m)/2\text{, for $m>0$}\\
    (a_{-m}+ib_{-m})/2\text{, for $m<0$}
\end{cases}$$
We obtain
$$\frac{a_0}{2}+\sum_{n=1}^\infty a_n\cos\frac{n\pi x}{L}+\sum_{n=1}^\infty b_n\sin\frac{n\pi x}{L}=\sum_{m=-\infty}^\infty c_me^{im\pi x/L}$$
Equivalently, if we extend our inner product to the complex functions
$$\langle f,g\rangle=\int_{-L}^Lf(x)g^*(x)\,\mathrm dx$$
Then $\langle e^{im\pi x/L},e^{in\pi x/L}\rangle=2L\delta_{mn}$, which means they are orthogonal as well and we can then obtain
$$c_m=\frac{1}{2L}\langle f(x),e^{im\pi x/L}\rangle=\frac{1}{2L}\int_{-L}^Lf(x)e^{-im\pi x/L}\,\mathrm dx$$
By thinking them as a set of basis of a space of nice-enough functions in the way we did for $\sin$ and $\cos$.
Parseval's Theorem can then be stated as
$$\int_{-L}^Lf(x)^2\,\mathrm dx=2L\sum_{n=-\infty}^\infty|c_n|^2$$
\subsection{Some Motivations of Fourier Series}
\begin{definition}
    The complex inner product $\langle,\rangle:\mathbb C^N\times\mathbb C^N\to\mathbb C$ is defined by
    $$\langle\underline{u},\underline{v}\rangle=\underline{u}^\dagger\underline{v}$$
\end{definition}
An $N\times N$  matrix $A$ is self-adjoint (or Hermitian) if
$$\forall\underline{u},\underline{v}\in\mathbb C^N,\langle A\underline{u},\underline{v}\rangle=\langle\underline{u},A\underline{v}\rangle$$
One can show easily that this is just saying $A^\dagger=A$.
It can be easily shown that $A$ satisfies:\\
1. All eigenvalues are real for all $n$.\\
2. Eigenvectors associated with different eigenvalues are orthogonal with respect to $\langle,\rangle$.\\
Spectral Theorem then shows that we have an orthonormal basis of $\mathbb C^N$ consisting of eigenvectors $\{\underline{v}_1,\ldots,\underline{v}_N\}$.\\
Now, given any $\underline{b}$, if we want to solve for $\underline{x}$ in $A\underline{x}=\underline{b}$, then a way to do it is to express $\underline{b}=\sum_nb_n\underline{v}_n$ and observe that if $\sum_nc_n\underline{v}_n$ is a solution then
$$\sum_nb_n\underline{v}_n=A\left(\sum_{n=1}^Nc_n\underline{v}_n\right)=\sum_{n=1}^Nc_n\lambda_n\underline{v}_n$$
where $\lambda_n$ is the eigenvalue associated with $\underline{v}_n$.
So if $A$ is nonsingular, then none of the $\lambda_n$ is zero and we can write $c_n=b_n/\lambda_n$ and get the solution
$$\underline{x}=\sum_{n=1}^N\frac{b_n}{\lambda_n}\underline{v}_n$$
This means we can easily solve an linear equation if there is a basis consisting of eigenvectors of the matrix.
We want an analogy of this in solving linear ODEs.
Consider the differential operator
$$\mathcal Ly=-\frac{\mathrm d^2y}{\mathrm dx^2}$$
and suppose we want to solve $\mathcal Ly=f(x)$ for a function $f(x)$ subject to boundary conditions $y(0)=y(L)=0$.
The related eigenvalue problem is then $\mathcal L y_n=\lambda_ny_n$ with $y_n(0)=y_n(L)=0$ which has solutions
$$y_n(x)=\sin\frac{n\pi x}{L},\lambda_n=\left( \frac{n\pi}{L} \right)^2$$
So we will want to write
$$y(x)=\sum_{n=1}^\infty c_n\sin\frac{n\pi x}{L},f(x)=\sum_{n=1}^\infty b_n\sin\frac{n\pi x}{L},b_n=\frac{2}{L}\int_0^Lf(x)\sin\frac{n\pi x}{L}\,\mathrm dx$$
and ignore every convergence problem.
Then this substitution yields
$$\sum_{n=1}^\infty b_n\sin\frac{n\pi x}{L}=\mathcal Ly=-\frac{\mathrm d^2y}{\mathrm dx^2}\left( \sum_{n=1}^\infty c_n\sin\frac{n\pi x}{L} \right)=\sum_{n=1}^\infty c_n\left( \frac{n\pi}{L} \right)^2\sin\frac{n\pi x}{L}$$
Hence, $c_n=b_n(L/(n\pi))^2$ by orthogonality, so we can get a particular solution of the problem in the form
$$y(x)=\sum_{n=1}^\infty \frac{b_n}{\lambda_n}y_n$$
which is the analogy we wanted.
\begin{example}\label{odd_sq_fourier_ode}
    Let $L=1$ and set $f$ to be the odd square wave with $f(x)=1$ for $x\in[0,1)$.
    This has Fourier series
    $$4\sum_{m=1}^\infty\frac{\sin[(2m-1)\pi x]}{(2m-1)\pi}$$
    So the above discussion instantly yield a solution
    $$y(x)=\sum_{n=1}^\infty \frac{b_n}{\lambda_n}y_n=4\sum_{m=1}^\infty\frac{\sin[(2m-1)\pi x]}{[(2m-1)\pi]^3}$$
    which is the Fourier series of $y(x)=x(1-x)/2$ on $[0,1)$ extending as an odd periodic function with period $2$.\\
    Indeed, as one can verifty, if we integrate $\mathcal y=1$ directly with the appropriate boundary conditions, we can get basically the same solution.
\end{example}
\subsection{A Glimpse into Green's Functions}
Fix $L=1$ and consider an odd function $f$.
We have
\begin{align*}
    y(x)&=\sum_{n=1}^\infty\frac{b_n}{\lambda_n}\sin(\pi x)\\
    &=\sum_{n=1}^\infty\frac{2}{(n\pi)^2}\left(\int_0^1f(\xi)\sin(n\pi\xi)\,\mathrm d\xi\right)\sin(n\pi x)\\
    &=\int_0^12\sum_{n=1}^\infty\frac{\sin(n\pi x)\sin(n\pi\xi)}{(n\pi)^2}f(\xi)\,\mathrm d\xi\\
    &=\int_0^1G(x,\xi)f(\xi)\,\mathrm d\xi
\end{align*}
where
$$G(x,\xi)=2\sum_{n=1}^\infty\frac{\sin(n\pi x)\sin(n\pi\xi)}{(n\pi)^2}$$
But we have seen $G$ before!
It is exactly the general see-saw wave
$$G(x,\xi)=\begin{cases}
    x(1-\xi)\text{, for $x\in[0,\xi)$}\\
    \xi(1-x)\text{, for $x\in[\xi,1)$}
\end{cases}$$
This is the Green's function for this ODE $\mathcal Ly=f$.
One can actually solve this integral and get what we got in Example \ref{odd_sq_fourier_ode}.
    \section{Sturm-Liouville Theory}
\subsection{Review of Second-Order Linear ODEs}
For a general inhomogeneous ODE $\mathcal Ly=f(x)$ where
$$\mathcal Ly=\alpha(x)\frac{\mathrm d^2y}{\mathrm dx^2}+\beta(x)\frac{\mathrm dy}{\mathrm dx}+\gamma(x)y$$
In general, the homogeneous equation $\mathcal Ly=0$ has two linearly independent solutions $y_1,y_2$.
The complementary function $y_c(x)=Ay_1+By_2$ for constants $A,B$ is then the general solution to $\mathcal Ly=0$ by linearity.\\
If we can find a particular solution (aka particular integral) $y_p$ to $\mathcal Ly=f$, then $y_p+y_c=y_p+Ay_1+By_2$ for $A,B$ constants is the general solution to $\mathcal Ly=f$ again by linearity.
Two pieces of boundary data is then needed to determine the constants $A,B$.\\
There are several types of boundary conditions.
We sometimes get the Dirichlet condition of specifying the function's value at the endpoints, or the Neumann consitions of specifying the derivative's values at the endpoints.
Sometimes these two types of conditions are mixed.\\
The sort of conditions we often consider are homogeneous conditions, i.e. the function vanishes at the endpoints.
The reason of it is that it allows the superposition of solutions in a linear DE.
What if we come across a nonhomogeneous condition?
We can use the complementary solution to cancel stuff out.\\
Sometimes, we specify initial data of the function and its derivative as boundary conditions.\\
Another matter of interest is the general eigenvalue problem.
To solve $\mathcal Ly=f$ using eigenvalue decompositions like we did previously, we must first solve (subject to boundary conditions) the related eigenvalue problem
$$\mathcal Ly=\alpha(x)\frac{\mathrm d^2y}{\mathrm dx^2}+\beta(x)\frac{\mathrm dy}{\mathrm dx}+\gamma(x)y=-\lambda\rho(x)y$$
where $\rho$ is nonegative.
This form often occurs after seperation of variables in a PDE.
\subsection{Self-Adjoint Operators}
\begin{definition}
    For two functions $f,g:[a,b]\to\mathbb C$ we define their inner product to be
    $$\langle f,g\rangle=\int_a^bf^*(x)g(x)\,\mathrm dx$$
\end{definition}
We can guarantee to rewrite the original eigenvalue problem into the Sturm-Liouville form, i.e. $\mathcal Ly=\lambda wy$ where we are able to rewrite $\mathcal Ly=-(py^\prime)^\prime+qy$ and $w$ is a nonnegative wavefunction.
\footnote{The reason why there is a wavefunction there is just for convenience.}
How to convert a second order linear ODE to this form?
Simply multiply the diffential equation by an integrating factor $F$ that will be specified later and we can write
$$\frac{\mathrm d}{\mathrm dx}(F\alpha y^\prime)-F^\prime\alpha y^\prime-F\alpha^\prime y^\prime+F\beta y^\prime+F\gamma y=-\lambda F\rho y$$
So to eiminate the $y^\prime$ terms, we set
$$F(x)=\exp\left(\int\frac{\beta-\alpha^\prime}{\alpha}\,\mathrm dx\right)$$
which reduced the equation to
$$(F\alpha y^\prime)^\prime+F\gamma y=-\lambda F\rho y$$
Setting $p=F\alpha,q=F\gamma$ and $w=F\rho\ge 0$.
\begin{example}
    Consider the Hermite equation that appears in quantum mechanics
    $$y^{\prime\prime}-2xy^\prime+2ny=0$$
    Then $\alpha=1,\beta=-2x,\gamma=0,\lambda\rho=2n$, so the above procedure translates this to the Sturm-Liouville form
    $$\mathcal L=(-e^{-x^2}y^\prime)^\prime=2ne^{-x^2}y$$
\end{example}
\begin{definition}
    Let $\mathcal L:C\to C$ be an operator, where $C$ on a class of functions $[a,b]\to\mathbb C$ equipped with the inner product we defined previously.
    This operator $\mathcal L$ is self-adjoint if $\langle y_1,\mathcal Ly_2\rangle=\langle\mathcal Ly_1,y_2\rangle$ for any $y_1,y_2\in C$.
\end{definition}
If we let $\mathbb L$ be the operator in the Strum-Liouville form, then
\begin{align*}
    \langle y_1,\mathcal Ly_2\rangle-\langle\mathcal Ly_1,y_2\rangle&=\int_a^b[-y_1(py_2^\prime)^\prime+y_1qy_2+y_2(py_1^\prime)^\prime-y_2qy_1]\,\mathrm dx\\
    &=\int_a^b[-y_1(py_2^\prime)^\prime+y_2(py_1^\prime)^\prime]\,\mathrm dx\\
    &=\int_a^b[-(y_1(py_2^\prime)^\prime+y_1^\prime py_2^\prime)+(y_2(py_1^\prime)^\prime+y_2^\prime py_1^\prime)]\,\mathrm dx\\
    &=\int_a^b[-(py_1y_2^\prime)^\prime+(py_1^\prime y_2)^\prime]\,\mathrm dx\\
    &=[-py_1y_2^\prime+py_1^\prime y_2]_a^b
\end{align*}
So for this operator to be self-adjoint, we need some good enough boundary conditions so that enough stuff vanishes.
This includes homogeneous boundary condition $y(a)=y(b)=0$ or $y^\prime(a)=y^\prime(b)=0$ or mixed $y+ky^\prime=0$ etc..
We say a Sturm-Liouville problem is regular if the boundary conditions are homogeneous.
Periodic boundary conditions also work, where we can take $y(a)=y(b)$ and the derivatives are specified (or periodic) at the boundary.
There can also be singular points of this ODE, where $p(a)=p(b)=0$.
We can have combinations of above too.
\subsection{Properties of Self-Adjoint Operators}
\begin{definition}
    The inner product of $y_1,y_2:[a,b]\to\mathbb C$ with respect to weight $w:[a,b]\to\mathbb R_{\ge 0}$ is
    $$\langle f,g\rangle_w=\int_a^bwf^*g\,\mathrm dx=\langle wf,g\rangle=\langle f,wg\rangle$$
\end{definition}
Analogous to the finite dimensional case, we have
\begin{theorem}\label{self-adjoint}
    For a sufficiently nice self-adjoint operator $\mathcal L$ on a sufficiently nice space of functions:\\
    (a) Eigenvalues of $\mathcal L$ are real.\\
    (b) Eigenfunctions of it with different eigenvalues are orthogonal with respect to the weight $w$.\\
    (c) We can take the eigenfunctions as a set of basis for the function space, just like Fourier series.
\end{theorem}
\begin{proof}[Proof of (a)]
    1. If $\mathcal Ly=\lambda wy$, taking complex conjugate gives $\mathcal Ly^*=\lambda^*wy^*$.
    Hence as $\mathcal L$ is self-adjoint,
    $$0=\int_a^b(y^*\mathcal Ly-y\mathcal Ly^*)\,\mathrm dx=(\lambda-\lambda^*)\int_a^bw|y|^2\,\mathrm dx$$
    which means $\lambda=\lambda^*$, so $\lambda$ is real.
\end{proof}
If $\lambda$ is non-degenerate (simple), i.e. it has a one-dimensional eigenspace, then $y$ is guaranteed to be real.
Even if it has dimension $2$ (not more because the ODE is second order), we can still find two real functions as basis of the eigenspace.
Also, by considering $u\mathcal Lv-v\mathcal Lu=(-p(uv^\prime-u^\prime v))^\prime$, one can show that a regular Sturm-Liouville problem always has all eigenvalues simple.
\begin{proof}[Proof of (b)]
    Suppose $\mathcal Ly_m=\lambda_mwy_m$ and $\mathcal Ly_n=\lambda_nwy_n$, then
    $$0=\int_a^by_n\mathcal Ly_m-y_m\mathcal Ly_n\,\mathrm dx=(\lambda_m-\lambda_n)\int_a^bwy_ny_m\,\mathrm dx$$
    But $\lambda_m$ and $\lambda_n$ are distinct.
    The claim follows.
\end{proof}
As an aside, we do not really need the weight function in order to formulate Sturm-Liouville theory, since we can do the transformation $\tilde{y}=\sqrt{w}y$ and replace $\mathcal Ly$ by $(1/\sqrt{w})\mathcal L(\tilde{y}/\sqrt{w})$.
Yet the analytic property is generally simpler if we keep $w$.\\
What?
How about (c), you say?
Bold of you to assume we'll prove it.
We are just gonna take it (and several other properties we want it to have) as truth and do stuff with this idea.
\subsection{Eigenfunction Expansions}
So basically we just want to find an expansion
$$f=\sum_{n=1}^\infty a_ny_n$$
where $y_n$ is a set of eigenfunctions of some self-adjoint operator.
Theorem \ref{self-adjoint}(c) shows that we can do it.
To find the coefficients $a_n$, we can they use the orthogonality to get
$$\int_a^bwy_mf\,\mathrm dx=\sum_{n=1}^\infty a_n\int_a^bwy_ny_m\,\mathrm dx=a_m\int_a^bwy_n^2\,\mathrm dx$$
So
$$a_n=\left(\int_a^bwy_nf\,\mathrm dx\middle)\right/\left(\int_a^bwy_n^2\,\mathrm dx\right)$$
It's a common practice not to normalise the eigenfunctions as it is not really always clean.
Of course, if we want, we can always write down
$$Y_n=y_n\left/\sqrt{\int_a^bwy_n^2\,\mathrm dx}\right.$$
So we can get rid of the denominator in $a_n$ and the coefficients will have the expression
$$A_n=\int_a^bwy_nf\,\mathrm dx=a_n\int_a^bwy_n^2\,\mathrm dx$$
but it isn't that useful and can cause some messiness.
\begin{example}
    Recall the particular operator already in Sturm-Liouville form $\mathcal Ly=y^{\prime\prime}$, then (with appropriate boundary conditions) we can easily get the eigenvalues $\lambda_n=(n\pi/L)^2$ and eigenfunctions being the trigonometrics.
    This just reproduces the Fourier series.
\end{example}
\subsection{Completeness and Parseval's Identity}
We expand
\begin{align*}
    0&=\int_a^bw\left( f(x)-\sum_{n=1}^\infty a_ny_n \right)^2\,\mathrm dx\\
    &=\int_a^bw\left( f^2-2f\sum_{n=1}^\infty a_ny_n+\sum_{n=1}^\infty a_n^2y_n^2 \right)\,\mathrm dx\\
    &=\int_a^bwf^2\,\mathrm dx-\sum_{n=1}^\infty a_n^2\int_a^bwy_n^2\,\mathrm dx
\end{align*}
Hence we have
$$\int_a^bwf^2\,\mathrm dx=\sum_{n=1}^\infty a_n^2\int_a^bwy_n^2\,\mathrm dx=\sum_{n=1}^\infty A_n^2$$
which Parseval's identity in this general case.
Easily our previous Parseval's Theorem on Fourier series is a special case.\\
If some of the eigenfunctions are missing from the series, then this gives
$$\int_a^bwf^2\,\mathrm dx\ge\sum_{n=1}^\infty A_n^2$$
This is known as Bessel's Inequality.\\
Consider the partial sums $\sum_{n\le N}a_ny_n$, then we shall have $S_N\to f$ as $N\to\infty$ where we would like the style of convergence to be
$$\epsilon_N=\int_a^bw[f(x)-S_N(x)]^2\,\mathrm dx\to 0,N\to\infty$$
An interesting question is that, while we know (maybe) the series converges as we want, if we truncate the sequence in some $N$, would the coefficients $\{a_n\}_{n\le N}$ provide the best approxmation (with respect to the error defined in this way) of that particular partial sum, or a different set of partial coefficient will yield a better result?
To answer this, we evaluate
$$\frac{\partial\epsilon_N}{\partial a_n}=-2\int_a^bwy_n\left( f-\sum_{k=1}^Na_ky_k \right)\,\mathrm dx=-2\int_a^bwfy_n-a_nwy_n^2\,\mathrm dx$$
which is zero when $a_n$ is of the expression we got earlier.
We can see it is indeed a minimum by observing that
$$\frac{\partial^2\epsilon_N}{\partial a_n^2}=2\int_a^bwy_n^2\,\mathrm dx\ge 0$$
This answers our question.
\subsection{Legendre's Equation}
Take the usual spherical polar coordinate
$$\begin{cases}
    x=r\sin\theta\cos\phi\\
    y=r\sin\theta\sin\phi\\
    z=r\cos\theta
\end{cases}$$
where Laplace's equation $\nabla^2 u=0$ translates to
$$\frac{1}{r^2}\frac{\partial}{\partial r}\left( r^2\frac{\partial u}{\partial r} \right)+\frac{1}{r^2\sin\theta}\frac{\partial}{\partial\theta}\left( \sin\theta\frac{\partial u}{\partial\theta} \right)+\frac{1}{r^2\sin^2\theta}\frac{\partial^2u}{\partial\phi^2}=0$$
Seperation of variables $u=R(r)\Theta(\theta)\Phi(\phi)$ then gives
$$\frac{1}{\sin\theta}(\Theta^\prime\sin\theta)^\prime+\left( K-\frac{m^2}{\sin^2\theta} \right)\Theta=0$$
where $K,m$ are constants which essentially makes it an eigenvalue problem.
Now the transformation $x=\cos\theta\in[-1,1]$ and renaming $\Theta$ as $y$ then gives Legendre's Equation
$$(1-x^2)y^{\prime\prime}-2xy^\prime+\lambda y=0$$
where $\lambda$ is a constant which is again intepreted as an eigenvalue.
This is already in Strum-Liouville form by taking $p=1-x^2,q=0,w=1$.
Now $p=1-x^2$ vanishes at the boundary $\pm 1$, so this equation has to be self-adjoint.
We assume that $y$ is bounded near the boundary.\\
We now seek a power series solution to the problem.
If we set
$$y=\sum_{n=0}^\infty c_nx^n$$
Then substitution gives
$$(n+2)(n+1)c_{n+2}-n(n-1)c_n-2nc_n+\lambda c_n=0\implies c_{n+2}=\frac{n(n+1)-\lambda}{(n+1)(n+2)}c_n$$
The iteration steps by $2$, so we get two linearly independent solutions
\begin{align*}
    y_{\text{even}}&=c_0\left( 1+\frac{-\lambda}{2!}x^2+\frac{(6-\lambda)(-\lambda)}{4!}x^4+\cdots \right)\\
    y_{\text{odd}}&=c_1\left( x+\frac{2-\lambda}{3!}x^3+\cdots \right)
\end{align*}
Note that $c_{n+2}/c_n\to 1$, so the both series has radius of convergence $1$ but they diverges at $x=\pm 1$.
However, this is not the end of the world!
These series may not be infinite.
If $\lambda=l(l+1)$ for some $l\in\mathbb N$, then one of these two series will terminate and give a polynomial solution.
These polynomials are called Legendre polynomials $P_l(x)$ which are eigenfunctions of the Legendre equation.
Conventionally we normalise $P_l$ by requiring $P_l(1)=1$.
One can check that this restricts $P_l([-1,1])\subset [-1,1]$ and $|P_l(-1)|=1$.
By calculation we have
$$P_0(x)=1,P_1(x)=x,P_2(x)=\frac{3x^2-1}{2},P_3(x)=\frac{5x^3-3x}{2},\ldots$$
We easily observe that $P_l$ has $l$ roots in $[-1,1]$, also $P_l$ is odd if $l$ is odd, and even when $l$ is even.
By orthogonality and some calculation,
$$\forall n\neq m,\int_{-1}^1P_nP_m\,\mathrm dx=0,\int_{-1}^1P_n^2\,\mathrm dx=\frac{2}{2n+1}$$
There are several other ways to characterise Legendre polynomails.
One can prove that we have
$$P_n(x)=\frac{1}{2^nn!}\frac{\mathrm d^n}{\mathrm dx^n}(x^2-1)^n,\sum_{n=0}^\infty P_n(x)t^n=\frac{1}{\sqrt{1-2xt+t^2}}$$
We also have the recursions
$$l(l+1)P_{l+1}(x)=(2l+1)xP_l(x)-lP_{l-1}(x),(2l+1)P_l(x)=\frac{\mathrm d}{\mathrm dx}(P_{l+1}(x)-P_{l-1}(x))$$
If we take these $P_l$ as a set of eigenfunctions, then any well-behaved $f$ on $[-1,1]$ can be expressed as
$$f(x)=\sum_{l=0}^\infty a_lP_l(x),a_l=\frac{2l+1}{2}\int_{-1}^1f(x)P_l(x)\,\mathrm dx$$
\begin{example}
    We can verify that $f(x)=(15x^2-3)/2=P_0(x)+5P_2(x)$.
\end{example}
\begin{example}
    The odd equare wave with $f([0,1))=\{1\}$ has the expansion
    $$\sum_{m=1}^\infty (P_{2m}(0)-P_{2m+2}(0))P_{2m+1}(x)$$
\end{example}
\subsection{Inhomogeneous ODEs}
Consider the ODE $\mathcal Ly=f=wF$ (with homogeneous boundary conditions so that $\mathcal L$ is self-adjoint) where $w\ge 0$ is our wavefunction.
Given eigenfunctions $\{y_n\}$ satisfying $\mathcal Ly_n=\lambda_nwy_n$ for eigenvalues $\{\lambda_n\}$.
We now try to find a solution in the form $y=\sum_nc_ny_n$.
To do this, we expand $F=\sum_na_ny_n$ where
$$a_n=\left( \int_a^bwFy_n\,\mathrm dx\middle) \right/\left( \int_a^bwy_n^2\,\mathrm dx \right)$$
then
$$w\sum_na_ny_n=wF=\mathcal Ly=\mathcal L\sum_nc_ny_n=\sum_nc_n\mathcal Ly=w\sum_nc_n\lambda_ny_n$$
So take
$$y=\sum_n\frac{a_n}{\lambda_n}y_n$$
gives a particular solution, assuming everything is well-defined and converges nicely enough.\\
An aside:
The driving force $F$ sometimes induces a linear response term $\tilde\lambda wy$, so the solution is $\mathcal Ly-\tilde{\lambda}wy=f$.
Then our particular solution can be
$$y=\sum_{\lambda_n\neq\tilde{\lambda}}\frac{a_n}{\lambda_n-\tilde\lambda}y_n$$
Now back to theme.
If we expand the expression of $a_n$, we can get
\begin{align*}
    y(x)&=\sum_{n=1}^\infty\frac{a_n}{\lambda_n}y_n(x)\\
    &=\sum_{n=1}^\infty\frac{y_n(x)}{\lambda_nN_n}\int_a^bw(\xi)F(\xi)y_n(\xi)\,\mathrm d\xi,N_n=\int_a^bwy_n^2\,\mathrm dx\\
    &=\int_a^b\left( \sum_{n=1}^\infty\frac{y_n(x)y_n(\xi)}{\lambda_nN_n} \right)w(\xi)F(\xi)\,\mathrm d\xi\\
    &=\int_a^bG(x,\xi)f(\xi)\,\mathrm d\xi,G(x,\xi)=\sum_{n=1}^\infty\frac{y_n(x)y_n(\xi)}{\lambda_nN_n}
\end{align*}
This $G(x,\xi)$ is called the Green's function of this particular eigenvalue problem of that self-adjint operator.
Worth noting that $G$ does not depend on the forcing term $f$.
The Green's function also induces
$$\mathcal L^-(\phi)=\int_a^bG(x,\xi)\phi(\xi)\,\mathrm d\xi$$
which can be taken as kind of an inverse operator to $\mathcal L$ since $\mathcal L(\mathcal L^-(f))=f$.
    \section{The Wave Equation}
\subsection{Waves on an Elastic String}
Consider small displacements $y(x,t)$ on a stretched string with fixed ends at $x=0$ and $x=L$, that is with boundary conditions $y(0,t)=y(L,t)=0$.
We want to determine the string's motion subject to initial conditions
$$y(x,0)=p(x),\frac{\partial y}{\partial t}(x,0)=q(x)$$
We want to derive its equation of motion.
We try to obtain a differential equation by balancing the forces on string segment $x,x+\delta x$ and taking $\delta x\to 0$.
By resolving in $x$ direction we get that the tension $T$ on the string is independent of $x$.
By resolving in $y$ direction we arrive at
$$(\mu\delta x)\frac{\partial^2y}{\partial t^2}=T\frac{\partial^2y}{\partial x^2}\delta x-g\mu\delta x$$
where $\mu$ is the mass per unit length (aka linear mass density).
Write $c=\sqrt{T/\mu}$ the wave speed and assume the acceleration due to gravity is negligible, then this equation is just
$$\frac{1}{c^2}\frac{\partial^2y}{\partial t^2}=\frac{\partial^2y}{\partial x^2}$$
which is called the one-dimensional wave equation.
\subsection{Seperation of Variables}
Our first attempt at a solution is to guess a solution of seperable form, that is $y(x,t)=X(x)T(t)$, then substitution gives
$$\frac{1}{c^2}X\ddot{T}=X^{\prime\prime}T\implies \frac{1}{c^2}\frac{\ddot{T}}{T}=\frac{X^{\prime\prime}}{X}$$
The left hand side depends only on $t$ and the right hand side depends only on $x$, so they can only equal if both sides equal a constant $-\lambda$ (called the seperation constant), then we get
$$\begin{cases}
    X^{\prime\prime}+\lambda X=0\\
    \ddot{T}+\lambda c^2T=0
\end{cases}$$
Such nice things don't always happen, but we are glad it happened in this particular case.
\footnote{Whether we can do it depends on the existence of symmetry in the boundary conditions, which is probably not going to be discussed here.}
So we reduced a PDE to two independent ODEs, which we know how to solve.
\subsection{Boundary Conditions and Normal Modes}
There are a few possibilities depending on the sign of $\lambda$.\\
If $\lambda<0$, then taking $\chi^2=-\lambda$ gives the general solution
$$X(x)=Ae^{\chi x}+Be^{-\chi x}=\tilde{A}\cosh(\chi x)+\tilde{B}\sinh(\chi x)$$
where $A,B,\tilde{A},\tilde{B}$ are constants.
But the boundary conditions $X(0)=X(L)=0$ would yield $X=0$ everywhere, so this is just the trivial solution.\\
Now if $\lambda=0$, then $X(x)=Ax+B$ where $A,B$ are constants, but again we must have $A=B=0$ as $X(0)=X(L)=0$.
Trivial solution again.\\
What is left is $\lambda>0$, so $X(x)=A\cos(\sqrt{\lambda}x)+B\sin(\sqrt{\lambda}x)$, then the boundary condition $X(0)=X(L)=0$ gives the family of solutions
$$\lambda_n=\left( \frac{n\pi}{L} \right)^2,X_n(x)=B_n\sin\frac{n\pi x}{L}$$
where $B_n$ are constants.
It is not hard to observe that they are just our familiar Fourier eigenvalues and eigenfunctions.
These are called the normal modes of the system since its spacial shape in $x$ does not change in time but the amplitudes may vary.
\footnote{Well, duh!}
The case $n=1$ is called the fundamental mode.
A plot shows the modes are simply just the patterns of simple vibrations we expect.
\subsection{Initial Conditions and Temporal Solutions}
Substituting $\lambda_n=(n\pi/L)^2$ into the time ODE gives
$$T_n(t)=C_n\cos\frac{n\pi ct}{L}+D_n\sin\frac{n\pi ct}{L}$$
where $C_n,D_n$ are constants.
So we obtain the family of solutions
$$y_n(x,t)=T_n(t)X_n(x)=\left( C_n\cos\frac{n\pi ct}{L}+D_n\sin\frac{n\pi ct}{L} \right)\sin\frac{n\pi x}{L}$$
As the system we are trying to deal with is homogeneous and linear, we have the superposition principle, so the general solution is
$$y(x,t)=\sum_{n=1}^\infty\left( C_n\cos\frac{n\pi ct}{L}+D_n\sin\frac{n\pi ct}{L} \right)\sin\frac{n\pi x}{L}$$
assuming it converges sufficiently well.
This satisfies the boundary conditions as $X_n$ does.
Substituting into the initial condition, we get
$$p(x)=\sum_{n=1}^\infty C_n\sin\frac{n\pi x}{L},q(x)=\sum_{n=1}^\infty\frac{n\pi c}{L}D_n\sin\frac{n\pi x}{L}$$
which allows us to find $C_n,D_n$ by expanding $p,q$ as Fourier sine series.
In particular,
$$C_n=\frac{2}{L}\int_0^Lp(x)\sin\frac{n\pi x}{L}\,\mathrm dx,D_n=\frac{2}{n\pi c}\int_0^Lq(x)\sin\frac{n\pi x}{L}\,\mathrm dx$$
This is possible (well, assuming everything), so we have found a particular solution to the system.
\begin{example}
    We pluck string at $x=\xi$, which requires
    $$y(x,0)=p(x)=\begin{cases}
        x(1-\xi)\text{, for $0\le x\le \xi$}\\
        \xi(1-x)\text{, for $\xi\le x\le 1$}
    \end{cases},\frac{\partial y}{\partial t}(x,0)=q(x)=0$$
    Then with our formulas we obtain
    $$C_n=\frac{2\sin(n\pi\xi)}{(n\pi)^2},D_n=0\implies y(x,t)=\sum_{n=1}^\infty\frac{2}{(n\pi)^2}\sin(n\pi\xi)\sin(n\pi x)\cos(n\pi ct)$$
    Of course, this case happens to be the way of making sound on a string instrument, where guitar has $\xi\in [1/4,1/3]$ and violin has $\xi\approx 1/7$.
\end{example}
By the usual trigonometric identities, the general solution we found earlier becomes $y(x,t)=f(x-ct)+g(x+ct)$ where
$$f(x-ct)=\frac{1}{2}\sum_{n=1}^\infty\left(C_n\sin\frac{n\pi(x-ct)}{L}+D_n\cos\frac{n\pi(x-ct)}{L}\right)$$
and
$$g(x+ct)=\frac{1}{2}\sum_{n=1}^\infty\left(C_n\sin\frac{n\pi(x+ct)}{L}-D_n\cos\frac{n\pi(x+ct)}{L}\right)$$
So the standing wave solution can be interpreted as a superposition of a right-moving wave (along $x-ct=\eta$, $\eta$ constant) and a left-moving wave (along $x+ct=\xi$, $\xi$ constant).
We can generalise this idea later.
\begin{example}
    In the special case where $q=0$, we have $f=g=p/2$, so
    $$y(x,t)=\frac{p(x-ct)+p(x+ct)}{2}$$
\end{example}
\subsection{Oscillation Energy}
A vibrating string has kinetic energy due to the motion of the particles in the string.
This is given by $mv^2/2$.
So the total kinetic energy on the string would be the integral
$$\operatorname{KE}=\frac{1}{2}\int_0^L\left(\frac{\partial y}{\partial t}\right)^2\mu\,\mathrm dx$$
where $\mu$ is the mass per unit length.
The (elastic) potential energy due to streching $\Delta x$ is then
$$\operatorname{PE}=T\int_0^L\left( \sqrt{1+\left( \frac{\partial y}{\partial x} \right)^2} -1\right)\,\mathrm dx\approx\frac{1}{2}T\int_0^L\left( \frac{\partial y}{\partial x} \right)^2\,\mathrm dx$$
for small $|\partial y/\partial x|$.
So the total summed energy of the string is then, via $c^2=T/\mu$,
$$E=\frac{1}{2}\mu\int_0^L\left( \left( \frac{\partial y}{\partial t} \right)^2+c^2\left(\frac{\partial y}{\partial x}\right)^2 \right)\,\mathrm dx$$
So by substituting our generalisation and using orthogonality,
$$E=\frac{1}{2}\mu\sum_{n=1}^\infty (A_n+B_n)$$
where
$$A_n=\int_0^L \left( \frac{n\pi c}{L}C_n\sin\frac{n\pi ct}{L}+ \frac{n\pi c}{L}D_n\cos\frac{n\pi ct}{L}\right)^2\sin^2\frac{n\pi x}{L}\,\mathrm dx$$
and
$$B_n=\int_0^Lc^2\left( C_n\cos\frac{n\pi ct}{L}+ D_n\sin\frac{n\pi ct}{L}\right)^2\frac{n^2\pi^2}{L^2}\cos^2\frac{n\pi x}{L}\,\mathrm dx$$
Simplifying this mess gives
$$E=\frac{1}{4}\mu\sum_{n=1}^\infty\frac{n^2\pi^2c^2}{L}(C_n^2+D_n^2)$$
which can be interpreted as the sum of the energy of all normal modes.
Also, this is constant, so it is conserved in time.
\subsection{Wave Reflection and Transmission}
Recall the travelling wave solution along the $x\pm ct$ directions.
We want to further develop this idea
\begin{definition}
    A simple harmonic travelling wave is defined as
    $$t=\operatorname{Re}(Ae^{i\omega (t-x/c)})=|A|\cos(\omega(t-x/c)+\phi)$$
    where $\phi=\arg A$ is the phase and $2\pi c/\omega$ is the wavelength.
\end{definition}
Sometimes we will just assume the $\operatorname{Re}$ is there without writing it out explicitly.\\
Consider a density discontinuity on the string at $x=0$ with
$$\mu=\begin{cases}
    \mu_-\text{, for $x<0$}\implies c_-=\sqrt{T/\mu_-}\\
    \mu_+\text{, for $x>0$}\implies c_+=\sqrt{T/\mu_+}
\end{cases}$$
So a wave $Ae^{i\omega (t-x/c_-)}$ approaching $0$ from left will break down to two parts:
The reflected wave $Be^{i\omega (t+x/c_-)}$ and the transmitted wave $De^{i\omega (t-x/c_+)}$.
The continuity condition on $y$ gives $A+B=D$ and by balancing the forces
$$T\left.\frac{\partial y}{\partial x}\right|_{x=0_-}=T\left.\frac{\partial y}{\partial x}\right|_{x=0_+}$$
(which is basically the continuity condition on $\partial y/\partial x$) it gives $2A=D(c_++c_-)/c_+$.
Combining them all gives
$$D=\frac{2c_+}{c_-+c_+},B=\frac{c_+-c_-}{c_-+c_+}A$$
In general, it is possible to have different phase shifts $\phi$.\\
Now, if $c_+=c_-$, then $D=A$ and $B=0$, so there is no reflection, which is intuitive.
If we have the Dirichlet boundary conditions $\mu_+/\mu_-\to\infty$ (interpreted as a fixed end $y=0$ at $x=0$), then $c_+/c_-\to 0$, so $D=0$ and $B=-A$ where we get total reflection (with opposite phase $\phi=\pi$).
This is also what is expected.
If we have the Neumann boundary conditions $\mu_+/\mu_-\to 0$ (interpreted as an extremely light string in $x>0$), then $c_+/c_-\to\infty$, so $D=2A$ and $B=A$, so we get total reflection with same phase $\phi=0$.
\subsection{Wave Equation in the Plane Polar Coordinates}
The wave equation in two dimensions is
$$\frac{1}{c^2}\frac{\partial^2u}{\partial t^2}=\nabla^2u$$
Under plane polar coordinates $u=u(r,\theta,t)$, we impose the boundary condition $u(1,\theta,t)=0$ for any $\theta,t$ (interpreted as a fixed rim) and initial conditions
$$u(r,\theta,0)=\phi(r,\theta),\frac{\partial u}{\partial t}(r,\theta,0)=\psi(r,\theta)$$
We use seperation of variables again.
If we substitute $u(r,\theta,t)=T(t)V(r,\theta)$, then we obtain the decoupled system
$$\begin{cases}
    \ddot{T}+\lambda c^2T=0\\
    \nabla^2V+\lambda V=0
\end{cases}$$
where $\lambda$ is a seperation constant.
Note that in plane polar,
$$\nabla^2V=\frac{\partial^2V}{\partial r^2}+\frac{1}{r}\frac{\partial V}{\partial r}+\frac{1}{r^2}\frac{\partial^2V}{\partial\theta^2}+\lambda V=0$$
We seperate the variables further by writing $V(r,\theta)=R(r)\Theta(\theta)$.
This gives
$$\begin{cases}
    \Theta^{\prime\prime}+\mu\Theta=0\\
    r^2R^{\prime\prime}+rR^\prime+(\lambda r^2-\mu)R=0
\end{cases}$$
where $\mu$ is again a seperation constant.
Assuming $\mu>0$.
Solving for $\Theta$ subject to $\Theta(0)=\Theta(\pi)$ gives our old friend
$$\Theta_m(\theta)=A_m\cos(m\theta)+B_m\sin(m\theta),m\in\mathbb Z_{>0}$$
with $\mu_m=m^2$.
For $R$, we divide the equation by $r$ and transform it into Sturm-Liouville form
$$\frac{\mathrm d}{\mathrm dr}(rR^\prime)-\frac{m^2}{r}=-\lambda rR$$
Note that the boundary condition means we only care about $r\in[0,1]$.
The boundary condition are self-adjoint, which is convenient.
\subsection{Bessel's Equation}
Substitute $z=\sqrt{\lambda}r$ gives
$$z^2\frac{\mathrm d^2R}{\mathrm dz^2}+z\frac{\mathrm dR}{\mathrm dz}+(z^2-m^2)R=0\iff (zR^\prime)^\prime+\left( z-\frac{m^2}{z} \right)R=0$$
Apparently $0$ is a regular singular point, so we substitute the power series
$$R=z^p\sum_{n=0}^\infty a_nz^n$$
which gives
$$\sum_{n=0}^\infty (a_n(n+p)(n+p-1)z^{n+p}+(n+p)z^{n+p}+z^{n+p+2}+m^2z^{n+p})=0$$
The indivial equation is then $p^2-m^2=0$, so $p=\pm m$.
For $p=m$, it is called the regular solution (otherwise you get a singular point at $0$).
Here we have the recurrence
$$(n+m)^2a_n+a_{n-2}-m^2a_n=0\implies a_n=-\frac{1}{n(n+2m)}a_{n-2}$$
This gives the even series with solutions
$$a_{2n}=a_0\frac{(-1)^n}{2^{2n}n!(n+m)(n+m-1)\cdots (m+1)}$$
Conveniently we set $a_0=1/(2^mm!)$ which gives the Bessel function
$$J_m(z)=\left( \frac{z}{2} \right)^m\sum_{n=0}^\infty\frac{(-1)^n}{n!(n+m)!}\left( \frac{z}{2} \right)^{2n}$$
Actually, if we set $y=\sqrt{z}R$, then we will obtain
$$y^{\prime\prime}+y\left( 1+\frac{1}{4z}-\frac{m^2}{z^2} \right)=0$$
which, as $z\to\infty$, gives the approximation $y^{\prime\prime}\approx -y$ which has (approximated) solutions $R\approx z^{-1/2}(A\cos z+B\sin z)$ for $A,B$ constants.\\
Back to Bessel's function.
In fact, when $m=\nu\notin\mathbb Z$, this power solution also works but replace $(n+m)!$ by $\Gamma(n+\nu+1)$.
The second solution with $p=-m$ is known as the Neumann functions (or Bessel functions of second kind) which satisfies
$$Y_m(z)=\lim_{\nu\to m}\frac{J_\nu(z)\cos(\nu\pi)-J_{-\nu}(z)}{\sin(\nu\pi)}$$
As one can verify, there are a number of identities associated with Bessel functions, e.g. we have $(z^mJ_m(z))^\prime=z^mJ_{m-1}(z)$.
This also implies
$$\begin{cases}
    J_m^\prime(z)+mJ_m(z)/z=J_{m-1}(z)\\
    J_{m-1}(z)+J_{m+1}(z)=2mJ_m(z)/z\\
    2J_m^\prime(z)=J_{m-1}(z)-J_{m+1}(z)
\end{cases}$$
Naturally, we want to study the aymptopic behaviour of $J_m$ and $Y_m$.
As $z\to 0$, easily $J_0(z)\to 1$, $J_m(z)\sim (z/2)^m/m!$ and $Y_0(z)\to 2\log(z/2)/\pi$, $Y_m(z)=-(m-1)!(2/z)^m/\pi$.
So $0$ is a singularity of $Y_m$.\\
For large $z\to\infty$, $J_m,Y_m$ converges to $0$ oscillatorily, i.e.
$$J_m(z)\sim\sqrt{\frac{2}{\pi z}}\cos\left( z-\frac{m\pi}{2}-\frac{\pi}{4} \right),Y_m(z)\sim\sqrt{\frac{2}{\pi z}}\sin\left( z-\frac{m\pi}{2}-\frac{\pi}{4} \right)$$
This hints that we might want to take a look at the infinitely many zeros in $\mathbb R_{>0}$ of Bessel function.
Let $j_{m,n}$ to be the $n^{th}$ positive zero of $J_m$.
So the asymptopic formula above shows approximately $j_{m,n}\approx \tilde{j}_{m,n}=n\pi +m\pi/2-\pi/4$ with accuracy $|(j_{m,n}-\tilde{j}_{m,n})/j_{m,n}|<1/(10n)$ for $n>m^2/2$.
A few of the actual values are below:
$$j_{0,1}\approx 2.405,j_{0,2}\approx 5.520,j_{0,3}\approx 8.653$$
It is a fun activity for the reader to try and draw $J_m$.
\subsection{A Vibrating Drum}
So the radial solutions become
$$R_m(z)=R_m(\sqrt{\lambda}r)=AJ_m(\sqrt{\lambda}r)+BY_m(\sqrt{\lambda} r)$$
For $A,B$ constants.
But we want the solution to be bounded near zero which would mean $B=0$.
We also need $R(1)=0$, therefore $\lambda_{m,n}=j_{m,n}^2$ are the eigenvalues.
Therefore the spacial solutions are
$$V_{m,n}(r,\theta)=\Theta_m(\theta)R_{m,n}(\sqrt{\lambda_{m,n}}r)=(A_{m,n}\cos(m\theta)+B_{m,n}\sin(m\theta))J_m(j_{m,n}r)$$
Putting these eigenvalues to the temporal equation then shows $T_{m,n}$ are just linear combinations of $\cos(j_{m,n}ct)$ and $\sin(j_{m,n}ct)$.
Putting everything together we get $u(r,\theta,t)=A+B+C$ where
$$A=\sum_{n=1}^\infty J_0(j_{0n})r(A_{0n}\cos(j_{0n}ct)+C_{0n}\sin(j_{0n}ct))$$
$$B=\sum_{m=1}^\infty\sum_{n=1}^\infty J_m(j_{m,n}r)(A_{m,n}\cos(m\theta)+B_{m,n}\sin(m\theta))\cos(j_{m,n}ct)$$
$$C=\sum_{m=1}^\infty\sum_{n=1}^\infty J_m(j_{m,n}r)(C_{m,n}\cos(m\theta)+D_{m,n}\sin(m\theta))\sin(j_{m,n}ct)$$
We still have to impose the initial conditions.
$$\phi(r,\theta)=u(r,\theta,0)=\sum_{m=0}^\infty\sum_{n=1}^\infty J_m(j_{m,n}r)(A_{m,n}\cos(m\theta)+B_{m,n}\sin(m\theta))$$
$$\psi(r,\theta)=\frac{\partial u}{\partial t}(r,\theta,0)=\sum_{m=0}^\infty\sum_{n=1}^\infty j_{mn}cJ_m(j_{m,n}r)(C_{m,n}\cos(m\theta)+D_{m,n}\sin(m\theta))$$
As usual, we find the coefficient by orthogonality of these eigenfunctions.
We do already know that they are orthogonal, what we really need is the normalisation constant.
Some calculation then reveals
$$\int_0^1J_m(j_{m,n}r)J_m(j_{m,k}r)r\,\mathrm dr=\frac{1}{2}(J_m^\prime(j_{m,n}))\delta_{nk}=\frac{1}{2}J_{m+1}(j_{m,n})^2\delta_{n,k}$$
Therefore for $p>0$,
$$A_{pq}=\left(\int_0^{2\pi}\int_0^1\cos(p\theta)J_p(j_{pq}r)\phi(r,\theta)r\,\mathrm dr\mathrm d\theta\right)\left( \frac{\pi+\delta_{0p}\pi}{2}J_{p+1}(j_{pq})^2 \right)^{-1}$$
We can obtain $B_{m,n},C_{m,n},D_{m,n}$ in similar ways.
\begin{example}
    Consider $\phi=1-r^2$, so we have $\forall m,B_{m,n}=0$ and $\forall m\neq 0,A_{m,n}=0$.
    We also set $\psi=0$, which means $C_{m,n}=D_{m,n}=0$ for all $m,n$.
    By calculations,
    $$A_{0,n}=\frac{2}{J_1(j_{0,n})^2}\frac{J_2(j_{0,n})}{j_{0,n}^2}\approx\frac{J_2(j_{0,n})}{n}$$
    for large $n$.
    So the solution is
    $$u(r,\theta,t)=\sum_{n=1}^\infty\frac{2}{J_1(j_{0,n})^2}\frac{J_2(j_{0,n})}{j_{0,n}^2}J_0(j_{0,n}r)\cos(j_{0,n}ct)$$
    So the fundamental frequency is $\omega=j_{0,1}c(2/d)\approx 4.8c/d$ which is higher than the value for a string, whose value is approximately $77\%$ of the figure for the drum.\\
    A sketch of the nodal lines will then show that the solution is pretty close to our intuition.
\end{example}
    \section{The Diffusion Equation}
\subsection{Physical Origin}
We want to understand physical phenomena that ``diffuses'' due to spatial gradient.
An early example was Fick's Law $\underline{J}=-D\nabla c$ where $J$ is the flux, $c$ is the concentration and $D$ is the diffusion coefficient.
For heat flow, we also have Fourier's Law saying $\underline{q}=-k\nabla\Theta$ where $\underline{q}$ is the heat flux, $k$ is the thermal conductivity and $\Theta$ is the temperature.\\
In a volume $V$, the overall heat energy $Q$ is
$$Q=\int_V c_V\rho\Theta\,\mathrm dV$$
where $c_V$ is the specific heat capacity of the matertial of the volume $V$, and $\rho_V$ is the mass density.
Te rate of change of it would then be, making use of Fourier's law,
$$\frac{\mathrm dQ}{\mathrm dt}=\int_Vc_V\rho\frac{\partial\Theta}{\partial t}\,\mathrm dV$$
On the other hand, integrating Fick's Law over $S=\partial V$,
$$-\frac{\mathrm dQ}{\mathrm dt}=\int_S\underline{q}\cdot\underline{\hat{n}}\,\mathrm dS=\int_S(-k\nabla\Theta)\cdot\underline{\hat{n}}\,\mathrm dS=\int_V(-k\nabla^2\Theta)\,\mathrm dV$$
by Fourier's Law.
Therefore,
$$\int_Vc_V\rho\frac{\partial\Theta}{\partial t}-k\nabla^2\Theta\,\mathrm dV=0$$
for all volumn $V$.
So the integrand must vanish everywhere (assuming it is continuous), which gives
$$\frac{\partial\Theta}{\partial t}-D\nabla^2\Theta=0$$
where $D=k/(c_V\rho)$.
This is known as the diffusion equation.\\
We can also derive this from a more fundamental point of view.
Consider gas particles diffuse by scattering.
So for every small time change $\Delta t$ in time a particle moves for a distance $\xi$ with probability (PDF) $p(\xi)$.
Let $\langle X\rangle$ be the mean of $X$.
We assume $\langle\xi\rangle=0$.
Suppose the PDF after $N\Delta t$ steps is $P_{N\Delta t}(x)$, then for $(N+1)\Delta t$ step,
\begin{align*}
    P_{(N+1)\Delta t}(x)&=\int_{-\infty}^\infty p(\xi)P_{N\Delta t}(x-\xi)\,\mathrm d\xi\\
    &\approx \int_{-\infty}^\infty p(\xi)\left(P_{N\Delta t}(x)-\xi P_{N\Delta t}^\prime(x)+\frac{\xi^2}{2}P_{N\Delta t}^{\prime\prime}(x)\right)\,\mathrm d\xi\\
    &=P_{N\Delta t}(x)+P_{N\Delta t}^{\prime\prime}(x)\frac{\langle\xi^2\rangle}{2}
\end{align*}
Identify $P_{N\Delta t}(x)=P(x,N\Delta t)$, then we have
$$P(x,(N+1)\Delta t)-P(x,N\Delta t)=\frac{\partial^2P}{\partial x^2}(x,N\Delta t)\frac{\langle\xi^2\rangle}{2}$$
By some probabilistic argument, $\langle\xi^2\rangle\propto\Delta t$, so this gives
$$\frac{\partial P}{\partial t}=D\frac{\partial^2P}{\partial x^2}$$
for a constant $D$.
\subsection{Similarity Solution}
The characteristic relation between variance and time suggests that we may start seeking a solution in terms of the dimensionless parameter $\eta=x/(2\sqrt{Dt})$.
That is, we want to find solutions of the form $\Theta(x,t)=\Theta(\eta)$.
Change variables in this way,
$$\frac{\partial\Theta}{\partial t}=\frac{\partial\eta}{\partial t}\frac{\mathrm d\Theta}{\mathrm d\eta}=-\frac{1}{2}\frac{x}{\sqrt{D}t^{3/2}}\Theta^\prime=-\frac{\eta}{2t}\Theta^\prime$$
$$D\frac{\partial^2\Theta}{\partial x^2}=D\frac{\partial}{\partial x}\left(\frac{\partial\eta}{\partial x}\frac{\mathrm d\Theta}{\mathrm d\eta}\right)=D\frac{\partial}{\partial x}\left( \frac{1}{2\sqrt{Dt}}\Theta^\prime \right)=\frac{D}{4Dt}\Theta^{\prime\prime}=\frac{1}{4t}\Theta^{\prime\prime}$$
Putting them all together gives $\Theta^{\prime\prime}=-2\eta\Theta^\prime$, which gives $\Theta^\prime\propto e^{-\eta^2}$, therefore
$$\Theta(\eta)=\Theta(0)+\frac{2C}{\sqrt{\pi}}\int_0^\eta e^{-u^2}\,\mathrm du=\Theta(0)+C\operatorname{erf}(\eta)=\Theta(0)+C\operatorname{erf}\left( \frac{x}{2\sqrt{Dt}} \right)$$
where $C$ is a constant and $\operatorname{erf}$ is the error function defined by
$$\operatorname{erf}(z)=\frac{2}{\sqrt{\pi}}\int_0^ze^{-u^2}\,\mathrm du$$
This can describes discontinuous initial conditions (e.g. step functions) that spreads over time.
\subsection{Heat Conduction in a Finite Bar}
Suppose we have a bar of length $2L$ at $[-L,L]$ and initial temperature
$$\Theta(x,0)=H(x)=\begin{cases}
    1\text{, for $x\in [0,L]$}\\
    0\text{, for $x\in [-L,0)$}
\end{cases}$$
with boundary conditions $\Theta(L,t)=1,\Theta(-L,t)=0$.
We want to use Sturm-Liouville theory, but there is a problem here:
Our boundary conditions is not homogeneous.
So we must make the condition homogeneous by a suitable superposition.
But there is an obvious choice of this, namely $\Theta_s(x,t)=(x+L)/(2L)$.
Therefore we with a transformation $\hat\Theta=\Theta-\Theta_s$ the problem becomes
$$\frac{\partial\hat\Theta}{\partial t}=D\frac{\partial^2\hat\Theta}{\partial x^2},\hat\Theta(-L,t)=\hat\Theta(L,t)=0,\hat\Theta(x,0)=H(x)-\frac{x+L}{2L}$$
Now we do seperation of variables $\hat\Theta(x,t)=X(x)T(t)$, which gives $X^{\prime\prime}=-\lambda X,\dot{T}=-D\lambda T$ where $\lambda$ is the seperation constant.
The boundary conditions imply that $\lambda>0$ and
$$X(x)=A\cos{\sqrt{\lambda}x}+B\sin(\sqrt{\lambda}x)$$
where $A,B$ are constants.
The initial condition is odd, so $A=0$ and consequently the eigenvalues are $\lambda_n=(n\pi/L)^2$ for $n=1,2,3,\ldots$, so we obtained the family of solutions
$$X_n=B_n\sin\frac{n\pi x}{L},\lambda_n=\frac{n^2\pi^2}{L^2},n=1,2,3,\ldots$$
Put $\lambda_n$ in the temporal equation gives
$$T_n(t)=C_n\exp\left( -\frac{Dn^2\pi^2}{L^2}t \right)$$
So
$$\hat\Theta(x,t)=\sum_{n=1}^\infty b_n\sin\left(\frac{n\pi x}{L}\right)\exp\left( -\frac{Dn^2\pi^2}{L^2}t \right)$$
Now we impose initial conditions at $t=0$ which gives
\begin{align*}
    b_n&=\frac{1}{L}\int_{-L}^L\hat\phi(x,0)\sin\frac{n\pi x}{L}\,\mathrm dx\\
    &=\frac{2}{L}\int_0^L\hat\phi(x,0)\sin\frac{n\pi x}{L}\,\mathrm dx\\
    &=\frac{2}{L}\int_0^L\left( H(x)-\frac{x+L}{2L} \right)\sin\frac{n\pi x}{L}\,\mathrm dx\\
    &=\frac{2}{L}\int_0^L\left( H(x)-\frac{1}{2} \right)\sin\frac{n\pi x}{L}\,\mathrm dx-\frac{2}{L}\int_0^L\frac{x}{2L}\sin\frac{n\pi x}{L}\,\mathrm dx\\
    &=\begin{cases}
        2/[(2m-1)\pi]-1/(n\pi)\text{, for $n=2m-1$ odd}\\
        1/n\pi\text{, for $n$ even}
    \end{cases}\\
    &=\frac{1}{n\pi}
\end{align*}
Therefore we get the final solution
$$\Theta(x,t)=\frac{2+L}{2L}+\hat\Theta(x,t)=\frac{x+L}{2L}+\sum_{n=1}^\infty \frac{1}{n\pi}\sin\left(\frac{n\pi x}{L}\right)\exp\left( -\frac{Dn^2\pi^2}{L^2}t \right)$$
A plot reveals that this solution is very similar to the similarity solution we got earlier especially for small $t$.
    \section{The Laplace Equation}
We already encountered Laplace equation $\Delta^2\phi=0$ several times.
It has very wide applications in mathematical physics, applied mathematics and pure mathematics.
In physics, it often describes some physical systems (e.g. heat flow) that is in stationary state as it does not depend on time.
We can also see this (possibly with a forcing term) in potential theory.
For example, Laplace used it to describe gravitational systems.
It also appears in the study of incompressible fluid flow.\\
We often want to solve Laplace's equation in a domain $D$ subject to boundary conditions.
The most common ones are the dirichlet conditions where we are given the value of $\phi$ on $\partial D$ and the Neumann conditions where we specify $\underline{\hat{n}}\cdot\nabla\phi$ on $\partial D$.
\subsection{3D Cartesian Coordinates}
In 3D Cartesian coordinates, the equation becomes
$$\frac{\partial^2\phi}{\partial x^2}+\frac{\partial^2\phi}{\partial y^2}+\frac{\partial^2\phi}{\partial z^2}=0$$
The seperation of variables $\phi(x,y,z)=X(x)Y(y)Z(z)$ gives the systems
$$X^{\prime\prime}=-\lambda_lX,Y^{\prime\prime}=-\lambda_mY,Z^{\prime\prime}=-\lambda_nZ=(\lambda_l+\lambda_m)Z$$
where $\lambda_l,\lambda_m$ are seperation constants.
Therefore the general solution arising from this way is
$$\phi(x,y,z)=\sum_{l,m,n}a_{l,m,n}X_l(x)Y_m(y)Z_n(z)$$
\begin{example}[Steady Heat Conduction]
    Consider a semi-infinite rectangular bar $[0,a]\times [0,b]\times [0,\infty]$ as the domain with boundary conditions $\phi=0$ at $x=0,a$ and $y=0,b$, $\phi=1$ at $z=0$ and $\phi\to 0$ as $z\to\infty$.
    We shall try to find the eigenmodes.
    For $X^{\prime\prime}=-\lambda_lX$ with $X(0)=X(a)=0$ we get $\lambda_l=l^2\pi^2/a^2$ and $X_l(x)=\sin(l\pi x/a)$ for $l=1,2,3,\ldots$.
    For $Y^{\prime\prime}=-\lambda_mY$ we have $\lambda_m=m^2\pi^2/b^2$ and $Y_m(y)=\sin(m\pi y/b)$ again for $m=1,2,3,\ldots$.
    For $Z$, the equation would be
    $$Z^{\prime\prime}=-\lambda_nZ=(\lambda_l+\lambda_m)Z=\pi^2\left( \frac{l^2}{a^2}+\frac{m^2}{b^2} \right)Z$$
    which has exponential solutions.
    But $Z$ is bounded at infinity, therefore necessarily
    $$Z_n=Z_{l,m}=\exp\left( -\sqrt{\frac{l^2}{a^2}+\frac{m^2}{b^2}}\pi z \right)$$
    which gives the general solution
    $$\phi(x,y,z)=\sum_{l,m}a_{l,m}\sin\frac{l\pi x}{a}\sin\frac{m\pi y}{b}\exp\left( -\sqrt{\frac{l^2}{a^2}+\frac{m^2}{b^2}}\pi z \right)$$
    Now the condition $\phi(x,y,0)=1$ gives
    $$a_{l,m}=\frac{2}{b}\int_0^b\frac{2}{a}\int_0^a\sin\frac{l\pi x}{a}\sin\frac{m\pi y}{b}\,\mathrm dx\mathrm dy=\frac{16}{\pi^2lm}$$
    for odd $l,m$ and $0$ if any of them is even.
    Therefore the heat flow solution is
    $$\phi(x,y,z)=\sum_{l,m\text{ odd}}\frac{16}{\pi^2lm}\sin\frac{l\pi x}{a}\sin\frac{m\pi y}{b}\exp\left( -\sqrt{\frac{l^2}{a^2}+\frac{m^2}{b^2}}\pi z \right)$$
    This may look complicated, and yes it is complicated.
    However, for large $l,m$ (and large $z$), the exponential term would be very much close to $0$.
    This allows us to get a very nice approximation by considering just lower order terms.
\end{example}
\subsection{2D Plane Polar Coordinates}
In plane polar, Laplace's equation translates to
$$0=\nabla^2\phi=\frac{1}{r}\frac{\partial}{\partial r}\left( r\frac{\partial\phi}{\partial r} \right)+\frac{1}{r^2}\frac{\partial^2\phi}{\partial\theta^2}$$
Again we do a seperation of variables $\phi(r,\theta)=R(r)\Theta(\theta)$ to get
$$\begin{cases}
    \Theta^{\prime\prime}+\mu\Theta=0\\
    r(rR^\prime)^\prime-\mu R=0
\end{cases}$$
where $\mu$ is the seperation constant.
Assuming periodic boundary conditions, then the polar equation yields $\mu=m^2$ and $\Theta_m(\theta)$ is a superposition of $\cos(m\theta)$ and $\sin(m\theta)$.
So the radial equation becomes $r(rR^\prime)^\prime-m^2R=0$.
For $m\neq 0$, trying $R=\alpha r^\beta$ shows that $\beta=\pm m$ works, so $R_m$ is composed of $r^m$ and $r^{-m}$.
If $m=0$, $R_0$ is a linear combination of constant and $\log r$ by just integrating.
So the general solution is just
\begin{align*}
    \phi(r,\theta)&=\frac{a_0}{2}+c_0\log r\\
    &\quad+\sum_{m=1}^\infty(a_m\cos(m\theta)+b_n\sin(m\theta))r^m\\
    &\quad+\sum_{m=1}^\infty(c_m\cos(m\theta)+d_m\sin(m\theta))r^{-m}
\end{align*}
For constants $a_m,b_m,c_m,d_m$.
\begin{example}[Soap Film on a Unit Disk]
    We want to solve Laplace's equation on the unit disk, where the boundary condition is given a distorted circular wire $\phi(1,\theta)=f(\theta)$.
    Of course we want our solution to be continuous in the inside of the disk, in particular at $0$, therefore $c_m=d_m=0$ for all $m$.
    Therefore we just got
    $$\phi(r,\theta)=\frac{a_0}{2}+\sum_{m=1}^\infty(a_m\cos(m\theta)+b_m\sin(m\theta))r^m$$
    left.
    But then $f(\theta)=\phi(1,\theta)$ gives a Fourier series (again!), so
    $$a_m=\frac{1}{\pi}\int_0^{2\pi}f(\theta)\cos(m\theta)\,\mathrm d\theta,b_m=\frac{1}{\pi}\int_0^{2\pi}f(\theta)\sin(m\theta)\,\mathrm d\theta$$
    For a nontrivial distortion, the term $r^m$ then tells us that the high harmonics are concentrated near the edge of the wire.
\end{example}
\subsection{3D Cylindrical Polar Coordinates}
Here Laplace's equation become
$$0=\nabla^2\phi=\frac{1}{r}\frac{\partial}{\partial r}\left( r\frac{\partial\phi}{\partial r} \right)+\frac{1}{r^2}\frac{\partial^2\phi}{\partial\theta^2}+\frac{\partial^2\phi}{\partial z^2}$$
Seperation of variables $\phi(r,\theta,z)=R(r)\Theta(\theta)Z(z)$ gives
$$\begin{cases}
    \Theta^{\prime\prime}=-\mu\Theta\\
    Z^{\prime\prime}=\lambda Z\\
    r(rR^\prime)^\prime+(\lambda r^2-\mu)R=0
\end{cases}$$
where $\mu,\lambda$ are the seperation constants.
For the polar equation, periodic boundary conditions give $\mu_m=m^2$ and $\Theta_m(\theta)$ is a superposition of $\sin(m\theta)$ and $\cos(m\theta)$.
The radial equation is Bessel's equation (surprise?) with eigenfunctions $R_{mn}=J_m(j_{mn}r/a)$ under boundary condition $R(a)=0$ and the requirement of it not being singular (so we can exclude the Neumann functions).
The $Z$ equation then becomes $Z^{\prime\prime}=kZ$ where $k=j_{mn}/a$ which gives $Z=e^{-kz}$ (the $e^{kz}$ solution is eliminated by the boundary condition $Z\to 0$ as $z\to\infty$).
So the general solution is
$$\phi(r,\theta,z)=\sum_{m=0}^\infty\sum_{n=1}^\infty(a_{mn}\cos(m\theta)+b_{mn}\sin(m\theta))J_m(j_{mn}r/a)\exp(-j_{mn}r/a)$$
\begin{example}
    The boundary condition $\phi=0$ at $r=a$, $\phi=T_0$ at $z=0$ and $\phi\to 0$ as $z\to\infty$ gives the solution
    $$\phi(r,\theta,z)=\sum_{n=1}^\infty\frac{2T_0}{j_{0n}J_1(j_{0n})}J_0(j_{0n}r/a)\exp(-j_{0n}z/a)$$
\end{example}
\subsection{3D Spherical Polar Coordinates}
Recall that the spherical polar coordinate transforms from Cartesian coordinates by
$$x=r\sin\theta\cos\phi,y=r\sin\theta\sin\phi,z=r\cos\theta$$
for $r\in\mathbb R_{\ge 0},\theta\in [0,\pi],\phi\in[0,2\pi]$ where we have $\mathrm dV=r^2\sin\theta\,\mathrm dr\mathrm d\theta\mathrm d\phi$.
Laplace's equation transforms into
$$0=\nabla^2\Phi=\frac{1}{r^2}\frac{\partial}{\partial r}\left( r^2\frac{\partial\Phi}{\partial r} \right)+\frac{1}{r^2\sin\theta}\frac{\partial}{\partial\theta}\left( \sin\theta\frac{\partial\Phi}{\partial\theta} \right)+\frac{1}{r^2\sin^2\theta}\frac{\partial^2\Phi}{\partial\phi^2}$$
We only consider the axis-symmetric case where $\partial\Phi/\partial\phi=0$.
Again seperate the variables $\Phi(r,\theta,\phi)=R(r)\Theta(\theta)$ gives
$$\begin{cases}
    ((\sin\theta)\Theta^\prime)^\prime+\lambda(\sin\theta)\Theta=0\\
    (r^2R^\prime)^\prime-\lambda R=0
\end{cases}$$
where $\lambda$ is the seperation constant.
The substitution $x=\cos\theta$ transforms the polar equation into
$$\frac{\mathrm d}{\mathrm dx}\left( (1-x^2)\frac{\mathrm d\Theta}{\mathrm dx} \right)+\lambda\Theta=0$$
which is exactly Legendre's equation.
So we obtain the eigenvalues $\lambda_l=l(l+1)$ with eigenfunctions $\Theta_l(\theta)=P_l(x)=P_l(\cos\theta)$ where $P_l$ is the $l^{th}$ Legendre polynomial.
Putting it into the radial equation gives $(r^2R^\prime)^\prime-l(l+1)R=0$, which gives (by educated guess) the solution $R_l$ being a superposition of $r^l$ and $r^{-l-1}$.
The general axis-symmetric solution is then
$$\Phi=\sum_{l=0}^\infty(a_lr^l+b_lr^{-l-1})P_l(\cos\theta)$$
where $a_l,b_l$ can be determined by boundary conditions.
\begin{example}
    Consider the boundary condition $\Phi(1,\theta,\phi)=f(\theta)$ for some $f$.
    Regularity implies $b_l=0$ for any $l$.
    So we have
    $$f(\theta)=\sum_{l=0}^\infty a_lP_l(\cos\theta)\implies F(x)=\sum_{l=0}^\infty a_lP_l(x)$$
    with $f(\theta)=F(\cos\theta)$.
    This gives
    $$a_l=\frac{2l+1}{2}\int_{-1}^1F(x)P_l(x)\,\mathrm dx$$
    in the special case where $f(\theta)=\sin^2\theta$ we have $\Phi=2(1-P_2(\cos\theta)r^2)/3$.
\end{example}
Consider a charge on $z$-axis at $\underline{r}_0=(0,0,1)$ and the potential at $P$ is defined by
$$\Phi(\underline{r})=\frac{1}{|\underline{r}-\underline{r}_0|}=\frac{1}{\sqrt{r^2-2r\cos\theta+1}}=\frac{1}{\sqrt{r^2-2rx+1}}$$
where $x=\cos\theta$.
It is easy to see that $\Phi$ satisfies $\nabla^2\Phi=0$ in $\mathbb R^3\setminus\{\underline{r}_0\}$.
Therefore there is some $a_l$ such that
$$\frac{1}{\sqrt{r^2-2rx+1}}=\sum_{l=0}^\infty a_lP_l(x)r^l$$
We have $P_l(1)=1$ at $x=1$, therefore plugging in $x=1$ gives $a_l=1$ for any $l$, therefore
$$\frac{1}{\sqrt{r^2-2rx+1}}=\sum_{l=0}^\infty P_l(x)r^l$$
is the generating function of the Legendre polynomials.
\begin{example}[Electric Multipoles]
    Consider the case where we put charges along $z$-axis at $z=\pm a,0$ viewed from a large distance $r>>a$ with $\Phi\to 0$ as $r\to\infty$.
    Therefore $a_n=0$ for all $n$.
    When $l=0$, we just get a point charge and thus $\Phi\propto 1/r$
    This is called the monopole field of the point charge $q$.
    When $l=1$, we get the dipole (i.e. opposite charges sitting opposite each other) $\Phi\propto(\cos\theta)/r^2$ for two opposite charges.
    When $l=2$, it is like putting a charge $2q$ at the origin and $-q$ at opposite position across the origin, in which case $\Phi\propto(3\cos^2\theta-1)/(2r^3)$ gives the quadruple field.
\end{example}
    \section{The Dirac Delta Function}
\subsection{Definition}
We want to define a generalised function $\delta(x-\xi)$ with the following properties:
$$\forall x\neq\xi,\delta(x-\xi)=0,\int_{-\infty}^\infty\delta(x-\xi)\,\mathrm dx=1$$
So $\delta(x-\xi)$ can be thought as a ``function'' with an infinite spike at $x=\xi$.
Of course, it would be ridiculous to use it really as a function.
Almost always, we use it in conjunction with an integral, so we can take it as a linear operator having the property that
$$\left( \int_{-\infty}^\infty\mathrm dx\,\delta(x-\xi) \right)f(x)=\int_{-\infty}^\infty\delta(x-\xi)f(x)\,\mathrm=f(\xi)$$
\begin{note}
    The $\delta$ function is some sort of ``generalised function'', or ``distribution'' which admits rigorous mathematical formulation.
    However, this will not be discussed here.
\end{note}
We want the $\delta$ function to represent a unit point source or an impulse in physical situations.
Loosely, we can take $\delta$ as the ``limit'' of a family of well-defined functions.
For example, we can consider
$$\delta_\epsilon(x)=\frac{1}{\epsilon\sqrt{\pi}}\exp\left( -\frac{x^2}{\epsilon^2} \right)$$
So we can interpret $\delta$ as saying
$$\int_{-\infty}^\infty\delta(x)f(x)\,\mathrm dx=\lim_{\epsilon\to 0}\int_{-\infty}^\infty\delta_\epsilon(x)f(x)\,\mathrm dx=f(0)$$
which, as one can verify, works for sufficiently nice $f$.
This is known as the Gaussian approximation.
There are some other (discrete) choices of $\delta$ too, for example,
$$\delta_n(x)=\frac{n}{2}1_{|x|\le 1/n},\delta_n(x)=\frac{\sin(nx)}{\pi x}=\frac{1}{2\pi}\int_{-n}^ne^{ikx}\,\mathrm dk,\delta_n(x)=\frac{n}{2}\operatorname{sech}^2(nx)$$
\subsection{Properties}
We interpret the integral of $\delta$ to be the Heaviside function
$$H(x)=\begin{cases}
    1\text{, for $x\ge 0$}\\
    0\text{, for $x<0$}
\end{cases}=\int_{-\infty}^x\delta(t)\,\mathrm dt$$
One can verify that the integral of $\delta_n(x)=n\operatorname{sech}^2(x)/2$, that is $(\tanh(nx)+1)/2$, tends to $H(x)$ as $n\to\infty$.\\
We are gonna do something more sacrilegeous, that would be
$$\int_{-\infty}^\infty\delta^\prime(x-\xi)f(x)\,\mathrm dx=-\int_{-\infty}^\infty\delta(x-\xi)f^\prime(x)\,\mathrm dx=-f^\prime(\xi)$$
for a sufficiently nice $f$.
\footnote{And a sufficiently nice crowd of students who does not have access to life-threatening weapons. Cure yourself by checking out some rigorous theories formulated by Dirac, Schwartz and Temple.}
\begin{example}
    For the Gaussian approximation,
    $$\delta_\epsilon^\prime(x)=-\frac{2x}{\epsilon^3\sqrt{\pi}}e^{-x^2/\epsilon^2}$$
    which one can plot and have an idea of what the heck is going on with $\delta^\prime$.
\end{example}
Also, we have the sampling property
$$\int_a^bf(x)\delta(x-\xi)\,\mathrm dx=\begin{cases}
    f(\xi)\text{, for $\xi\in(a,b)$}\\
    0\text{, otherwise}
\end{cases}$$
Also $\delta$ is even and $\delta^\prime$ is odd.
In addition we have the scaling property
$$\int_{-\infty}^\infty f(x)\delta(a(x-\xi))\,\mathrm dx=\frac{1}{|a|}f(\xi)$$
and its advanced version:
If $g$ has $n$ isolated zeros as $x_1,\ldots,x_n$ with $g^\prime(x_i)\neq 0$ for all $i$, then
$$\delta(g(x))=\sum_{i=1}^n\frac{\delta(x-x_i)}{|g^\prime(x_i)|}$$
\begin{example}
    Take $g(x)=x^2-1$, then
    \begin{align*}
        \int_{-\infty}^\infty f(x)\delta(x^2-1)\,\mathrm dx&=\int_{1-\epsilon}^{1+\epsilon}\frac{f(x)}{2|x|}\delta(x-1)\,\mathrm dx+\int_{-1-\epsilon}^{-1+\epsilon}\frac{f(x)}{|2x|}\delta(x+1)\\
        &=\frac{f(1)+f(-1)}{2}
    \end{align*}
\end{example}
There is also this isolation property: $g(x)\delta(x)=g(0)\delta(x)$ given that $g$ is continuous at $0$.
\begin{example}
    We have
    $$\int_0^\infty\delta^\prime(x^2-1)x^2\,\mathrm dx=-\frac{1}{4}$$
\end{example}
\subsection{Eigenfunction Expansions}
For $-L\le x<L$, if we want to represent
$$\delta(x)=\sum_{n\in\mathbb Z}c_ne^{in\pi x/L}$$
Then the coefficients are
$$c_n=\frac{1}{2L}\int_{-L}^L\delta(x)e^{-in\pi x/L}\,\mathrm dx=\frac{1}{L}\implies \delta(x)=\frac{1}{2L}\sum_{n\in\mathbb Z}e^{in\pi x/L}$$
We obvious want to check that it has compatible properties.
Indeed, if $f(x)=\sum_{n\in\mathbb Z}d_ne^{in\pi x/L}$ on $[-L,L)$, then
\begin{align*}
    \langle f,\delta\rangle&=\int_{-L}^Lf^*(x)\delta(x)\,\mathrm dx\\
    &=\frac{1}{2L}\sum_{n\in\mathbb Z}d_n\int_{-L}^Le^{-in\pi x/L}e^{in\pi x/L}\,\mathrm dx\\
    &=\sum_{n\in\mathbb Z}d_n\\
    &=f(0)
\end{align*}
Note that we only defined $\delta$ on $[-L,L)$, so we can use the Fourier series to extend it periodically to the whole real line and obtain what is called a Dirac comb:
$$\sum_{m\in\mathbb Z}\delta(x-2mL)=\sum_{n\in\mathbb Z}e^{in\pi x/L}$$
For general eigenfunctions $\{y_n\}$, suppose we have
$$\delta(x-\xi)=\sum_{n=1}^\infty a_ny_n(x)$$
for $x,\xi\in [a,b]$, then the coefficients are
\begin{align*}
    a_n&=\left. \int_a^bw(x)y_n(x)\delta(x-\xi)\,\mathrm dx \middle/ \int_a^bwy_n^2\,\mathrm dx\right.\\
    &=\left. w(\xi)y_n(\xi) \middle/ \int_a^bwy_n^2\,\mathrm dx\right.=w(\xi)Y_n(\xi)
\end{align*}
where $Y_n$ is the normalised eigenfunctions.
So
$$\delta(x-\xi)=w(\xi)\sum_{n=1}^\infty Y_n(\xi)Y_n(x)=w(x)\sum_{n=1}^\infty Y_n(\xi)Y_n(x)$$
since, by the isolation property, $w(x)\delta(x-\xi)/w(\xi)=\delta(x-\xi)$.
In other words,
$$\delta(x-\xi)=w(x)\int_{n=1}^\infty\frac{y_n(\xi)y_n(x)}{N_n},N_n=\int_a^bwy_n^2\,\mathrm dx$$
\begin{example}
    Consider the Fourier sine series with $y(0)=y(1)=0$ and $y_n(x)=\sin n\pi x$, then we have
    $$\delta(x-\xi)=2\sum_{n=1}^\infty\sin(n\pi\xi)\sin(x\pi x)$$
    for $\xi\in (0,1)$.
    Integrate both sides over $[0,1]$ with $\xi=1/2$ gives
    $$\frac{\pi}{4}=\sum_{n=1}^\infty\frac{(-1)^{m+1}}{2m-1}$$
\end{example}
Another interesting observation to make is that if we integrate the series in previous example twice, we obtained a Green's function we've seen before:
$$G(x,\xi)=2\sum_{n=1}^\infty\frac{\sin(n\pi x)\sin(n\pi\xi)}{(n\pi)^2}$$
    \section{Green's Function}
\subsection{Physical Motivation}
Consider the static force on a string which can be caused by gravity acting on the mass of the string.
Let the tension be $T$ and linear mass density be $\mu$.
Suppose the string is suspended between fixed ends $y(0)=y(1)=0$.
The static force is then $-\mu\delta xg$ on the $y$ direction on the piece $\delta x$.
By resolving forces, we get the equation $-y^{\prime\prime}=f(x)$ where $f(x)=-\mu g/T$.
We can solve it via direct integration.
For uniform mass density (i.e. $\mu$ is constant), we have
$$-y=-\frac{\mu g}{2T}x^2+k_1x+k_2\implies y(x)=\left( -\frac{\mu g}{T} \right)\frac{1}{2}x(1-x)$$
by the boundary conditions.
This is a parabolic curve.
\footnote{You might be expecting the catenary curve instead -- but not really, since in that problem we require the string to be non-elastic (i.e. has a fixed length) which makes it a completely different situation.}
Another way to solve this problem is to disassemble the force on the string as the sum of infinitesimal parts, and consider their superposition.
Assume the string is massless but with a point mass $\delta m$ suspended at $x=\xi_i$.
Obviously the solution is simply two line segments meeting at some points $(\xi_i,y_i)$.
Suppose the segment closer to $0$ makes an angle $\theta_1$ with the horizontal and the other segment makes an angle $\theta_2$, then resolving in $y$-direction gives
$$0=T(\sin\theta_1+\sin\theta_2)-\delta mg=T\left( \frac{-y_i}{\xi_i}+\frac{-y_i}{1-\xi_i}-\delta mg \right)$$
Solving this gives us $y_i=-\delta mg\xi_i(1-\xi_i)/T$.
So the solution is
$$y_i(x)=\frac{-\delta mg}{T}\begin{cases}
    x(1-\xi_i)\text{, for $x<\xi_i$}\\
    \xi(1-x)\text{., for $x>\xi_i$}
\end{cases}=f_iG(x,\xi),f_i=\frac{-\delta mg}{T}$$
where $f_i$ is interpreted as the source $f$ around a infinitesimal neighbourhood of $\xi_i$.
Now the superposition of the solution for $N$ point masses $\delta m$ at $x=\{\xi_i\}$ gives
$$y(x)=\sum_{i=1}^Nf_iG(x,\xi_i)$$
If we take the continuum limit
$$f_i=-\frac{\delta mg}{T}=-\frac{\mu\delta x g}{T}=f(x)\,\mathrm dx$$
we have
$$y(x)=\int_0^1f(\xi)G(x,\xi)\,\mathrm d\xi=\left( -\frac{\mu g}{T} \right)\frac{1}{2}x(1-x)$$
\subsection{Definitions of Green's Function}
We wish to solve the inhomogeneous ODE $\mathcal Ly=f(x)$ where $\mathcal L=\alpha y^{\prime\prime}+\beta y^\prime+\gamma y$ on $[a,b]$ subject to boundary conditions $y(a)=y(b)=0$.
We require $\alpha\neq 0$ over $[a,b]$ and $\alpha,\beta,\gamma$ all continuous and bounded.
\begin{definition}
    The Green's function $G$ for the operator $\mathcal L$ is the solution to $\mathcal LG(x,\xi)=\delta(x-\xi)$ subject to homogeneous boundary conditions $G(a,\xi)=G(b,\xi)=0$ for all $\xi$.
\end{definition}
So by linearily, if such $G$ does exist, then we have
$$y(x)=\int_a^bG(x,\xi)f(\xi)\,\mathrm d\xi$$
Indeed,
$$\mathcal Ly=\int_a^b\mathcal LG(x,\xi)f(\xi)\,\mathrm d\xi=\int_a^b\delta(x-\xi)f(\xi)\,\mathrm d\xi=f(x)$$
Loosely speaking, we can write $y=\mathcal L^{-1}f$ where
$$\mathcal L^{-1}=\int_a^b\mathrm d\xi\, G(x,\xi)$$
Now, by our established study of these sort of ODEs, the Green's function splits into two smooth enough parts
$$G(x,\xi)=\begin{cases}
    G_1(x,\xi)\text{, for $x\in [a,\xi)$}\\
    G_2(x,\xi)\text{, for $x\in (\xi,b)$}
\end{cases}$$
such that the following conditions hold:\\
1. $\mathcal LG_1=\mathcal LG_2=0$ at any $x\neq\xi$.\\
2. $G_1(a,\xi)=G_2(b,\xi)=0$ for any $\xi$.\\
3. $G$ is continuous, so $G_1(\xi,\xi)=G_2(\xi,\xi)$.\\
4. The jump condition of $G^\prime$, that is
$$[G^\prime(\cdot,\xi)]^{\xi_+}_{\xi_-}=G_2^\prime(\xi_+,\xi)-G_1^\prime(\xi_-,\xi)=\frac{1}{\alpha(\xi)}$$
How to construct $G$?
Note that $\mathcal L$ is a second order differential opertaor, so for $x\in[a,\xi)$, $G_1(x,\xi)=A(\xi)y_1(x)+B(\xi)y_2(x)$ for linearly independent $y_1,y_2$.
The boundary condition $G_1(a,\xi)=0$ gives $G_1(x,\xi)=C(\xi)y_-(x)$ where $y_-(a)=0$.
Similarly for $x\in(\xi,b]$ we have $G_2(x,\xi)=D(\xi)y_+(x)$ with $y_+(b)=0$.
Then continuity condition gives $C(\xi)y_-(\xi)=D(\xi)y_+(\xi)$ and $D(\xi)y_+^\prime(\xi_+)-C(\xi)y_-^\prime(\xi_-)=\alpha(\xi)^{-1}$.
In other words we have the system
$$\begin{pmatrix}
    y_-(\xi)&y_+(\xi)\\
    -y_-^\prime(\xi_-)&y_+^\prime(\xi_+)
\end{pmatrix}\begin{pmatrix}
    C(\xi)\\
    D(\xi)
\end{pmatrix}=\begin{pmatrix}
    0\\
    1/\alpha(\xi)
\end{pmatrix}$$
which has the solution (after extending everything continuously to $x=\xi$)
$$C(\xi)=\frac{y_+(\xi)}{\alpha(\xi)W(\xi)},D(\xi)=\frac{y_-(\xi)}{\alpha(\xi)W(\xi)}$$
where $W(\xi)=y_-(\xi)y_+^\prime(\xi)-y_+(\xi)y_-^\prime(\xi)$ is the Wronskian which is nonzero if $y_+,y_-$ are linearly independent.
So the final Green's function is
$$G(x,\xi)=\begin{cases}
    y_-(x)y_+(\xi)/(\alpha(\xi)W(\xi))\text{, for $x\in[a,\xi]$}\\
    y_+(x)y_-(\xi)/(\alpha(\xi)W(\xi))\text{, for $x\in[\xi,b]$}
\end{cases}$$
Therefore the solution to the original boundary value problem $\mathcal Ly=f$ for $y=0$ at $a,b$ is
\begin{align*}
    y(x)&=\int_a^bG(x,\xi)f(\xi)\,\mathrm d\xi\\
    &=\int_a^xG_2(x,\xi)f(\xi)\,\mathrm d\xi+\int_x^bG_1(x,\xi)f(\xi)\,\mathrm d\xi\\
    &=y_+(x)\int_a^x\frac{y_-(\xi)f(\xi)}{\alpha(\xi)W(\xi)}\,\mathrm d\xi+y_-(x)\int_x^b\frac{y_+(\xi)f(\xi)}{\alpha(\xi)W(\xi)}\,\mathrm d\xi
\end{align*}
\begin{note}
    1. If $\mathcal L$ is in Sturm-Liouville form, then $\beta=\alpha^\prime$, then $\alpha(\xi)W(\xi)$ is a constant and hence $G$ has to be symmetric.\\
    2. Often we take $\alpha=1$.\\
    3. The indefinite integrals are the particular integrals in the particular integral in the Sturm-Liouville solution.
\end{note}
\begin{example}
    For $y^{\prime\prime}-y=f(x),y(0)=y(1)=0$, the homogeneous solutions are $y_1=e^x$ and $y_2=e^{-x}$.
    Imposing the homogeneous boundary conditions reveals that $y_-(x)=\sinh(x)$ and $y_+(x)=\sinh(1-x)$.
    The countinuity condition of $G$ gives $C=D\sinh(1-\xi)/(\sinh\xi)$.
    The jump condition of $G^\prime$ then gives
    $$D=-\frac{\sinh\xi}{\sinh 1},C=-\frac{\sinh(1-\xi)}{\sinh 1}$$
    So
    $$y(x)=-\frac{\sinh(1-x)}{\sinh 1}\int_0^x\sinh(\xi)f(\xi)\,\mathrm d\xi-\frac{\sinh x}{\sinh 1}\int_x^1\sinh(1-\xi)f(\xi)\,\mathrm d\xi$$
\end{example}
For inhomogenous boundary conditions, we simply need to find a solution to $\mathcal Ly_p=0$ satisfying them and solve for $\mathcal Ly_g=f$ under homogeneous boundary conditions by Green's functions.
Adding them up gives the particular solution $y=y_p+y_g$.
\begin{example}
    For $y^{\prime\prime}-y=f(x)$ with $y(0)=0,y(1)=1$, we have the solution $y_p(x)=\sinh x/\sinh 1$ to $y^{\prime\prime}-y=0$ under the same boundary conditions, so the particular solution would be
    \begin{align*}
        y(x)&=y_p(x)+y_g(x)\\
        &=\frac{\sinh x}{\sinh 1}-\frac{\sinh(1-x)}{\sinh 1}\int_0^x\sinh(\xi)f(\xi)\,\mathrm d\xi-\frac{\sinh x}{\sinh 1}\int_x^1\sinh(1-\xi)f(\xi)\,\mathrm d\xi
    \end{align*}
\end{example}
How about high-order ODEs?
Suppose we have $\mathcal Ly=f(x)$ with the highest order term $\alpha(x)y^{(n)}(x)$ in $\mathcal Ly$ with $\alpha\neq 0$ everywhere, then $\mathcal LG(x,\xi)=\delta(x-\xi)$ has the properties:\\
1. $G_1,G_2$ are solutions to $\mathcal LG=0$.\\
2. $G_1,G_2$ satisfy the homogeneous boundary conditions.\\
3. Continuity condition of $G_1^{(i)}(\xi,\xi)=G_2^{(i)}(\xi,\xi)$ for all $i=1,\ldots,n-2$.\\
4. Jump condition of $[G^{(n-1)}(\cdot,\xi)]_{\xi_-}^{\xi_+}=G_2^{(n-1)}(\xi_+,\xi)-G_1^{(n-1)}(\xi_-,\xi)=1/\alpha(\xi)$.\\
We want to take a look at the eigenfunction expansion of $G$.
Suppose $\mathcal L$ is in Sturm-Liouville form with eigenfunctions $y_n(x)$ with eigenvalues $\lambda_n$.
We seek an expansion
$$G(x,xi)\sum_{n=1}^\infty A_n(\xi)y_n(x)$$
satisfying $\mathcal LG=\delta(x-\xi)$.
Now suppose such an expansion exists, then
\begin{align*}
    \sum_{n=1}^\infty A_n(\xi)\lambda_nw(x)y_n(x)&=\sum_{n=1}^\infty A_n(\xi)\mathcal Ly_n(x)=\mathcal LG\\
    &=\delta(x-\xi)=w(x)\sum_{n=1}^\infty y_n(\xi)\frac{y_n(x)}{N_n}
\end{align*}
where $N_n=\langle y_n,y_n\rangle_w$ is the normalisation constant.
Consequently $A_n(\xi)=y_n(\xi)/(\lambda_nN_n)$, therefore
$$G(x,\xi)=\sum_{n=1}^\infty\frac{y_n(\xi)y_n(x)}{\lambda_nN_n}=\sum_{n=1}^\infty\frac{Y_n(\xi)Y_n(x)}{\lambda_n}$$
\subsection{Construction of Green's Function from Initial Values}
We want to solve $\mathcal Ly(t)=f(t)$ for $t\ge a$ with $y(a)=y^\prime(a)=0$.
The Green's function then should satisfy $\mathcal LG=\delta(t-\tau)$ with $G(a)=G^\prime(a)=0$.
For $t<\tau$, $G=G_1$ satisfies $\mathcal LG_1=0$.
So $G_1=Ay_1+By_2$ for $A,B$ constants and $y_1,y_2$ linearly independent solutions to $\mathcal Ly=0$.
The initial conditions then give
$$\begin{pmatrix}
    y_1(a)&y_2(a)\\
    y_1^\prime(a)&y_2^\prime(a)
\end{pmatrix}\begin{pmatrix}
    A\\
    B
\end{pmatrix}=\begin{pmatrix}
    0\\
    0
\end{pmatrix}$$
But $y_1,y_2$ are independent, so the Wronskian is nowhere zero, hence nonzero at $a$.
Therefore necessarily $A=B=0$, so $G_1(t,\tau)=0$ for $a\le t<\tau$.\\
For $t>\tau$, we have $G=G_2$ for $\mathcal LG_2=0$ and $G_2(\tau,\tau)=0$ by continuity.
So we can choose solution $y_+$ to $\mathcal Ly=0$ such that $G_2(t,\tau)=D(\tau)y_+(t)$.
But we must have (extending everything continuously to $\tau$)
$$\frac{1}{\alpha(\tau)}=G_2^\prime(\tau,\tau)-G_1^\prime(\tau,\tau)$$
which gives $D(\tau)=1/(\alpha(\tau)y_+^\prime(\tau))$, hence
$$G(t,\tau)=\begin{cases}
    0\text{, for $t\le\tau$}\\
    y_+(t)/(\alpha(\tau)y_+^\prime(\tau))\text{, for $t\ge\tau$}
\end{cases}$$
So we get the solution
$$y(t)=\int_a^tG_2(t,\tau)f(\tau)\,\mathrm d\tau=\int_a^t\frac{y_+(t)f(\tau)}{\alpha(\tau)y_+^\prime(\tau)}\,\mathrm d\tau$$
which looks simpler as we built in the causality with the initial conditions.
\begin{example}
    For $y^{\prime\prime}-y=f(t)$ with $y(0)=y^\prime(0)=0$.
    We then obtain $G_2(t,\tau)=\sinh(t-\tau)$.
    Therefore
    $$y(t)=\int_0^tf(\tau)\sinh(t-\tau)\,\mathrm d\tau$$
\end{example}
    \section{Fourier Transforms}
\subsection{Introduction}
\begin{definition}
    The Fourier transform (FT) of a function $f(x)$ is
    $$\tilde{f}(k)=(\mathcal F(f))(k)=\int_{-\infty}^\infty f(x)e^{-ikx}\,\mathrm dx$$
    The inverse Fourier transform is
    $$f(x)=(\mathcal F^{-1}(\tilde{f}))(x)=\frac{1}{2\pi}\int_{-\infty}^\infty\tilde{f}(k)e^{ikx}\,\mathrm dk$$
\end{definition}
Beware that there are several conventioned for FT.
$\tilde{f}$ is said to be in the frequency domain.
\begin{theorem}[Fourier Inversion Theorem]
    $\mathcal F^{-1}\circ\mathcal F(f)=f$ given that all integrals involved are well-behaved (a sufficient condition is $f,\tilde{f}$ both being absolutely integrable).
\end{theorem}
\begin{example}
    Take $f(x)=(1/(\sigma\sqrt\pi))e^{-x^2/\sigma^2}$, then we can either find $\tilde{f}$ by completing square or observe that
    $$\tilde{f}(k)=\frac{1}{\sigma\sqrt{\pi}}\int_{-\infty}^\infty e^{-x^2/\sigma^2}e^{-ikx}\,\mathrm dx=\frac{1}{\sigma\sqrt{\pi}}\int_{-\infty}^\infty e^{-x^2/\sigma^2}\cos(kx)\,\mathrm dx$$
    Differentiate under the integral sign,
    \begin{align*}
        \frac{\mathrm d\tilde{f}}{\mathrm dk}&=-\frac{1}{\sigma\sqrt\pi}\int_{-\infty}^\infty xe^{-x^2/\sigma^2}\sin(kx)\,\mathrm dx\\
        &=-\frac{1}{\sigma\sqrt\pi}\int_{-\infty}^\infty\left( \frac{k\sigma^2}{2} \right)e^{-x^2/\sigma^2}\cos(kx)\,\mathrm dx\\
        &=-\frac{k\sigma^2}{2}\tilde{f}(k)
    \end{align*}
    Integrating this differential equation on $\tilde{f}$ gives $\tilde{f}(k)=Ce^{-k^2\sigma^2/4}$ for some constant $C$.
    Setting $k=0$ gives $C=1$, therefore $\tilde{f}(k)=e^{-k^2\sigma^2/4}$.
    One can show that $\mathcal F^{-1}\tilde{f}=f$.
\end{example}
\begin{example}
    The Fourier transform of $f(x)=e^{-a|x|}$ is $\tilde{f}(k)=2a/(a^2+k^2)$ either by direct integration or superposition.
\end{example}
\subsection{Relation with Fourier Series}
Recall that the Fourier series has the form
$$f(x)=\sum_{n=-\infty}^\infty c_ne^{ik_nx},k_n=n\Delta k,\Delta k=\frac{\pi}{L}$$
Now we know that
$$c_n=\frac{1}{2L}\int_{-L}^Lf(x)e^{-ik_nx}\,\mathrm dx=\frac{\Delta k}{2\pi}\int_{-L}^Lf(x)e^{-ik_nx}\,\mathrm dx$$
Imagine taking $L\to\infty,\Delta k\to 0$ then we get the expression of Fourier transform multiplied by an infinitesimal term.
To further justify the analogy, observe that we get
$$f(x)=\sum_{n=-\infty}^\infty\frac{\Delta k}{2\pi}e^{ik_nx}\int_{-L}^Lf(x')e^{-ik_nx'}\,\mathrm dx'$$
This looks like a Riemann sum, so let us take the limit $L\to\infty,\Delta k\to 0$ which gives
$$f(x)=\frac{1}{2\pi}\int_{-\infty}^\infty \left( \int_{-\infty}^\infty f(x')e^{-ikx'}\,\mathrm dx' \right)e^{ikx}\,\mathrm dk=\mathcal F^{-1}\circ\mathcal F(f)(x)$$
Note that when $f$ is discontinuous at $x$, then $\mathcal F^{-1}\circ\mathcal F(f)(x)=(f(x_-)+f(x_+))/2$ which is a similar behaviour to that of a Fourier series.
\subsection{Properties of Fourier Transform}
The operators $\mathcal F,\mathcal F^{-1}$ is both linear.
Also, the translation $h(x)=f(x-\lambda)$ gives $\tilde{h}(k)=e^{-i\lambda k}\tilde{f}(k)$.
Correspondingly, the frequency shift $h(x)=e^{i\lambda x}f(x)$ gives $\tilde{h}(k)=\tilde{f}(k-\lambda)$.
The scaling $h(x)=f(\lambda x)$ for $\lambda\neq 0$ gives $\tilde{h}(k)=\tilde{f}(k/\lambda)/|\lambda|$.\\
These are all trivial facts, here is a slightly more interesting one:
$h(x)=xf(x)$ gives $\tilde{h}(k)=i\tilde{f}^\prime(k)$.
Indeed,
$$\tilde{h}(k)=\int_{-\infty}^\infty xf(x)e^{-ikx}\,\mathrm dx=-\frac{1}{i}\frac{\mathrm d}{\mathrm dk}\int_{-\infty}^\infty f(x)e^{-ikx}\,\mathrm dx=i\tilde{f}^\prime(k)$$
What's more important is that if $f$ vanishes at $\pm\infty$ and $h(x)=f^\prime(x)$ gives $\tilde{h}(k)=ik\tilde{f}(k)$ via integration by parts
$$\tilde{h}(k)=\int_{-\infty}^\infty f^\prime(x)e^{-ikx}\,\mathrm dx=[f(x)e^{-ikx}]_{-\infty}^\infty-\int_{-\infty}^\infty (-ik)f(x)e^{-ikx}\,\mathrm dx=ik\tilde{f}(k)$$
So we can employ Fourier transform to turn a (nice enough) differential equation into an algebraic one.\\
We also have a sense of duality between $x$ and $k$ here.
Observe that we have
$$f(-x)=\frac{1}{2\pi}\int_{-\infty}^\infty\tilde{f}(k)e^{-ikx}\,\mathrm dk,f(-k)=\frac{1}{2\pi}\int_{-\infty}^\infty\tilde{f}(x)e^{-ikx}\,\mathrm dx$$
Therefore $g(x)=\tilde{f}(x)$ iff $\tilde{g}(k)=2\pi f(-k)$..
So $f(-x)=(2\pi)^{-1}\mathcal F^2(f)(x)$.
Iterating this gives $\mathcal F^4(f)(x)=4\pi^2f(x)$.
\begin{example}
    Consider the ``top hat'' function
    $$f(x)=\begin{cases}
        1\text{, if $|x|\le a$}\\
        0\text{, if $|x|>a$}
    \end{cases}$$
    We get
    $$\tilde{f}(k)=\int_{-\infty}^\infty f(x)e^{-ikx}\,\mathrm dx=\int_{-a}^a\cos(kx)\,\mathrm dx=\frac{2\sin(ka)}{k}$$
    Fourier inversion theorem then gives
    $$\frac{1}{\pi}\int_{-\infty}^\infty e^{ikx}\frac{\sin ka}{k}\,\mathrm dk=\begin{cases}
        1\text{, if $|x|<a$}\\
        0\text{, if $|x|>a$}
    \end{cases}$$
    Set $x=0$ and $k\to x$ gives
    $$\int_0^\infty\frac{\sin(ax)}{x}\,\mathrm dx=\frac{\pi}{2}\operatorname{sgn}(a)=\begin{cases}
        \pi/2\text{, if $a>0$}\\
        0\text{, if $a=0$}\\
        -\pi/2\text{, if $a<0$}
    \end{cases}$$
    which is pretty awesome.
\end{example}
\subsection{Convolution and Parseval's Theorem}
Recall that
\begin{definition}
    The convolution of $f$ and $g$ is
    $$(f\ast g)(x)=\int_{-\infty}^\infty f(y)g(x-y)\,\mathrm dy$$
\end{definition}
We want to multiply together functions in frequency domain, that is $\tilde{h}=\tilde{f}\tilde{g}$, and find its inverse Fourier transform.
\begin{align*}
    h(x)&=\frac{1}{2\pi}\int_{-\infty}^\infty\tilde{f}(k)\tilde{g}(k)e^{ikx}\,\mathrm dk\\
    &=\frac{1}{2\pi}\int_{-\infty}^\infty\left( \int_{-\infty}^\infty f(y)e^{-iky}\,\mathrm dy \right)\tilde{g}(k)e^{ikx}\,\mathrm dk\\
    &=\int_{-\infty}^\infty f(y)\left( \frac{1}{2\pi}\int_{-\infty}^\infty\tilde{g}(k)e^{ik(x-y)}\,\mathrm dk \right)\,\mathrm dy\\
    &=\int_{-\infty}^\infty f(y)g(x-y)\,\mathrm dy\\
    &=(f\ast g)(x)
\end{align*}
By duality, we also have
$$h(x)=f(x)g(x)\implies \tilde{h}(k)=\frac{1}{2\pi}\int_{-\infty}^\infty \tilde{f}(p)\tilde{g}(k-p)\,\mathrm dp=\frac{1}{2\pi}(\tilde{f}\ast\tilde{g})(k)$$
Now consider $h(x)=g^*(-x)$, then
\begin{align*}
    \tilde{h}(k)&=\int_{-\infty}^\infty g^\ast(-x)e^{-ikx}=\left( \int_{-\infty}^\infty g(-x)e^{ikx}\,\mathrm dx \right)^*\\
    &=\left( \int_{-\infty}^\infty g(y)e^{-iky}\,\mathrm dy \right)^*=\tilde{g}^*(k)
\end{align*}
By our study of convolution we have
$$\int_{-\infty}^\infty f(y)g^*(y-x)\,\mathrm dy=\frac{1}{2\pi}\int_{-\infty}^\infty\tilde{f}(k)\tilde{g}^*(k)e^{ikx}\,\mathrm dx$$
Set $x\to 0$ gives
$$\int_{-\infty}^\infty f(y)g^*(y)\,\mathrm dy=\frac{1}{2\pi}\int_{-\infty}^\infty\tilde{f}(k)\tilde{g}^*(k)\,\mathrm dk\implies\langle g,f\rangle=\frac{1}{2\pi}\langle \tilde{g},\tilde{f}\rangle$$
Setting $g=f$ then gives
\begin{theorem}[Parseval's Theorem]
    $$\int_{-\infty}^\infty|f(x)|^2\,\mathrm dx=\frac{1}{2\pi}\int_{-\infty}^\infty |\tilde{f}(k)|^2\,\mathrm dk$$
\end{theorem}
\subsection{Fourier Transform of Generalised Functions}
We want to apply $\mathcal F$ to generalised functions.
If one wants to be precise, one can view them as the limit of the sequence of the Fourier transforms of well-defined well-behaved functions that approaches the generalised function in question.
The details can be justified by Parseval's theorem, but the treatment is beyond the scope of this course.\\
Of course the main culprit of generalised functions is $\delta$.
Surprisingly, $\delta$ is found naturally in Fourier transform.
For a well-behaved $f$,
\begin{align*}
    f(x)&=\mathcal F^{-1}(\mathcal F(f))(x)=\frac{1}{2\pi}\int_{-\infty}^\infty\int_{-\infty}^\infty f(u)e^{-iku}e^{ikx}\,\mathrm du\mathrm dk\\
    &=\int_{-\infty}^\infty f(u)\left( \frac{1}{2\pi}\int_{-\infty}^\infty e^{ik(x-u)}\,\mathrm dk \right)\,\mathrm du
\end{align*}
So we might identify
\footnote{The number of things that would go wrong with this is overwhelming, but what the hell.}
$$\delta(x-u)=\frac{1}{2\pi}\int_{-\infty}^\infty e^{ik(x-u)}\,\mathrm dk$$
A direct calculation, on the other hand yields
$$\tilde\delta(k)=\int_{-\infty}^\infty \delta(x)e^{ikx}\,\mathrm dx=1$$
So dually, the Fourier transform of the constant $f(x)=1$ is $\tilde{f}(k)=2\pi\delta(x)$.
Also $f(x)=\delta(x-a)$ has $\tilde{f}(k)=e^{ika}$.
If we have exponentials then we naturally have trigonometrics, so here goes:
$f(x)=\cos(\omega x)$ has $\tilde{f}(k)=\pi(\delta(k+\omega)+\delta(k-\omega))$ and $f(x)=\sin(\omega x)$ has $\tilde{f}(k)=i\pi (\delta(k+\omega)-\delta(k-\omega))$.\\
Let's move on to something slightly better.
Consider the Heaviside function
$$H(x)=\begin{cases}
    1\text{, if $x>0$}\\
    0\text{, if $x<0$}\\
    1/2\text{, if $x=0$}
\end{cases}$$
Then $H(x)+H(-x)=1$ for any $x$.
Therefore $\tilde{H}(k)+\tilde{H}(-k)=2\pi\delta(k)$.
But $H^\prime(x)=\delta(x)$, hence $ik\tilde{H}(k)=1$.
So for these results to be consistent (note that $k\delta(k)=1$), we had to have $\tilde{H}(k)=\pi\delta(k)+(ik)^{-1}$.
The formula can look nicer if we shift a bit and consider instead $f(x)=\operatorname{sgn}(x)/2$ which has $\tilde{f}(k)=(ik)^{-1}$.
\subsection{Applications of Fourier Transforms}
The first very important application of Fourier transforms is in boundary value problems of ODEs.
Consider the problem $y^{\prime\prime}-y=f(x)$ with homogeneous boundary conditions $y\to 0$ as $x\to\pm\infty$.
Apply Fourier transform on both sides gives
$$(-k^2-1)\tilde{y}=\tilde{f}\implies \tilde{y}(k)=\tilde{f}(k)\tilde{g}(k),\tilde{g}(k)=-\frac{1}{1+k^2}$$
Therefore
\begin{align*}
    y(x)&=\int_{-\infty}^\infty f(u)g(x-u)\,\mathrm du\\
    &=-\frac{1}{2}\int_{-\infty}^\infty f(u)e^{-|x-u|}\,\mathrm du\\
    &=-\frac{1}{2}\int_{-\infty}^xf(u)e^{u-x}\,\mathrm du-\frac{1}{2}\int_x^\infty f(u)e^{x-u}\,\mathrm du
\end{align*}
which is in the form of the solution we would have obtained if we use Green's function instead.
Of course, we can solve this by inverse Fourier transform as well, which is quite easy once we introduce Fast Fourier Transform (FFT).
We will go through it later.\\
Another motivation of Fourier transform in signal processing.
Suppose we are given some sort of input signal $J(t)$ which is acted on by some linear differential operator $\mathcal L$ to yield output $O(t)$.
The Fourier transform
$$\tilde{J}(\omega)=\int_{-\infty}^\infty J(t)e^{i\omega t}\,\mathrm dt$$
is called the resolution.
In the frequency domain, the action of $\mathcal L$ in $J(t)$ is just multiplying $\tilde{J}(\omega)$ by a transfer function $\tilde{R}\omega$ to yield the output
$$O(t)=\frac{1}{2\pi}\int_{-\infty}^\infty\tilde{R}(\omega)\tilde{J}(\omega)e^{i\omega t}\,\mathrm d\omega$$
The inverse Fourier transform $R$ of $\tilde{R}$ is called the response function.
So $O$ is simply $J\ast R$.
For example, if there is no input $J(t)=0$ for $t<0$.
By causality, we expect $R(t)=0$ for $t<0$.
Therefore
$$O(t)=\int_0^tJ(u)R(t-u)\,\mathrm du$$
which is in the same form as the Green's function in an initial value problem.\\
We want to explore the general transfer functions for a class of ODEs.
Suppose the input/output relation is
$$\mathcal LO=\left( \sum_{i=0}^na_i\frac{\mathrm d^i}{\mathrm dx^i} \right)O=J$$
where $a_i$ are constants.
Taking Fourier transform gives
$$(a_0+a_1(i\omega)+\ldots,a_n(i\omega)^n)\tilde{O}(\omega)=\tilde{J}(\omega)$$
Therefore $\tilde{R}(\omega)=(a_0+a_1(i\omega)+\ldots,a_n(i\omega)^n)^{-1}$.
Factorise the polynomial to turn it into the form $\tilde{R}(\omega)=((i\omega-c_1)^{k_1}\cdots (i\omega-c_r)^{k_r})^{-1}$ where $i\neq j\implies c_i\neq c_j$.
Partial fraction gives
$$\tilde{R}(\omega)=\frac{1}{(i\omega-c_1)^{k_1}\cdots (i\omega-c_r)^{k_r}}=\sum_{j=1}^r\sum_{m=1}^{k_j}\frac{\Gamma_{jm}}{(i\omega-c_j)^m}$$
for constants $\Gamma_{jm}$.
But we know that
$$\mathcal F^{-1}\left( \frac{1}{(i\omega-a)^m} \right)=\begin{cases}
    t^{m-1}e^{at}/(m-1)!\text{, if $t>0$}\\
    0\text{, if $t<0$}
\end{cases}$$
Getting back to time domain, we obtain the response function
$$R(t)=\sum_{j=1}^r\sum_{m=1}^{k_j}\Gamma_{jm}\frac{t^{m-1}}{(m-1)!}e^{c_jt},t>0$$
\begin{example}
    Consider the damp oscillator $\mathcal Ly=y^{\prime\prime}+2py^\prime+(p^2+q^2)y=f(t)$ with damping $p>0$ with homogeneous initial conditions $y(0)=y^\prime(0)=0$.
    The Fourier transform of this is $(i\omega)^2\tilde{y}+2ip\omega\tilde{y}+(p^2+q^2)\tilde{y}=\tilde{f}$, so
    $$\tilde{y}=\frac{\tilde{f}}{-\omega^2+2ip\omega+p^2+q^2}=\tilde{R}\tilde{f},\tilde{R}=\frac{1}{-\omega^2+2ip\omega+p^2+q^2}$$
    So
    $$y(t)=\int_0^tR(t-\tau)f(\tau)\,\mathrm d\tau,R(t-\tau)=\frac{1}{2\pi}\int_{-\infty}^\infty\frac{\exp(i\omega(t-\tau))\,\mathrm d\omega}{p^2+q^2+2ip\omega-\omega^2}$$
    which, as one can verify, is analogous to the Green's function methods since $\mathcal LR(t-\tau)=\delta(t-\tau)$.
\end{example}
\subsection{Discrete Fourier Transform}
Suppose we sample a signal $h(t)$ at equal times $t_n=n\Delta$ with time-sampling $\Delta$ and values $h_n=h(n\Delta)$ with $n\in\mathbb Z$.
That is, the sampling frequency is $f_s=1/\Delta$ (and angular frequency $\omega_s=2\pi f_s=2\pi/\Delta$).
The Nyquist frequency $f_c=1/(2\Delta)$ is the highest frequency actually sampled at $\Delta$.
Suppose we have a (sinusoidal) signal with a given frequency $f$
$$g_f(t)=A\cos(2\pi ft+\phi)=\frac{A}{2}(e^{i\phi}e^{2\pi ift}+e^{-i\phi}e^{-2\pi ift})$$
What happens if we sample at $f=f_c$?
We have
$$g_{f_c}(t_n)=A\cos\left( 2\pi\frac{1}{2\Delta}n\Delta+\phi \right)=(A\cos\phi)\cos(\pi n)=A'\cos(2\pi f_ct_n)$$
So information about phase and amplitude are lost.
Even worse if we sample above $f>f_c$:
If we sample at $f=f_c+\delta f$ for some small $\delta f>0$, then
$$g_f(t_n)=A\cos(2\pi(f_c+\delta f)t_n+\phi)=A\cos(2\pi(f_c-\delta f)t_n-\phi)$$
So the effect is just aliasing a ``ghost signal'' to frequency $f_c-\delta f$ (or $-(f_c-\delta f)$), which is a contamination of the information.
\begin{definition}
    A signal $g(t)$ is bandwidth limited if it contains no frequency above some $\omega_{\max{}}=2\pi f_{\max{}}$, that is $\tilde{g}(\omega)=0$ for any $|\omega|>\omega_{\max{}}$.
\end{definition}
So for a bandwidth limited signal $g(t)$ would have
$$g(t)=\frac{1}{2\pi}\int_{-\infty}^\infty\tilde{g}(\omega)e^{i\omega t}\,\mathrm d\omega=\frac{1}{2\pi}\int_{-\omega_{\max{}}}^{\omega_{\max{}}}\tilde{g}(\omega)e^{i\omega t}\,\mathrm d\omega$$
\begin{theorem}[Sampling Theorem]
    Let $g$ be a bandwidth limited signal and $\Delta=1/(2f_{\max{}})$, then define
    $$g_n=g(t_n)=\frac{1}{2\pi}\int_{-\omega_{\max{}}}^{\omega_{\max{}}}\tilde{g}(\omega)e^{i\pi n\omega/\omega_{\max{}}}\,\mathrm d\omega$$
    which induces a Fourier series
    $$\tilde{g}_{\mathrm{per}}(\omega)=\frac{\pi}{\omega_{\max{}}}\sum_{n=-\infty}^\infty g_ne^{-i\pi n\omega/\omega_{\max{}}}$$
    Then, we have
    $$\tilde{g}(\omega)=\tilde{g}_{\mathrm{per}}(\omega)\tilde{h}(\omega),\tilde{h}(\omega)=\begin{cases}
        1\text{, if $|\omega|\le\omega_{\max{}}$}\\
        0\text{, otherwise}
    \end{cases}$$
    Inverting which gives
    \begin{align*}
        g(t)&=\frac{1}{2\omega_{\max{}}}\sum_{n=-\infty}^\infty g_n\int_{-\omega_{\max{}}}^{\omega_{\max{}}}\exp\left( i\omega\left( t-\frac{n\pi}{\omega_{\max{}}} \right) \right)\,\mathrm d\omega\\
        &=\sum_{n=-\infty}^\infty g_n\frac{\sin(\omega_{\max{}} t-\pi n)}{\omega_{\max{}} t-\pi n}
    \end{align*}
    So $g(t)$ can be exactly represented after sampling at discrete times $t_n$.
\end{theorem}
\begin{proof}
    Self-explanatory.
\end{proof}
Suppose we have a finite number $N$ of samples $h_m=h(t_m)$ where $t_m=m\Delta$ for $m=0,\ldots,N-1$.
We want to approximate the Fourier Transform for $N$ frequencies within $f_c=1/(2\Delta)$ with equally spaced frequencies with space $\Delta_f=1/(N\Delta)$ in the range $[-f_c,f_c]$.
So basically we are just looking for $f_n=n\Delta_f=n/(N\Delta)$ where $n=-N/2,\ldots,0,\ldots,N/2$.
Note that $f_c$ and $-f_c$ are aliased together, so the $-N/2$ and $N/2$ are basically the same.
Also $(m+N/2)\Delta_f=f_c+\delta f$ is aliased to $(-m+N/2)\Delta_f=-(f_c-\delta f)$, so we choose instead $f_n=n/(N\Delta)$ with $n=0,\ldots,N-1$.
\begin{definition}
    Observe that
    \begin{align*}
        \tilde{h}(f_n)&=\int_{-\infty}^\infty h(t)e^{-2\pi if_nt}\,\mathrm dt\approx\Delta\sum_{n=0}^{N-1}h_me^{-2\pi i f_nt_m}\\
        &=\Delta\sum_{m=0}^{N-1}h_me^{-2\pi imn/N}=\Delta\tilde{h}_d(f_n)
    \end{align*}
    Here $\tilde{h}_d(f_n)=\tilde{h}_n$ is the discrete Fourier transform (DFT).
\end{definition}
So the matrix $[\operatorname{DFT}]_{mn}=e^{-2\pi imn/N}$ defines the discrete Fourier transform for $h=\{h_m\}$ as we have $\tilde{h}_d=[\operatorname{DFT}]h$.
The inverse of this matrix is its adjoint, i.e. $[\operatorname{DFT}]^{-1}=N^{-1}[\operatorname{DFT}]^\dagger$ and it is built from the $N^{th}$ roots of unity.
\begin{example}
    If $N=4$ and $\omega=-i$, then
    $$[\operatorname{DFT}]=\begin{pmatrix}
        1&1&1&1\\
        1&-i&-1&i\\
        1&-1&1&-1\\
        1&i&-1&-i
    \end{pmatrix}$$
\end{example}
The inverse DFT is
\begin{align*}
    h_m&=h(t_m)\approx\frac{1}{2\pi}\int_{-\infty}^\infty\tilde{h}(\omega)e^{i\omega t_m}\,\mathrm d\omega=\int_{-\infty}^\infty\tilde{h}(f)e^{2\pi ift_m}\,\mathrm df\\
    &\approx\frac{1}{N\Delta}\sum_{n=0}^{N-1}\Delta\tilde{h}_d(f_n)e^{2\pi imn/N}\\
    &=\frac{1}{N}\sum_{n=0}^{N-1}\tilde{h}_ne^{2\pi imn/N}
\end{align*}
In this frame, we can establish analogues of Parseval's theorem
$$\sum_{m=0}^{N-1}|h_m|^2=\frac{1}{N}\sum_{n=0}^{N-1}|\tilde{h}_n|^2$$
and convolution theorem
$$c_k=\sum_{m=0}^{N-1}g_mh_{k-m}\iff \tilde{c}_k=\tilde{g}_k\tilde{h}_k$$
This looks complicated, but there is actually a very efficient algorithm to do it, known as fast Fourier transform (FFT), by exploiting the symmetries of the expression.
    \section{Characteristics}
\subsection{Well-Posed Cauchy Problems}
Solving PDEs depends on the equation and the boundary/initial data.
A Cauchy problem is a PDF together with auxiliary data specified on a surface in 3D or a curve in 2D (known as the Cauchy data).
\begin{definition}
    A Cauchy problem is well-posed if:\\
    1. A solution exists.\\
    2. The solution is unique.\\
    3. The solution depends continuously on the auxiliary data.
\end{definition}
\subsection{Method of Characteristics}
Suppose we have a curve $C$ parameterised by $(x(s),y(s))$ in space and a tangent $v=(x^\prime(s),y^\prime(s))$ to the curve at some point $P$.
For a function $\phi(x,y)$, we can define a directional derivative
$$\left.\frac{\mathrm d\phi}{\mathrm ds}\right|_C=x^\prime(s)\frac{\partial\phi}{\partial x}+y^\prime(s)\frac{\partial\phi}{\partial y}=v\cdot \nabla\phi|_C$$
If $v\cdot\nabla\phi=0$, then $\mathrm d\phi/\mathrm ds=0$ and $\phi$ is constant along $C$.
Now suppose we have a vector field $u=(\alpha(x,y),\beta(x,y))$ with its family of integral curves (i.e. curves which tangent to the field at any point) non-intersecting and filling $\mathbb R^2$.
Now find a curve $B$ by $(x(t),y(t))$ tranverse to $u$ such that its tangent vector $w=(x^\prime(t),y^\prime(t))$ is never parallel to $u$.
Label each integral curve $C$ of $u$ using $t$ at the intersection point with $B$ and then use $s$ to parameterise along the curve (i.e. take $s=0$ at $B$).
Our integral curves then satisfy $x^\prime(s)=\alpha(x,y),y^\prime(s)=\beta(x,y)$, solving which gives a family of characteristic curves along whcih $t$ remains constant.
In some sense, we are creating a new coordinate $(s,t)$ at which the PDE is in a nice form.
\subsection{Characteristics of a First-Order PDE}
Consider the first-order PDE
$$\alpha(x,y)\frac{\partial\phi}{\partial x}+\beta(x,y)\frac{\partial\phi}{\partial y}=0$$
with specified Cauchy data on an initial curve $B$ parameterised by $(x(t),y(t))$.
Note that we immediately have $\alpha\phi_x+\beta\phi_y=u\cdot\nabla\phi=\phi^\prime(s)|_C$ where $C$ is an integral curve of $u=(\alpha,\beta)$.
These are called characteristic curves of the PDE.
The PDE then gives $\phi^\prime(s)=\alpha\phi_x+\beta\phi_y=0$, therefore $\phi$ is constant along the curve $C$.
Therefore the Cauchy data $f(t)$ defined on (sufficiently nice) $B$ at $s=0$ will be propagated constantly along $C$ to give the solution $\phi(s,t)=\phi(x(s,t),y(s,t))=f(t)$.
To get $\phi$, simply invert $s=s(x,y),t=t(x,y)$ (provided it has nonzero Jacobian) and we have $\phi(x,y)=f(t(x,y))$.
\begin{example}
    1. Consider the simple ODE $\partial\phi/\partial x=0$ with $\phi(0,y)=h(y)$ given on the $y$-axis.
    The family of curves we want are the integral curves $x^\prime(s)=\alpha=1,y^\prime(s)=\beta=0$.
    The $y$ axis is parameterised as $(x(t),y(t))=(0,t)$.
    Therefore at $s=0$ we need $(x,y)=(0,t)$, hence the family of curves $C$ are characterised as $x=s,y=t$ (simple indeed).
    Therefore $\phi(s,t)=h(t)$, and by inversion $\phi(x,y)=h(t)$.\\
    2. Let's do some example that are a bit less simple.
    We turn our attention to $e^x\phi_x+\phi_y=0$ with $\phi(x,0)=\cosh x$.
    The characteristic equation is $x^\prime(s)=e^x,y^\prime(s)=1$.
    The initial curve is parameterised as $x(t)=t,y(t)=0$ which shall apply when $s=0$.
    Solving these gives $e^{-x}=e^{-t}-s,y=s$.
    $\phi^\prime(s)=0$ gives $\phi(s,t)=\cosh t$.
    The inversion gives $s=y,t=-\log(y+e^{-x})$, so $\phi(x,y)=\cosh(-\log(y+e^{-x}))$.
\end{example}
So the homogeneous case is easy enough.
How about inhomogeneous ones?
Of course we can do the good old "guess and superposition" manoeuvre, but we can do better.\\
We want to solve $\alpha(x,y)\phi_x+\beta(x,y)\phi_y=\gamma(x,y)$ with specified Cauchy data $\phi(x(t),y(t))=f(t)$ on a curve $B$.
The characteristic curves $C$ satisfy the same system as usual, but there is a twist:
$\phi^\prime(s)|_C=u\cdot\nabla\phi=\gamma(x,y)$ instead of $0$.
So $f(t)$ is no longer propagating constantly and we must actually solve the ODE.
\begin{example}
    Consider $\phi_x+2\phi_y=ye^x$ with $\phi=\sin x$ along $y=x$.
    The characteristic equations are $x^\prime(s)=1,y^\prime(s)=2$.
    On $y=x$, we parameterise $(x(t),y(t))=(t,t)$, which then gives $x=s+t,y=2s+t$.
    Now we turn to $\phi^\prime(s)=\gamma=ye^x=(2s+t)e^{s+t}$ subject to $\phi=\sin t$ at $s=0$.
    By simple integration we obtain $\phi=(2s-2+t)e^{s+t}+C$ where $C$ is constant in $s$.
    The initial conditions then gives $\phi(s,t)=(2s-2+t)e^{s+t}+\sin t+(2-t)e^t$.
    With inversion $s=y-x,t=2x-y$ gives
    $$\phi(x,y)=(y-2)e^x+(y-2x+2)e^{2x-y}+\sin(2x-y)$$
\end{example}
\subsection{Classification of Second-Order Linear PDEs}
In $\mathbb R^2$, the general homogeneous second order linear PDE has the form
$$0=\mathcal L\phi=a\frac{\partial^2\phi}{\partial x^2}+2b\frac{\partial^2\phi}{\partial x\partial y}+c\frac{\partial^2\phi}{\partial y^2}+d\frac{\partial\phi}{\partial x}+e\frac{\partial\phi}{\partial y}+f\phi$$
The principal part of this ODE is defined as
$$\sigma_p(x,y,k_x,k_y)=k^\top Ak=\begin{pmatrix}
    k_x&k_y
\end{pmatrix}\begin{pmatrix}
    a(x,y)&b(x,y)\\
    b(x,y)&c(x,y)
\end{pmatrix}\begin{pmatrix}
    k_x\\
    k_y
\end{pmatrix}$$
The PDEs are classified by the properties of the eigenvalues of $A$.
If $b^2-ac<0$, then the eigenvalues have the same sign, so the equation is elliptic.
If $b^2-ac>0$, then the eignevalues have different signs and the equation is called hyperbolic.
If $b^2-ac=0$, then some eigenvalue is $0$, and the equation is called parabolic.
\begin{example}
    The wave equation is hyperbolic.
    The heat equation is parabolic.
    The Laplace equation is elliptic.
\end{example}
A curve defined by $f(x,y)$ being constant is a characteristic curve for this second order PDE if $(\nabla f)A(\nabla f)^\top =0$.
If the curve can be written as $y=y(x)$, then $f_x/f_y=\mathrm dy/\mathrm dx$, so a substitution gives $a(y^\prime)^2-2by^\prime+c=0$ and hence $y^\prime(x)=(b\pm\sqrt{b^2-ac})/a$.
Therefore this classification makes sense as the sign of $b^2-ac$ determines the behaviour of characteristics.
If the equation is hyperbolic, then $b^2-ac>0$ which gives two distinct solutions;
if it is parabolic, then we get exactly one solution;
but if it is elliptic, then there is simply no (real) characteristic curves in this form.\\
If we can transform these to characteristic coordinates $(u,v)$, the PDE will take the canonical form
$$0=\frac{\partial^2\phi}{\partial u\partial v}+\text{lower order terms}$$
\begin{example}
    Consider $-y\phi_{xx}+\phi_{yy}=0$ which is hyperbolic for $y>0$, elliptic for $y<0$ and parabolic for $y=0$.
    We are interested in the hyperbolic $y>0$ case.
    Then $\mathrm dy/\mathrm dx=\pm y^{-1/2}$ by quadratic formula.
    Integrating gives $(2/3)y^{3/2}\pm x=C_\pm$, $C_\pm$ constants.
    The characteristic coordinates are then set as $u=(2/3)y^{3/2}+ x,v=(2/3)y^{3/2}- x$.
    After a lot of calculations using chain rule, the equation can be transformed into
    $$\phi_{uv}+\frac{1}{6(u+v)}(\phi_u+\phi_v)=0$$
\end{example}
\subsection{General Solution for Wave Equation}
Going back to our old friend
$$\frac{1}{c^2}\frac{\partial^2\phi}{\partial t^2}-\frac{\partial^2\phi}{\partial x^2}=0$$
with initial conditions $\phi(x,0)=f(x),\phi_t(x,0)=g(x)$.
Then the characteristic equation is $\mathrm dx/\mathrm dt=\pm C$, $C$ constant.
Therefore we can do the change of coordinate $u=x-ct,v=x+ct$ which just gives
$$\frac{\partial^2\phi}{\partial u\partial v}=0\implies \phi=G(u)+H(v)=G(x-ct)+H(x+ct)$$
For differentiable $G,H$.
The initial conditions then give the eqautions
$$\begin{cases}
    f(x)=\phi(x,0)=G(x)+H(x)\\
    g(x)=\phi_t(x,0)=-cG^\prime(x)+cH^\prime(x)
\end{cases}$$
Combining them gives
$$H(x)=\frac{1}{2}(f(x)-f(0))+\frac{1}{2c}\int_0^xg(y)\,\mathrm dy,G(x)=\frac{1}{2}(f(x)+f(0))-\frac{1}{2c}\int_0^xg(y)\,\mathrm dy$$
So
$$\phi(x,t)=\frac{f(x-ct)+f(x+ct)}{2}+\frac{1}{2c}\int_{x-ct}^{x+ct}g(y)\,\mathrm dy$$
This means that the wave propagates at $v=c$ and $\phi$ is fully determined by the values of $f,g$ (which is the initial data at $t=0$) in the interval $[x-ct,x+ct]$.
This can be interpreted as the light cone, which gives the causal structure of relativity.
That is, data at $x=x_0$ only influence $[x_0-ct,x_0+ct]$ after time $t$.
    \section{Solving PDEs with Green's Functions}
\subsection{Diffusion Equation and Fourier Transform}
Recall that the heat equation for a conducting wire is
$$\frac{\partial\Theta}{\partial t}-D\frac{\partial^2\Theta}{\partial x^2}$$
with initial condition $\Theta(x,0)=h(x)$ and boundary condition $\Theta\to 0$ as $x\to\pm\infty$.
Taking Fourier transform wrt $x$ gives
$$\frac{\partial}{\partial t}\tilde{\Theta}(k,t)=-Dk^2\tilde\Theta(k,t)\implies\tilde\Theta(k,t)=Ce^{-Dk^2t}$$
for some $C$ that is constant in $t$.
Initial condition $\tilde{\Theta}(k,0)=\tilde{h}(k)$ gives $\tilde\Theta(k,t)=\tilde{h}(k)e^{-Dk^2t}$.
Inverting it gives
\begin{align*}
    \Theta(x,t)&=\frac{1}{2\pi}\int_{-\infty}^\infty \tilde{h}(k)e^{-Dk^2t}e^{ikx}\,\mathrm dk\\
    &=\frac{1}{\sqrt{4\pi Dt}}\int_{-\infty}^\infty h(u)\exp\left( -\frac{(x-u)^2}{4Dt} \right)\,\mathrm du\\
    &=\int_{-\infty}^\infty h(u)S_d(x-u,t)\,\mathrm du
\end{align*}
where $S_d(x,t)=(4\pi Dt)^{-1/2}e^{-x^2/(4Dt)}$ is called the fundamental solution (or diffusion kernel/sourse function), which is just the FT of $e^{-Dk^2t}$.
Note that the case $\Theta(x,0)=\theta_0\delta(x)$ gives our old friend $\Theta=\theta_0(4\pi Dt)^{-1/2}e^{-\eta^2}$ where $\eta=x/(2\sqrt{Dt})$ is the similarity parameter.
\begin{example}
    Suppose initially $f(x)=\theta_0\sqrt{a/\pi}e^{-ax^2}$.
    Then
    \begin{align*}
        \Theta&=\frac{\theta_0\sqrt{a}}{\sqrt{4\pi^2Dt}}\int_{-\infty}^\infty\exp\left( -au^2-\frac{(x-u)^2}{4Dt} \right)\,\mathrm du\\
        &=\frac{\theta_0\sqrt{a}}{\sqrt{4\pi^2Dt}}\int_{-\infty}^\infty\exp\left( -\frac{1+4aDt}{4Dt}\left( u-\frac{x}{1+4aDt} \right)^2 \right)\exp\left( \frac{-ax^2}{1+4aDt} \right)\,\mathrm du\\
        &=\theta_0\sqrt{\frac{a}{\pi(1+4aDt)}}\exp\left( \frac{-ax^2}{1+4aDt} \right)
    \end{align*}
    Asymptotically the width of the Gaussian spreads as $\sqrt{t}$ and the area under the curve being constant (which can be interpreted as the conservation of heat energy).
\end{example}
\subsection{Forced Diffusion Equation}
Consider
$$\frac{\partial\Theta}{\partial t}(x,t)-D\frac{\partial^2\Theta}{\partial x^2}(x,t)=f(x,t)$$
with homogeneous initial condition $\Theta(x,0)=0$.
The Green's function $G$ of this problem would satisfy
$$\frac{\partial G}{\partial t}-D\frac{\partial^2G}{\partial x^2}=\delta(x-\xi)\delta(t-\tau)$$
with $G(x,0;\xi,\tau)=0$.
Take FT wrt $x$ gives
$$\frac{\partial \tilde{G}}{\partial t}(k,t;\xi,\tau)+Dk^2\tilde{G}(k,t;\xi,\tau)=e^{-ik\xi}\delta(t-\xi)$$
Multiplying both sides using the integration factor $e^{Dk^2t}$ allows us to integrate the equation and obtain
$$e^{Dk^2t}\tilde{G}=e^{-ik\xi}\int_0^te^{Dk^2t'}\delta(t'-\tau)\,\mathrm dt'$$
The appearance of Heaviside function is because we need to ensure that $[0,t]$ contains $\tau$ in order to make use of the property of $\delta$ function.
Inverting the whole thing gives
\begin{align*}
    G(x,t;\xi,\tau)&=\frac{H(t-\tau)}{2\pi}\int_{-\infty}^\infty e^{ik(x-\xi)}e^{-Dk^2(t-\tau)}\,\mathrm dk\\
    &=\frac{H(t')}{2\pi}\int_{-\infty}^\infty e^{ikx'}e^{-Dk^2t'}\,\mathrm dk,x'=x-\xi,t'=t-\tau\\
    &=\frac{H(t')}{\sqrt{4\pi Dt'}}e^{-x'^2/(4Dt')}\\
    &=H(t-\tau)S_d(x-\xi,t-\tau)
\end{align*}
where $S_d$ is again our good ol' fundamental solution.
Therefore the general solution is
\begin{align*}
    \Theta(x,t)&=\int_0^\infty\int_{-\infty}^\infty G(x,t;\xi,\tau)f(\xi,\tau)\,\mathrm d\xi\mathrm d\tau\\
    &=\int_0^t\int_{-\infty}^\infty f(u,\tau)S_d(x-u,t-\tau)\,\mathrm du\mathrm d\tau
\end{align*}
which looks very familiar to our previous solution for the homogeneous case with initial conditions at $t=\tau$.
But this time, the initial condition of $\Theta(u,t)=f(u)$ at $t=\tau$ is replaced by the forcing term $f(u,\tau)$ in the equation, and the effect of this, as we see now, is just integrating the solution over $\tau$.
This phenomenon is called Duhamel's Principle which relates the solution of forced PDE with homogeneous boundary conditions to solutions of homogeneous PDEs with inhomogenoeus boundary conditions.
So in this philosophy, the forcing term is acting like an initial conditions for subsequent evolution, and the integral represents a superposition of all these initial-condition-like effects for $0<\tau<t$.
\subsection{Forced Wave Equation}
Consider the forced wave equation
$$\frac{\partial^2\phi}{\partial t^2}-c^2\frac{\partial^2\phi}{\partial x}=f(x,t)$$
subject to $\phi(x,0)=\phi_t(x,0)=0$.
The Green's function for this would satisfy
$$\frac{\partial^2\phi}{\partial t^2}-c^2\frac{\partial^2\phi}{\partial x}=\delta(x-\xi)\delta(t-\tau)$$
with $G=G_t=0$ at $t=0$.
Take FT wrt $x$ again,
$$\frac{\partial^2\tilde{G}}{\partial t^2}+c^2k^2\tilde{G}=e^{-ik\xi}\delta(t-\tau)$$
Solving it yields
$$\tilde{G}=\begin{cases}
    0\text{, if $t<\tau$}\\
    e^{-ik\xi}\sin(kc(t-\tau))/(kc)\text{, if $t>\tau$}
\end{cases}=e^{-ik\xi}\frac{\sin(kc(t-\tau))}{kc}$$
Inverting this gives
\begin{align*}
    G(x,t;\xi,\tau)&=\frac{H(t-\tau)}{2\pi c}\int_{-\infty}^\infty e^{ik(x-\xi)}\frac{\sin(kc(t-\tau))}{k}\,\mathrm dk\\
    &=\frac{H(t-\tau)}{2\pi c}\int_{-\infty}^\infty \frac{\cos(kA)\sin(kB)}{k}\,\mathrm dk,A=x-\xi,B=c(t-\tau)\\
    &=\frac{H(t-\tau)}{2\pi c}\int_{-\infty}^\infty \frac{\sin(k(A+B))-\sin(k(A-B))}{k}\,\mathrm dk\\
    &=\frac{H(t-\tau)}{2\pi c}(\operatorname{sgn}(A+B)-\operatorname{sgn}(A-B))\\
    &=\frac{1}{2c}H(c(t-\tau)-|x-\xi|)
\end{align*}
which is also called the causal fundamental solution since it hints the causal structure in $t$.
The general solution is then
\begin{align*}
    \phi(x,t)&=\int_0^\infty\int_{-\infty}^\infty f(\xi,t)G(x,t;\xi,\tau)\,\mathrm d\xi\mathrm d\tau\\
    &=\frac{1}{2c}\int_0^t\int_{x-c(t-\tau)}^{x+c(t-\tau)}f(\xi,\tau)\,\mathrm d\xi\mathrm d\tau
\end{align*}
which, by relating to our previous solution, can also be seen as an example of Dudamel's principle.
\subsection{Poisson's Equation}
We want to solve Poisson's equation, which is just a forced Laplace equation $\nabla^2\phi=-\rho$ on a domain $D$ subject to Dirichlet boundary conditions $\phi|_{\partial D}=0$.
To get its fundamental solution, note that we can get the notion of $\delta$ function on $\mathbb R^3$ analogously.
So the free space Green's function $G=G_{\rm FS}(\underline{r};\underline{r}')$ is naturally the solution to $\nabla^2G_{\rm FS}(\underline{r};\underline{r}')=\delta(\underline{r}-\underline{r}')$.
We can take this to be spherically symmetric $G(\underline{r};\underline{r}')=G(|\underline{r}-\underline{r}'|)=G(r)$.
Let $B$ be the ball with centre $\underline{r}'$ and radius $r$, then we have
$$1=\int_B\delta(\underline{r}-\underline{r}')\,\mathrm d^3\underline{r}=\int_B\nabla^2G_{\rm FS}\,\mathrm d^3r=\int_{\partial B}\nabla G_{\rm FS}\cdot\underline{n}\,\mathrm dS=4\pi r^2\frac{\partial G_{\rm FS}}{\partial r}$$
So $G_{\rm FS}=-1/(4\pi r)=-1/(4\pi|\underline{r}-\underline{r}'|)$ because of the boundary condition $G\to 0$ as $r\to\infty$.
Therefore we get the general solution
$$\Phi(\underline{r})=\frac{1}{4\pi}\int_D\frac{\rho(\underline{r}')}{|\underline{r}-\underline{r}'|}\,\mathrm d^3\underline{r}'$$
By the way, we can derive the Green's function in 2D in the similar way and get $G_2(\underline{r};\underline{r}')=-(2\pi)^{-1}\log(|\underline{r}-\underline{r}'|)+C_2$ where $C_2$ is a constant (which we almost always set to $0$).\\
We now turn to Green's Identities.
Consider two scalar functions $\phi,\psi$ twice differentiable on $D$.
Then
$$\int_D(\phi\nabla^2\psi+\nabla\phi\cdot\nabla\psi)\,\mathrm d^3\underline{r}=\int_D\nabla\cdot(\phi\nabla\psi)\,\mathrm d^3\underline{r}=\int_{\partial D}\phi\nabla\psi\cdot\underline{\hat{n}}\,\mathrm dS$$
by divergence theorem.
This is called Green's first identity.
Now switch $\psi$ and $\phi$ and substract from the first identity to get Green's second identity.
$$\int_{\partial D}\left( \phi\frac{\partial\psi}{\partial n}-\psi\frac{\partial\phi}{\partial n} \right)\,\mathrm dS=\int_D(\phi\nabla^2\psi-\psi\nabla^2\phi)\,\mathrm d^3\underline{r}$$
Now excise a small spherical ball $B_\epsilon$ around $\underline{r}'$ with radius $\epsilon$.
Now if $\phi$ is a solution to $\nabla^2\phi=-\rho$ and $\psi=G_{\rm FS}(\underline{r};\underline{r}')$, then the right hand side is
$$\int_{D-B_\epsilon}(\phi\nabla^2G_{\rm FS}-G_{\rm FS}\nabla^2\phi)\,\mathrm d^3r=\int_{D-B_\epsilon}G_{\rm FS}\rho\,\mathrm d^3\underline{r}$$
and the left hand side becomes
$$\int_{\partial D}\left( \phi\frac{\partial G_{\rm FS}}{\partial n}-G_{\rm FS}\frac{\partial\phi}{\partial n} \right)\,\mathrm dS+\int_{\partial B_\epsilon}\left( \phi\frac{\partial G_{\rm FS}}{\partial n}-G_{\rm FS}\frac{\partial\phi}{\partial n} \right)\,\mathrm dS$$
Taking $\epsilon\to 0$ gives
$$\int_{\partial B_\epsilon}\left( \phi\frac{\partial G_{\rm FS}}{\partial n}-G_{\rm FS}\frac{\partial\phi}{\partial n} \right)\,\mathrm dS\to-\phi(\underline{r}')$$
Therefore we conclude Green's third identity
$$\phi(\underline{r}')=\int_DG_{\rm FS}(\underline{r};\underline{r}')(-\rho(\underline{r}))\,\mathrm d^3\underline{r}+\int_{\partial D}\left( \phi(\underline{r})\frac{\partial G}{\partial n}(\underline{r};\underline{r}')-G_{\rm FS}(\underline{r};\underline{r}')\frac{\partial\phi}{\partial n}(\underline{r}) \right)\,\mathrm dS$$
Now we want to solve $\nabla^2\phi=-\rho$ on $D$ subject to inhomogeneous Dirichlet boundary conditions $\phi|_{\partial D}=h$.
In this case, we want the Green's function $G=G(\underline{r};\underline{r}')$ to satisfy:\\
(i) $\nabla^2G(\underline{r};\underline{r}')=0$ for any $\underline{r}\neq\underline{r}'$.\\
(ii) $G(\underline{r};\underline{r}')=0$ on $\partial D$.\\
(iii) $G(\underline{r},\underline{r}')=G_{\rm FS}(\underline{r};\underline{r}')+H(\underline{r},\underline{r}')$ for some $H$ such that $\nabla^2H=0$ in $D$.\\
Green's second identity with $\nabla^2\phi=-\rho$ and $\nabla^2H=0$ gives
$$\int_{\partial D}\left( \phi\frac{\partial H}{\partial n}-H\frac{\partial\phi}{\partial n} \right)\,\mathrm dS=\int_DH\rho\,\mathrm d^3\underline{r}$$
The Green's third identity simplifies to
\begin{align*}
    \phi(\underline{r}')&=\int_D(G-H)(-\rho)\,\mathrm d^3\underline{r}+\int_{\partial D}\left( \phi\frac{\partial (G-H)}{\partial n}-(G-H)\frac{\partial\phi}{\partial n} \right)\,\mathrm dS\\
    &=\int_DG(\underline{r};\underline{r}')(-\rho(\underline{r}))\,\mathrm d^3\underline{r}+\int_{\partial D}h(\underline{r})\frac{\partial G(\underline{r};\underline{r}')}{\partial n}\,\mathrm dS
\end{align*}
So this Green's function does give a way to obtain a particular solution.
Also $G(\underline{r};\underline{r}')=G(\underline{r};\underline{r}')$ by a simple use of Green's third identity again.\\
For Neumann boundary conditions with $\partial\phi/\partial n=k(\underline{r})$ on $\partial D$, we can do a similar thing and obtain
$$\phi(\underline{r}')=\int_DG(\underline{r};\underline{r}')(-\rho(\underline{r}))\,\mathrm d^3\underline{r}+\int_{\partial D}G(\underline{r};\underline{r}')(-k(\underline{r}))\,\mathrm dS$$
\subsection{Method of Images}
For symmetric domain $D$ we can construct Green's function with $G=0$ on $\partial D$ by cancelling the boundary potential with an opposite mirror image Green's function placed outside $D$.\\
Our first target is Laplace's equation on the half-plane $D=\{(x,y,z):z>0\}$ subject to $\phi(x,y,0)=h(x,y)$ and $\phi\to 0$ as $|\underline{r}|\to\infty$.
Now $G_{\rm FS}(\underline{r};\underline{r}')\to 0$ as $|\underline{r}|\to\infty$, but it is nonzero at $z=0$.
However, we can take $G(\underline{r};\underline{r}')=G(\underline{r};\underline{r}')-G(\underline{r},\underline{r}'')$ where $\underline{r}''=(x',y',-z')$ which totally works.
Also at $z=0$, $\partial G/\partial n=\partial G/\partial z=(2\pi)^{-1}z'((x-x')^2+(y-y')^2+z'^2)^{-3/2}$.
This means that the solution is
$$\Phi(x',y',z')=\frac{z'}{2\pi}\int_{-\infty}^\infty\int_{-\infty}^\infty((x-x')^2+(y-y')^2+z'^2)^{-3/2}h(x,t)\,\mathrm dx\mathrm dy$$
We now turn to the wave equation for $x>0$.
Consider $\ddot{\phi}-c^2\phi^{\prime\prime}=f$ with Dirichlet boundary conditions $\phi(0,t)=0$.
The same philosophy gives the Green's function
$$G(x,t;\xi,\tau)=\frac{1}{2c}H(c(t-\tau)-|x-\xi|)-\frac{1}{2c}H(c(t-\tau)-|x+\xi|)$$
So if $f=0$ and the initial condition is a Gaussian pause, then this solves to
$$\phi(x,t)=\exp((x-\xi+ct)^2)-\exp((-x-\xi+ct)^2)$$
\end{document}