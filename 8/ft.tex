\section{Fourier Transforms}
\subsection{Introduction}
\begin{definition}
    The Fourier transform (FT) of a function $f(x)$ is
    $$\tilde{f}(k)=(\mathcal F(f))(k)=\int_{-\infty}^\infty f(x)e^{-ikx}\,\mathrm dx$$
    The inverse Fourier transform is
    $$f(x)=(\mathcal F^{-1}(\tilde{f}))(x)=\frac{1}{2\pi}\int_{-\infty}^\infty\tilde{f}(k)e^{ikx}\,\mathrm dk$$
\end{definition}
Beware that there are several conventioned for FT.
$\tilde{f}$ is said to be in the frequency domain.
\begin{theorem}[Fourier Inversion Theorem]
    $\mathcal F^{-1}\circ\mathcal F(f)=f$ given that all integrals involved are well-behaved (a sufficient condition is $f,\tilde{f}$ both being absolutely integrable).
\end{theorem}
\begin{example}
    Take $f(x)=(1/(\sigma\sqrt\pi))e^{-x^2/\sigma^2}$, then we can either find $\tilde{f}$ by completing square or observe that
    $$\tilde{f}(k)=\frac{1}{\sigma\sqrt{\pi}}\int_{-\infty}^\infty e^{-x^2/\sigma^2}e^{-ikx}\,\mathrm dx=\frac{1}{\sigma\sqrt{\pi}}\int_{-\infty}^\infty e^{-x^2/\sigma^2}\cos(kx)\,\mathrm dx$$
    Differentiate under the integral sign,
    \begin{align*}
        \frac{\mathrm d\tilde{f}}{\mathrm dk}&=-\frac{1}{\sigma\sqrt\pi}\int_{-\infty}^\infty xe^{-x^2/\sigma^2}\sin(kx)\,\mathrm dx\\
        &=-\frac{1}{\sigma\sqrt\pi}\int_{-\infty}^\infty\left( \frac{k\sigma^2}{2} \right)e^{-x^2/\sigma^2}\cos(kx)\,\mathrm dx\\
        &=-\frac{k\sigma^2}{2}\tilde{f}(k)
    \end{align*}
    Integrating this differential equation on $\tilde{f}$ gives $\tilde{f}(k)=Ce^{-k^2\sigma^2/4}$ for some constant $C$.
    Setting $k=0$ gives $C=1$, therefore $\tilde{f}(k)=e^{-k^2\sigma^2/4}$.
    One can show that $\mathcal F^{-1}\tilde{f}=f$.
\end{example}
\begin{example}
    The Fourier transform of $f(x)=e^{-a|x|}$ is $\tilde{f}(k)=2a/(a^2+k^2)$ either by direct integration or superposition.
\end{example}
\subsection{Relation with Fourier Series}
Recall that the Fourier series has the form
$$f(x)=\sum_{n=-\infty}^\infty c_ne^{ik_nx},k_n=n\Delta k,\Delta k=\frac{\pi}{L}$$
Now we know that
$$c_n=\frac{1}{2L}\int_{-L}^Lf(x)e^{-ik_nx}\,\mathrm dx=\frac{\Delta k}{2\pi}\int_{-L}^Lf(x)e^{-ik_nx}\,\mathrm dx$$
Imagine taking $L\to\infty,\Delta k\to 0$ then we get the expression of Fourier transform multiplied by an infinitesimal term.
To further justify the analogy, observe that we get
$$f(x)=\sum_{n=-\infty}^\infty\frac{\Delta k}{2\pi}e^{ik_nx}\int_{-L}^Lf(x')e^{-ik_nx'}\,\mathrm dx'$$
This looks like a Riemann sum, so let us take the limit $L\to\infty,\Delta k\to 0$ which gives
$$f(x)=\frac{1}{2\pi}\int_{-\infty}^\infty \left( \int_{-\infty}^\infty f(x')e^{-ikx'}\,\mathrm dx' \right)e^{ikx}\,\mathrm dk=\mathcal F^{-1}\circ\mathcal F(f)(x)$$
Note that when $f$ is discontinuous at $x$, then $\mathcal F^{-1}\circ\mathcal F(f)(x)=(f(x_-)+f(x_+))/2$ which is a similar behaviour to that of a Fourier series.
\subsection{Properties of Fourier Transform}
The operators $\mathcal F,\mathcal F^{-1}$ is both linear.
Also, the translation $h(x)=f(x-\lambda)$ gives $\tilde{h}(k)=e^{-i\lambda k}\tilde{f}(k)$.
Correspondingly, the frequency shift $h(x)=e^{i\lambda x}f(x)$ gives $\tilde{h}(k)=\tilde{f}(k-\lambda)$.
The scaling $h(x)=f(\lambda x)$ for $\lambda\neq 0$ gives $\tilde{h}(k)=\tilde{f}(k/\lambda)/|\lambda|$.\\
These are all trivial facts, here is a slightly more interesting one:
$h(x)=xf(x)$ gives $\tilde{h}(k)=i\tilde{f}^\prime(k)$.
Indeed,
$$\tilde{h}(k)=\int_{-\infty}^\infty xf(x)e^{-ikx}\,\mathrm dx=-\frac{1}{i}\frac{\mathrm d}{\mathrm dk}\int_{-\infty}^\infty f(x)e^{-ikx}\,\mathrm dx=i\tilde{f}^\prime(k)$$
What's more important is that if $f$ vanishes at $\pm\infty$ and $h(x)=f^\prime(x)$ gives $\tilde{h}(k)=ik\tilde{f}(k)$ via integration by parts
$$\tilde{h}(k)=\int_{-\infty}^\infty f^\prime(x)e^{-ikx}\,\mathrm dx=[f(x)e^{-ikx}]_{-\infty}^\infty-\int_{-\infty}^\infty (-ik)f(x)e^{-ikx}\,\mathrm dx=ik\tilde{f}(k)$$
So we can employ Fourier transform to turn a (nice enough) differential equation into an algebraic one.\\
We also have a sense of duality between $x$ and $k$ here.
Observe that we have
$$f(-x)=\frac{1}{2\pi}\int_{-\infty}^\infty\tilde{f}(k)e^{-ikx}\,\mathrm dk,f(-k)=\frac{1}{2\pi}\int_{-\infty}^\infty\tilde{f}(x)e^{-ikx}\,\mathrm dx$$
Therefore $g(x)=\tilde{f}(x)$ iff $\tilde{g}(k)=2\pi f(-k)$..
So $f(-x)=(2\pi)^{-1}\mathcal F^2(f)(x)$.
Iterating this gives $\mathcal F^4(f)(x)=4\pi^2f(x)$.
\begin{example}
    Consider the ``top hat'' function
    $$f(x)=\begin{cases}
        1\text{, if $|x|\le a$}\\
        0\text{, if $|x|>a$}
    \end{cases}$$
    We get
    $$\tilde{f}(k)=\int_{-\infty}^\infty f(x)e^{-ikx}\,\mathrm dx=\int_{-a}^a\cos(kx)\,\mathrm dx=\frac{2\sin(ka)}{k}$$
    Fourier inversion theorem then gives
    $$\frac{1}{\pi}\int_{-\infty}^\infty e^{ikx}\frac{\sin ka}{k}\,\mathrm dk=\begin{cases}
        1\text{, if $|x|<a$}\\
        0\text{, if $|x|>a$}
    \end{cases}$$
    Set $x=0$ and $k\to x$ gives
    $$\int_0^\infty\frac{\sin(ax)}{x}\,\mathrm dx=\frac{\pi}{2}\operatorname{sgn}(a)=\begin{cases}
        \pi/2\text{, if $a>0$}\\
        0\text{, if $a=0$}\\
        -\pi/2\text{, if $a<0$}
    \end{cases}$$
    which is pretty awesome.
\end{example}
\subsection{Convolution and Parseval's Theorem}
Recall that
\begin{definition}
    The convolution of $f$ and $g$ is
    $$(f\ast g)(x)=\int_{-\infty}^\infty f(y)g(x-y)\,\mathrm dy$$
\end{definition}
We want to multiply together functions in frequency domain, that is $\tilde{h}=\tilde{f}\tilde{g}$, and find its inverse Fourier transform.
\begin{align*}
    h(x)&=\frac{1}{2\pi}\int_{-\infty}^\infty\tilde{f}(k)\tilde{g}(k)e^{ikx}\,\mathrm dk\\
    &=\frac{1}{2\pi}\int_{-\infty}^\infty\left( \int_{-\infty}^\infty f(y)e^{-iky}\,\mathrm dy \right)\tilde{g}(k)e^{ikx}\,\mathrm dk\\
    &=\int_{-\infty}^\infty f(y)\left( \frac{1}{2\pi}\int_{-\infty}^\infty\tilde{g}(k)e^{ik(x-y)}\,\mathrm dk \right)\,\mathrm dy\\
    &=\int_{-\infty}^\infty f(y)g(x-y)\,\mathrm dy\\
    &=(f\ast g)(x)
\end{align*}
By duality, we also have
$$h(x)=f(x)g(x)\implies \tilde{h}(k)=\frac{1}{2\pi}\int_{-\infty}^\infty \tilde{f}(p)\tilde{g}(k-p)\,\mathrm dp=\frac{1}{2\pi}(\tilde{f}\ast\tilde{g})(k)$$
Now consider $h(x)=g^*(-x)$, then
\begin{align*}
    \tilde{h}(k)&=\int_{-\infty}^\infty g^\ast(-x)e^{-ikx}=\left( \int_{-\infty}^\infty g(-x)e^{ikx}\,\mathrm dx \right)^*\\
    &=\left( \int_{-\infty}^\infty g(y)e^{-iky}\,\mathrm dy \right)^*=\tilde{g}^*(k)
\end{align*}
By our study of convolution we have
$$\int_{-\infty}^\infty f(y)g^*(y-x)\,\mathrm dy=\frac{1}{2\pi}\int_{-\infty}^\infty\tilde{f}(k)\tilde{g}^*(k)e^{ikx}\,\mathrm dx$$
Set $x\to 0$ gives
$$\int_{-\infty}^\infty f(y)g^*(y)\,\mathrm dy=\frac{1}{2\pi}\int_{-\infty}^\infty\tilde{f}(k)\tilde{g}^*(k)\,\mathrm dk\implies\langle g,f\rangle=\frac{1}{2\pi}\langle \tilde{g},\tilde{f}\rangle$$
Setting $g=f$ then gives
\begin{theorem}[Parseval's Theorem]
    $$\int_{-\infty}^\infty|f(x)|^2\,\mathrm dx=\frac{1}{2\pi}\int_{-\infty}^\infty |\tilde{f}(k)|^2\,\mathrm dk$$
\end{theorem}
\subsection{Fourier Transform of Generalised Functions}
We want to apply $\mathcal F$ to generalised functions.
If one wants to be precise, one can view them as the limit of the sequence of the Fourier transforms of well-defined well-behaved functions that approaches the generalised function in question.
The details can be justified by Parseval's theorem, but the treatment is beyond the scope of this course.\\
Of course the main culprit of generalised functions is $\delta$.
Surprisingly, $\delta$ is found naturally in Fourier transform.
For a well-behaved $f$,
\begin{align*}
    f(x)&=\mathcal F^{-1}(\mathcal F(f))(x)=\frac{1}{2\pi}\int_{-\infty}^\infty\int_{-\infty}^\infty f(u)e^{-iku}e^{ikx}\,\mathrm du\mathrm dk\\
    &=\int_{-\infty}^\infty f(u)\left( \frac{1}{2\pi}\int_{-\infty}^\infty e^{ik(x-u)}\,\mathrm dk \right)\,\mathrm du
\end{align*}
So we might identify
\footnote{The number of things that would go wrong with this is overwhelming, but what the hell.}
$$\delta(x-u)=\frac{1}{2\pi}\int_{-\infty}^\infty e^{ik(x-u)}\,\mathrm dk$$
A direct calculation, on the other hand yields
$$\tilde\delta(k)=\int_{-\infty}^\infty \delta(x)e^{ikx}\,\mathrm dx=1$$
So dually, the Fourier transform of the constant $f(x)=1$ is $\tilde{f}(k)=2\pi\delta(x)$.
Also $f(x)=\delta(x-a)$ has $\tilde{f}(k)=e^{ika}$.
If we have exponentials then we naturally have trigonometrics, so here goes:
$f(x)=\cos(\omega x)$ has $\tilde{f}(k)=\pi(\delta(k+\omega)+\delta(k-\omega))$ and $f(x)=\sin(\omega x)$ has $\tilde{f}(k)=i\pi (\delta(k+\omega)-\delta(k-\omega))$.\\
Let's move on to something slightly better.
Consider the Heaviside function
$$H(x)=\begin{cases}
    1\text{, if $x>0$}\\
    0\text{, if $x<0$}\\
    1/2\text{, if $x=0$}
\end{cases}$$
Then $H(x)+H(-x)=1$ for any $x$.
Therefore $\tilde{H}(k)+\tilde{H}(-k)=2\pi\delta(k)$.
But $H^\prime(x)=\delta(x)$, hence $ik\tilde{H}(k)=1$.
So for these results to be consistent (note that $k\delta(k)=1$), we had to have $\tilde{H}(k)=\pi\delta(k)+(ik)^{-1}$.
The formula can look nicer if we shift a bit and consider instead $f(x)=\operatorname{sgn}(x)/2$ which has $\tilde{f}(k)=(ik)^{-1}$.
\subsection{Applications of Fourier Transforms}
The first very important application of Fourier transforms is in boundary value problems of ODEs.
Consider the problem $y^{\prime\prime}-y=f(x)$ with homogeneous boundary conditions $y\to 0$ as $x\to\pm\infty$.
Apply Fourier transform on both sides gives
$$(-k^2-1)\tilde{y}=\tilde{f}\implies \tilde{y}(k)=\tilde{f}(k)\tilde{g}(k),\tilde{g}(k)=-\frac{1}{1+k^2}$$
Therefore
\begin{align*}
    y(x)&=\int_{-\infty}^\infty f(u)g(x-u)\,\mathrm du\\
    &=-\frac{1}{2}\int_{-\infty}^\infty f(u)e^{-|x-u|}\,\mathrm du\\
    &=-\frac{1}{2}\int_{-\infty}^xf(u)e^{u-x}\,\mathrm du-\frac{1}{2}\int_x^\infty f(u)e^{x-u}\,\mathrm du
\end{align*}
which is in the form of the solution we would have obtained if we use Green's function instead.
Of course, we can solve this by inverse Fourier transform as well, which is quite easy once we introduce Fast Fourier Transform (FFT).
We will go through it later.\\
Another motivation of Fourier transform in signal processing.
Suppose we are given some sort of input signal $J(t)$ which is acted on by some linear differential operator $\mathcal L$ to yield output $O(t)$.
The Fourier transform
$$\tilde{J}(\omega)=\int_{-\infty}^\infty J(t)e^{i\omega t}\,\mathrm dt$$
is called the resolution.
In the frequency domain, the action of $\mathcal L$ in $J(t)$ is just multiplying $\tilde{J}(\omega)$ by a transfer function $\tilde{R}\omega$ to yield the output
$$O(t)=\frac{1}{2\pi}\int_{-\infty}^\infty\tilde{R}(\omega)\tilde{J}(\omega)e^{i\omega t}\,\mathrm d\omega$$
The inverse Fourier transform $R$ of $\tilde{R}$ is called the response function.
So $O$ is simply $J\ast R$.
For example, if there is no input $J(t)=0$ for $t<0$.
By causality, we expect $R(t)=0$ for $t<0$.
Therefore
$$O(t)=\int_0^tJ(u)R(t-u)\,\mathrm du$$
which is in the same form as the Green's function in an initial value problem.\\
We want to explore the general transfer functions for a class of ODEs.
Suppose the input/output relation is
$$\mathcal LO=\left( \sum_{i=0}^na_i\frac{\mathrm d^i}{\mathrm dx^i} \right)O=J$$
where $a_i$ are constants.
Taking Fourier transform gives
$$(a_0+a_1(i\omega)+\ldots,a_n(i\omega)^n)\tilde{O}(\omega)=\tilde{J}(\omega)$$
Therefore $\tilde{R}(\omega)=(a_0+a_1(i\omega)+\ldots,a_n(i\omega)^n)^{-1}$.
Factorise the polynomial to turn it into the form $\tilde{R}(\omega)=((i\omega-c_1)^{k_1}\cdots (i\omega-c_r)^{k_r})^{-1}$ where $i\neq j\implies c_i\neq c_j$.
Partial fraction gives
$$\tilde{R}(\omega)=\frac{1}{(i\omega-c_1)^{k_1}\cdots (i\omega-c_r)^{k_r}}=\sum_{j=1}^r\sum_{m=1}^{k_j}\frac{\Gamma_{jm}}{(i\omega-c_j)^m}$$
for constants $\Gamma_{jm}$.
But we know that
$$\mathcal F^{-1}\left( \frac{1}{(i\omega-a)^m} \right)=\begin{cases}
    t^{m-1}e^{at}/(m-1)!\text{, if $t>0$}\\
    0\text{, if $t<0$}
\end{cases}$$
Getting back to time domain, we obtain the response function
$$R(t)=\sum_{j=1}^r\sum_{m=1}^{k_j}\Gamma_{jm}\frac{t^{m-1}}{(m-1)!}e^{c_jt},t>0$$
\begin{example}
    Consider the damp oscillator $\mathcal Ly=y^{\prime\prime}+2py^\prime+(p^2+q^2)y=f(t)$ with damping $p>0$ with homogeneous initial conditions $y(0)=y^\prime(0)=0$.
    The Fourier transform of this is $(i\omega)^2\tilde{y}+2ip\omega\tilde{y}+(p^2+q^2)\tilde{y}=\tilde{f}$, so
    $$\tilde{y}=\frac{\tilde{f}}{-\omega^2+2ip\omega+p^2+q^2}=\tilde{R}\tilde{f},\tilde{R}=\frac{1}{-\omega^2+2ip\omega+p^2+q^2}$$
    So
    $$y(t)=\int_0^tR(t-\tau)f(\tau)\,\mathrm d\tau,R(t-\tau)=\frac{1}{2\pi}\int_{-\infty}^\infty\frac{\exp(i\omega(t-\tau))\,\mathrm d\omega}{p^2+q^2+2ip\omega-\omega^2}$$
    which, as one can verify, is analogous to the Green's function methods since $\mathcal LR(t-\tau)=\delta(t-\tau)$.
\end{example}
\subsection{Discrete Fourier Transform}
Suppose we sample a signal $h(t)$ at equal times $t_n=n\Delta$ with time-sampling $\Delta$ and values $h_n=h(n\Delta)$ with $n\in\mathbb Z$.
That is, the sampling frequency is $f_s=1/\Delta$ (and angular frequency $\omega_s=2\pi f_s=2\pi/\Delta$).
The Nyquist frequency $f_c=1/(2\Delta)$ is the highest frequency actually sampled at $\Delta$.
Suppose we have a (sinusoidal) signal with a given frequency $f$
$$g_f(t)=A\cos(2\pi ft+\phi)=\frac{A}{2}(e^{i\phi}e^{2\pi ift}+e^{-i\phi}e^{-2\pi ift})$$
What happens if we sample at $f=f_c$?
We have
$$g_{f_c}(t_n)=A\cos\left( 2\pi\frac{1}{2\Delta}n\Delta+\phi \right)=(A\cos\phi)\cos(\pi n)=A'\cos(2\pi f_ct_n)$$
So information about phase and amplitude are lost.
Even worse if we sample above $f>f_c$:
If we sample at $f=f_c+\delta f$ for some small $\delta f>0$, then
$$g_f(t_n)=A\cos(2\pi(f_c+\delta f)t_n+\phi)=A\cos(2\pi(f_c-\delta f)t_n-\phi)$$
So the effect is just aliasing a ``ghost signal'' to frequency $f_c-\delta f$ (or $-(f_c-\delta f)$), which is a contamination of the information.
\begin{definition}
    A signal $g(t)$ is bandwidth limited if it contains no frequency above some $\omega_{\max{}}=2\pi f_{\max{}}$, that is $\tilde{g}(\omega)=0$ for any $|\omega|>\omega_{\max{}}$.
\end{definition}
So for a bandwidth limited signal $g(t)$ would have
$$g(t)=\frac{1}{2\pi}\int_{-\infty}^\infty\tilde{g}(\omega)e^{i\omega t}\,\mathrm d\omega=\frac{1}{2\pi}\int_{-\omega_{\max{}}}^{\omega_{\max{}}}\tilde{g}(\omega)e^{i\omega t}\,\mathrm d\omega$$
\begin{theorem}[Sampling Theorem]
    Let $g$ be a bandwidth limited signal and $\Delta=1/(2f_{\max{}})$, then define
    $$g_n=g(t_n)=\frac{1}{2\pi}\int_{-\omega_{\max{}}}^{\omega_{\max{}}}\tilde{g}(\omega)e^{i\pi n\omega/\omega_{\max{}}}\,\mathrm d\omega$$
    which induces a Fourier series
    $$\tilde{g}_{\mathrm{per}}(\omega)=\frac{\pi}{\omega_{\max{}}}\sum_{n=-\infty}^\infty g_ne^{-i\pi n\omega/\omega_{\max{}}}$$
    Then, we have
    $$\tilde{g}(\omega)=\tilde{g}_{\mathrm{per}}(\omega)\tilde{h}(\omega),\tilde{h}(\omega)=\begin{cases}
        1\text{, if $|\omega|\le\omega_{\max{}}$}\\
        0\text{, otherwise}
    \end{cases}$$
    Inverting which gives
    \begin{align*}
        g(t)&=\frac{1}{2\omega_{\max{}}}\sum_{n=-\infty}^\infty g_n\int_{-\omega_{\max{}}}^{\omega_{\max{}}}\exp\left( i\omega\left( t-\frac{n\pi}{\omega_{\max{}}} \right) \right)\,\mathrm d\omega\\
        &=\sum_{n=-\infty}^\infty g_n\frac{\sin(\omega_{\max{}} t-\pi n)}{\omega_{\max{}} t-\pi n}
    \end{align*}
    So $g(t)$ can be exactly represented after sampling at discrete times $t_n$.
\end{theorem}
\begin{proof}
    Self-explanatory.
\end{proof}
Suppose we have a finite number $N$ of samples $h_m=h(t_m)$ where $t_m=m\Delta$ for $m=0,\ldots,N-1$.
We want to approximate the Fourier Transform for $N$ frequencies within $f_c=1/(2\Delta)$ with equally spaced frequencies with space $\Delta_f=1/(N\Delta)$ in the range $[-f_c,f_c]$.
So basically we are just looking for $f_n=n\Delta_f=n/(N\Delta)$ where $n=-N/2,\ldots,0,\ldots,N/2$.
Note that $f_c$ and $-f_c$ are aliased together, so the $-N/2$ and $N/2$ are basically the same.
Also $(m+N/2)\Delta_f=f_c+\delta f$ is aliased to $(-m+N/2)\Delta_f=-(f_c-\delta f)$, so we choose instead $f_n=n/(N\Delta)$ with $n=0,\ldots,N-1$.
\begin{definition}
    Observe that
    \begin{align*}
        \tilde{h}(f_n)&=\int_{-\infty}^\infty h(t)e^{-2\pi if_nt}\,\mathrm dt\approx\Delta\sum_{n=0}^{N-1}h_me^{-2\pi i f_nt_m}\\
        &=\Delta\sum_{m=0}^{N-1}h_me^{-2\pi imn/N}=\Delta\tilde{h}_d(f_n)
    \end{align*}
    Here $\tilde{h}_d(f_n)=\tilde{h}_n$ is the discrete Fourier transform (DFT).
\end{definition}
So the matrix $[\operatorname{DFT}]_{mn}=e^{-2\pi imn/N}$ defines the discrete Fourier transform for $h=\{h_m\}$ as we have $\tilde{h}_d=[\operatorname{DFT}]h$.
The inverse of this matrix is its adjoint, i.e. $[\operatorname{DFT}]^{-1}=N^{-1}[\operatorname{DFT}]^\dagger$ and it is built from the $N^{th}$ roots of unity.
\begin{example}
    If $N=4$ and $\omega=-i$, then
    $$[\operatorname{DFT}]=\begin{pmatrix}
        1&1&1&1\\
        1&-i&-1&i\\
        1&-1&1&-1\\
        1&i&-1&-i
    \end{pmatrix}$$
\end{example}
The inverse DFT is
\begin{align*}
    h_m&=h(t_m)\approx\frac{1}{2\pi}\int_{-\infty}^\infty\tilde{h}(\omega)e^{i\omega t_m}\,\mathrm d\omega=\int_{-\infty}^\infty\tilde{h}(f)e^{2\pi ift_m}\,\mathrm df\\
    &\approx\frac{1}{N\Delta}\sum_{n=0}^{N-1}\Delta\tilde{h}_d(f_n)e^{2\pi imn/N}\\
    &=\frac{1}{N}\sum_{n=0}^{N-1}\tilde{h}_ne^{2\pi imn/N}
\end{align*}
In this frame, we can establish analogues of Parseval's theorem
$$\sum_{m=0}^{N-1}|h_m|^2=\frac{1}{N}\sum_{n=0}^{N-1}|\tilde{h}_n|^2$$
and convolution theorem
$$c_k=\sum_{m=0}^{N-1}g_mh_{k-m}\iff \tilde{c}_k=\tilde{g}_k\tilde{h}_k$$
This looks complicated, but there is actually a very efficient algorithm to do it, known as fast Fourier transform (FFT), by exploiting the symmetries of the expression.