\section{Sturm-Liouville Theory}
\subsection{Review of Second-Order Linear ODEs}
For a general inhomogeneous ODE $\mathcal Ly=f(x)$ where
$$\mathcal Ly=\alpha(x)\frac{\mathrm d^2y}{\mathrm dx^2}+\beta(x)\frac{\mathrm dy}{\mathrm dx}+\gamma(x)y$$
In general, the homogeneous equation $\mathcal Ly=0$ has two linearly independent solutions $y_1,y_2$.
The complementary function $y_c(x)=Ay_1+By_2$ for constants $A,B$ is then the general solution to $\mathcal Ly=0$ by linearity.\\
If we can find a particular solution (aka particular integral) $y_p$ to $\mathcal Ly=f$, then $y_p+y_c=y_p+Ay_1+By_2$ for $A,B$ constants is the general solution to $\mathcal Ly=f$ again by linearity.
Two pieces of boundary data is then needed to determine the constants $A,B$.\\
There are several types of boundary conditions.
We sometimes get the Dirichlet condition of specifying the function's value at the endpoints, or the Neumann consitions of specifying the derivative's values at the endpoints.
Sometimes these two types of conditions are mixed.\\
The sort of conditions we often consider are homogeneous conditions, i.e. the function vanishes at the endpoints.
The reason of it is that it allows the superposition of solutions in a linear DE.
What if we come across a nonhomogeneous condition?
We can use the complementary solution to cancel stuff out.\\
Sometimes, we specify initial data of the function and its derivative as boundary conditions.\\
Another matter of interest is the general eigenvalue problem.
To solve $\mathcal Ly=f$ using eigenvalue decompositions like we did previously, we must first solve (subject to boundary conditions) the related eigenvalue problem
$$\mathcal Ly=\alpha(x)\frac{\mathrm d^2y}{\mathrm dx^2}+\beta(x)\frac{\mathrm dy}{\mathrm dx}+\gamma(x)y=-\lambda\rho(x)y$$
where $\rho$ is nonegative.
This form often occurs after seperation of variables in a PDE.
\subsection{Self-Adjoint Operators}
\begin{definition}
    For two functions $f,g:[a,b]\to\mathbb C$ we define their inner product to be
    $$\langle f,g\rangle=\int_a^bf^*(x)g(x)\,\mathrm dx$$
\end{definition}
We can guarantee to rewrite the original eigenvalue problem into the Sturm-Liouville form, i.e. $\mathcal Ly=\lambda wy$ where we are able to rewrite $\mathcal Ly=-(py^\prime)^\prime+qy$ and $w$ is a nonnegative wavefunction.
\footnote{The reason why there is a wavefunction there is just for convenience.}
How to convert a second order linear ODE to this form?
Simply multiply the diffential equation by an integrating factor $F$ that will be specified later and we can write
$$\frac{\mathrm d}{\mathrm dx}(F\alpha y^\prime)-F^\prime\alpha y^\prime-F\alpha^\prime y^\prime+F\beta y^\prime+F\gamma y=-\lambda F\rho y$$
So to eiminate the $y^\prime$ terms, we set
$$F(x)=\exp\left(\int\frac{\beta-\alpha^\prime}{\alpha}\,\mathrm dx\right)$$
which reduced the equation to
$$(F\alpha y^\prime)^\prime+F\gamma y=-\lambda F\rho y$$
Setting $p=F\alpha,q=F\gamma$ and $w=F\rho\ge 0$.
\begin{example}
    Consider the Hermite equation that appears in quantum mechanics
    $$y^{\prime\prime}-2xy^\prime+2ny=0$$
    Then $\alpha=1,\beta=-2x,\gamma=0,\lambda\rho=2n$, so the above procedure translates this to the Sturm-Liouville form
    $$\mathcal L=(-e^{-x^2}y^\prime)^\prime=2ne^{-x^2}y$$
\end{example}
\begin{definition}
    Let $\mathcal L:C\to C$ be an operator, where $C$ on a class of functions $[a,b]\to\mathbb C$ equipped with the inner product we defined previously.
    This operator $\mathcal L$ is self-adjoint if $\langle y_1,\mathcal Ly_2\rangle=\langle\mathcal Ly_1,y_2\rangle$ for any $y_1,y_2\in C$.
\end{definition}
If we let $\mathbb L$ be the operator in the Strum-Liouville form, then
\begin{align*}
    \langle y_1,\mathcal Ly_2\rangle-\langle\mathcal Ly_1,y_2\rangle&=\int_a^b[-y_1(py_2^\prime)^\prime+y_1qy_2+y_2(py_1^\prime)^\prime-y_2qy_1]\,\mathrm dx\\
    &=\int_a^b[-y_1(py_2^\prime)^\prime+y_2(py_1^\prime)^\prime]\,\mathrm dx\\
    &=\int_a^b[-(y_1(py_2^\prime)^\prime+y_1^\prime py_2^\prime)+(y_2(py_1^\prime)^\prime+y_2^\prime py_1^\prime)]\,\mathrm dx\\
    &=\int_a^b[-(py_1y_2^\prime)^\prime+(py_1^\prime y_2)^\prime]\,\mathrm dx\\
    &=[-py_1y_2^\prime+py_1^\prime y_2]_a^b
\end{align*}
So for this operator to be self-adjoint, we need some good enough boundary conditions so that enough stuff vanishes.
This includes homogeneous boundary condition $y(a)=y(b)=0$ or $y^\prime(a)=y^\prime(b)=0$ or mixed $y+ky^\prime=0$ etc..
We say a Sturm-Liouville problem is regular if the boundary conditions are homogeneous.
Periodic boundary conditions also work, where we can take $y(a)=y(b)$ and the derivatives are specified (or periodic) at the boundary.
There can also be singular points of this ODE, where $p(a)=p(b)=0$.
We can have combinations of above too.
\subsection{Properties of Self-Adjoint Operators}
\begin{definition}
    The inner product of $y_1,y_2:[a,b]\to\mathbb C$ with respect to weight $w:[a,b]\to\mathbb R_{\ge 0}$ is
    $$\langle f,g\rangle_w=\int_a^bwf^*g\,\mathrm dx=\langle wf,g\rangle=\langle f,wg\rangle$$
\end{definition}
Analogous to the finite dimensional case, we have
\begin{theorem}\label{self-adjoint}
    For a sufficiently nice self-adjoint operator $\mathcal L$ on a sufficiently nice space of functions:\\
    (a) Eigenvalues of $\mathcal L$ are real.\\
    (b) Eigenfunctions of it with different eigenvalues are orthogonal with respect to the weight $w$.\\
    (c) We can take the eigenfunctions as a set of basis for the function space, just like Fourier series.
\end{theorem}
\begin{proof}[Proof of (a)]
    1. If $\mathcal Ly=\lambda wy$, taking complex conjugate gives $\mathcal Ly^*=\lambda^*wy^*$.
    Hence as $\mathcal L$ is self-adjoint,
    $$0=\int_a^b(y^*\mathcal Ly-y\mathcal Ly^*)\,\mathrm dx=(\lambda-\lambda^*)\int_a^bw|y|^2\,\mathrm dx$$
    which means $\lambda=\lambda^*$, so $\lambda$ is real.
\end{proof}
If $\lambda$ is non-degenerate (simple), i.e. it has a one-dimensional eigenspace, then $y$ is guaranteed to be real.
Even if it has dimension $2$ (not more because the ODE is second order), we can still find two real functions as basis of the eigenspace.
Also, by considering $u\mathcal Lv-v\mathcal Lu=(-p(uv^\prime-u^\prime v))^\prime$, one can show that a regular Sturm-Liouville problem always has all eigenvalues simple.
\begin{proof}[Proof of (b)]
    Suppose $\mathcal Ly_m=\lambda_mwy_m$ and $\mathcal Ly_n=\lambda_nwy_n$, then
    $$0=\int_a^by_n\mathcal Ly_m-y_m\mathcal Ly_n\,\mathrm dx=(\lambda_m-\lambda_n)\int_a^bwy_ny_m\,\mathrm dx$$
    But $\lambda_m$ and $\lambda_n$ are distinct.
    The claim follows.
\end{proof}
As an aside, we do not really need the weight function in order to formulate Sturm-Liouville theory, since we can do the transformation $\tilde{y}=\sqrt{w}y$ and replace $\mathcal Ly$ by $(1/\sqrt{w})\mathcal L(\tilde{y}/\sqrt{w})$.
Yet the analytic property is generally simpler if we keep $w$.\\
What?
How about (c), you say?
Bold of you to assume we'll prove it.
We are just gonna take it (and several other properties we want it to have) as truth and do stuff with this idea.
\subsection{Eigenfunction Expansions}
So basically we just want to find an expansion
$$f=\sum_{n=1}^\infty a_ny_n$$
where $y_n$ is a set of eigenfunctions of some self-adjoint operator.
Theorem \ref{self-adjoint}(c) shows that we can do it.
To find the coefficients $a_n$, we can they use the orthogonality to get
$$\int_a^bwy_mf\,\mathrm dx=\sum_{n=1}^\infty a_n\int_a^bwy_ny_m\,\mathrm dx=a_m\int_a^bwy_n^2\,\mathrm dx$$
So
$$a_n=\left(\int_a^bwy_nf\,\mathrm dx\middle)\right/\left(\int_a^bwy_n^2\,\mathrm dx\right)$$
It's a common practice not to normalise the eigenfunctions as it is not really always clean.
Of course, if we want, we can always write down
$$Y_n=y_n\left/\sqrt{\int_a^bwy_n^2\,\mathrm dx}\right.$$
So we can get rid of the denominator in $a_n$ and the coefficients will have the expression
$$A_n=\int_a^bwy_nf\,\mathrm dx=a_n\int_a^bwy_n^2\,\mathrm dx$$
but it isn't that useful and can cause some messiness.
\begin{example}
    Recall the particular operator already in Sturm-Liouville form $\mathcal Ly=y^{\prime\prime}$, then (with appropriate boundary conditions) we can easily get the eigenvalues $\lambda_n=(n\pi/L)^2$ and eigenfunctions being the trigonometrics.
    This just reproduces the Fourier series.
\end{example}
\subsection{Completeness and Parseval's Identity}
We expand
\begin{align*}
    0&=\int_a^bw\left( f(x)-\sum_{n=1}^\infty a_ny_n \right)^2\,\mathrm dx\\
    &=\int_a^bw\left( f^2-2f\sum_{n=1}^\infty a_ny_n+\sum_{n=1}^\infty a_n^2y_n^2 \right)\,\mathrm dx\\
    &=\int_a^bwf^2\,\mathrm dx-\sum_{n=1}^\infty a_n^2\int_a^bwy_n^2\,\mathrm dx
\end{align*}
Hence we have
$$\int_a^bwf^2\,\mathrm dx=\sum_{n=1}^\infty a_n^2\int_a^bwy_n^2\,\mathrm dx=\sum_{n=1}^\infty A_n^2$$
which Parseval's identity in this general case.
Easily our previous Parseval's Theorem on Fourier series is a special case.\\
If some of the eigenfunctions are missing from the series, then this gives
$$\int_a^bwf^2\,\mathrm dx\ge\sum_{n=1}^\infty A_n^2$$
This is known as Bessel's Inequality.\\
Consider the partial sums $\sum_{n\le N}a_ny_n$, then we shall have $S_N\to f$ as $N\to\infty$ where we would like the style of convergence to be
$$\epsilon_N=\int_a^bw[f(x)-S_N(x)]^2\,\mathrm dx\to 0,N\to\infty$$
An interesting question is that, while we know (maybe) the series converges as we want, if we truncate the sequence in some $N$, would the coefficients $\{a_n\}_{n\le N}$ provide the best approxmation (with respect to the error defined in this way) of that particular partial sum, or a different set of partial coefficient will yield a better result?
To answer this, we evaluate
$$\frac{\partial\epsilon_N}{\partial a_n}=-2\int_a^bwy_n\left( f-\sum_{k=1}^Na_ky_k \right)\,\mathrm dx=-2\int_a^bwfy_n-a_nwy_n^2\,\mathrm dx$$
which is zero when $a_n$ is of the expression we got earlier.
We can see it is indeed a minimum by observing that
$$\frac{\partial^2\epsilon_N}{\partial a_n^2}=2\int_a^bwy_n^2\,\mathrm dx\ge 0$$
This answers our question.
\subsection{Legendre's Equation}
Take the usual spherical polar coordinate
$$\begin{cases}
    x=r\sin\theta\cos\phi\\
    y=r\sin\theta\sin\phi\\
    z=r\cos\theta
\end{cases}$$
where Laplace's equation $\nabla^2 u=0$ translates to
$$\frac{1}{r^2}\frac{\partial}{\partial r}\left( r^2\frac{\partial u}{\partial r} \right)+\frac{1}{r^2\sin\theta}\frac{\partial}{\partial\theta}\left( \sin\theta\frac{\partial u}{\partial\theta} \right)+\frac{1}{r^2\sin^2\theta}\frac{\partial^2u}{\partial\phi^2}=0$$
Seperation of variables $u=R(r)\Theta(\theta)\Phi(\phi)$ then gives
$$\frac{1}{\sin\theta}(\Theta^\prime\sin\theta)^\prime+\left( K-\frac{m^2}{\sin^2\theta} \right)\Theta=0$$
where $K,m$ are constants which essentially makes it an eigenvalue problem.
Now the transformation $x=\cos\theta\in[-1,1]$ and renaming $\Theta$ as $y$ then gives Legendre's Equation
$$(1-x^2)y^{\prime\prime}-2xy^\prime+\lambda y=0$$
where $\lambda$ is a constant which is again intepreted as an eigenvalue.
This is already in Strum-Liouville form by taking $p=1-x^2,q=0,w=1$.
Now $p=1-x^2$ vanishes at the boundary $\pm 1$, so this equation has to be self-adjoint.
We assume that $y$ is bounded near the boundary.\\
We now seek a power series solution to the problem.
If we set
$$y=\sum_{n=0}^\infty c_nx^n$$
Then substitution gives
$$(n+2)(n+1)c_{n+2}-n(n-1)c_n-2nc_n+\lambda c_n=0\implies c_{n+2}=\frac{n(n+1)-\lambda}{(n+1)(n+2)}c_n$$
The iteration steps by $2$, so we get two linearly independent solutions
\begin{align*}
    y_{\text{even}}&=c_0\left( 1+\frac{-\lambda}{2!}x^2+\frac{(6-\lambda)(-\lambda)}{4!}x^4+\cdots \right)\\
    y_{\text{odd}}&=c_1\left( x+\frac{2-\lambda}{3!}x^3+\cdots \right)
\end{align*}
Note that $c_{n+2}/c_n\to 1$, so the both series has radius of convergence $1$ but they diverges at $x=\pm 1$.
However, this is not the end of the world!
These series may not be infinite.
If $\lambda=l(l+1)$ for some $l\in\mathbb N$, then one of these two series will terminate and give a polynomial solution.
These polynomials are called Legendre polynomials $P_l(x)$ which are eigenfunctions of the Legendre equation.
Conventionally we normalise $P_l$ by requiring $P_l(1)=1$.
One can check that this restricts $P_l([-1,1])\subset [-1,1]$ and $|P_l(-1)|=1$.
By calculation we have
$$P_0(x)=1,P_1(x)=x,P_2(x)=\frac{3x^2-1}{2},P_3(x)=\frac{5x^3-3x}{2},\ldots$$
We easily observe that $P_l$ has $l$ roots in $[-1,1]$, also $P_l$ is odd if $l$ is odd, and even when $l$ is even.
By orthogonality and some calculation,
$$\forall n\neq m,\int_{-1}^1P_nP_m\,\mathrm dx=0,\int_{-1}^1P_n^2\,\mathrm dx=\frac{2}{2n+1}$$
There are several other ways to characterise Legendre polynomails.
One can prove that we have
$$P_n(x)=\frac{1}{2^nn!}\frac{\mathrm d^n}{\mathrm dx^n}(x^2-1)^n,\sum_{n=0}^\infty P_n(x)t^n=\frac{1}{\sqrt{1-2xt+t^2}}$$
We also have the recursions
$$l(l+1)P_{l+1}(x)=(2l+1)xP_l(x)-lP_{l-1}(x),(2l+1)P_l(x)=\frac{\mathrm d}{\mathrm dx}(P_{l+1}(x)-P_{l-1}(x))$$
If we take these $P_l$ as a set of eigenfunctions, then any well-behaved $f$ on $[-1,1]$ can be expressed as
$$f(x)=\sum_{l=0}^\infty a_lP_l(x),a_l=\frac{2l+1}{2}\int_{-1}^1f(x)P_l(x)\,\mathrm dx$$
\begin{example}
    We can verify that $f(x)=(15x^2-3)/2=P_0(x)+5P_2(x)$.
\end{example}
\begin{example}
    The odd equare wave with $f([0,1))=\{1\}$ has the expansion
    $$\sum_{m=1}^\infty (P_{2m}(0)-P_{2m+2}(0))P_{2m+1}(x)$$
\end{example}
\subsection{Inhomogeneous ODEs}
Consider the ODE $\mathcal Ly=f=wF$ (with homogeneous boundary conditions so that $\mathcal L$ is self-adjoint) where $w\ge 0$ is our wavefunction.
Given eigenfunctions $\{y_n\}$ satisfying $\mathcal Ly_n=\lambda_nwy_n$ for eigenvalues $\{\lambda_n\}$.
We now try to find a solution in the form $y=\sum_nc_ny_n$.
To do this, we expand $F=\sum_na_ny_n$ where
$$a_n=\left( \int_a^bwFy_n\,\mathrm dx\middle) \right/\left( \int_a^bwy_n^2\,\mathrm dx \right)$$
then
$$w\sum_na_ny_n=wF=\mathcal Ly=\mathcal L\sum_nc_ny_n=\sum_nc_n\mathcal Ly=w\sum_nc_n\lambda_ny_n$$
So take
$$y=\sum_n\frac{a_n}{\lambda_n}y_n$$
gives a particular solution, assuming everything is well-defined and converges nicely enough.\\
An aside:
The driving force $F$ sometimes induces a linear response term $\tilde\lambda wy$, so the solution is $\mathcal Ly-\tilde{\lambda}wy=f$.
Then our particular solution can be
$$y=\sum_{\lambda_n\neq\tilde{\lambda}}\frac{a_n}{\lambda_n-\tilde\lambda}y_n$$
Now back to theme.
If we expand the expression of $a_n$, we can get
\begin{align*}
    y(x)&=\sum_{n=1}^\infty\frac{a_n}{\lambda_n}y_n(x)\\
    &=\sum_{n=1}^\infty\frac{y_n(x)}{\lambda_nN_n}\int_a^bw(\xi)F(\xi)y_n(\xi)\,\mathrm d\xi,N_n=\int_a^bwy_n^2\,\mathrm dx\\
    &=\int_a^b\left( \sum_{n=1}^\infty\frac{y_n(x)y_n(\xi)}{\lambda_nN_n} \right)w(\xi)F(\xi)\,\mathrm d\xi\\
    &=\int_a^bG(x,\xi)f(\xi)\,\mathrm d\xi,G(x,\xi)=\sum_{n=1}^\infty\frac{y_n(x)y_n(\xi)}{\lambda_nN_n}
\end{align*}
This $G(x,\xi)$ is called the Green's function of this particular eigenvalue problem of that self-adjint operator.
Worth noting that $G$ does not depend on the forcing term $f$.
The Green's function also induces
$$\mathcal L^-(\phi)=\int_a^bG(x,\xi)\phi(\xi)\,\mathrm d\xi$$
which can be taken as kind of an inverse operator to $\mathcal L$ since $\mathcal L(\mathcal L^-(f))=f$.