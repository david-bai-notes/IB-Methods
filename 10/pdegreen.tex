\section{Solving PDEs with Green's Functions}
\subsection{Diffusion Equation and Fourier Transform}
Recall that the heat equation for a conducting wire is
$$\frac{\partial\Theta}{\partial t}-D\frac{\partial^2\Theta}{\partial x^2}$$
with initial condition $\Theta(x,0)=h(x)$ and boundary condition $\Theta\to 0$ as $x\to\pm\infty$.
Taking Fourier transform wrt $x$ gives
$$\frac{\partial}{\partial t}\tilde{\Theta}(k,t)=-Dk^2\tilde\Theta(k,t)\implies\tilde\Theta(k,t)=Ce^{-Dk^2t}$$
for some $C$ that is constant in $t$.
Initial condition $\tilde{\Theta}(k,0)=\tilde{h}(k)$ gives $\tilde\Theta(k,t)=\tilde{h}(k)e^{-Dk^2t}$.
Inverting it gives
\begin{align*}
    \Theta(x,t)&=\frac{1}{2\pi}\int_{-\infty}^\infty \tilde{h}(k)e^{-Dk^2t}e^{ikx}\,\mathrm dk\\
    &=\frac{1}{\sqrt{4\pi Dt}}\int_{-\infty}^\infty h(u)\exp\left( -\frac{(x-u)^2}{4Dt} \right)\,\mathrm du\\
    &=\int_{-\infty}^\infty h(u)S_d(x-u,t)\,\mathrm du
\end{align*}
where $S_d(x,t)=(4\pi Dt)^{-1/2}e^{-x^2/(4Dt)}$ is called the fundamental solution (or diffusion kernel/sourse function), which is just the FT of $e^{-Dk^2t}$.
Note that the case $\Theta(x,0)=\theta_0\delta(x)$ gives our old friend $\Theta=\theta_0(4\pi Dt)^{-1/2}e^{-\eta^2}$ where $\eta=x/(2\sqrt{Dt})$ is the similarity parameter.
\begin{example}
    Suppose initially $f(x)=\theta_0\sqrt{a/\pi}e^{-ax^2}$.
    Then
    \begin{align*}
        \Theta&=\frac{\theta_0\sqrt{a}}{\sqrt{4\pi^2Dt}}\int_{-\infty}^\infty\exp\left( -au^2-\frac{(x-u)^2}{4Dt} \right)\,\mathrm du\\
        &=\frac{\theta_0\sqrt{a}}{\sqrt{4\pi^2Dt}}\int_{-\infty}^\infty\exp\left( -\frac{1+4aDt}{4Dt}\left( u-\frac{x}{1+4aDt} \right)^2 \right)\exp\left( \frac{-ax^2}{1+4aDt} \right)\,\mathrm du\\
        &=\theta_0\sqrt{\frac{a}{\pi(1+4aDt)}}\exp\left( \frac{-ax^2}{1+4aDt} \right)
    \end{align*}
    Asymptotically the width of the Gaussian spreads as $\sqrt{t}$ and the area under the curve being constant (which can be interpreted as the conservation of heat energy).
\end{example}
\subsection{Forced Diffusion Equation}
Consider
$$\frac{\partial\Theta}{\partial t}(x,t)-D\frac{\partial^2\Theta}{\partial x^2}(x,t)=f(x,t)$$
with homogeneous initial condition $\Theta(x,0)=0$.
The Green's function $G$ of this problem would satisfy
$$\frac{\partial G}{\partial t}-D\frac{\partial^2G}{\partial x^2}=\delta(x-\xi)\delta(t-\tau)$$
with $G(x,0;\xi,\tau)=0$.
Take FT wrt $x$ gives
$$\frac{\partial \tilde{G}}{\partial t}(k,t;\xi,\tau)+Dk^2\tilde{G}(k,t;\xi,\tau)=e^{-ik\xi}\delta(t-\xi)$$
Multiplying both sides using the integration factor $e^{Dk^2t}$ allows us to integrate the equation and obtain
$$e^{Dk^2t}\tilde{G}=e^{-ik\xi}\int_0^te^{Dk^2t'}\delta(t'-\tau)\,\mathrm dt'$$
The appearance of Heaviside function is because we need to ensure that $[0,t]$ contains $\tau$ in order to make use of the property of $\delta$ function.
Inverting the whole thing gives
\begin{align*}
    G(x,t;\xi,\tau)&=\frac{H(t-\tau)}{2\pi}\int_{-\infty}^\infty e^{ik(x-\xi)}e^{-Dk^2(t-\tau)}\,\mathrm dk\\
    &=\frac{H(t')}{2\pi}\int_{-\infty}^\infty e^{ikx'}e^{-Dk^2t'}\,\mathrm dk,x'=x-\xi,t'=t-\tau\\
    &=\frac{H(t')}{\sqrt{4\pi Dt'}}e^{-x'^2/(4Dt')}\\
    &=H(t-\tau)S_d(x-\xi,t-\tau)
\end{align*}
where $S_d$ is again our good ol' fundamental solution.
Therefore the general solution is
\begin{align*}
    \Theta(x,t)&=\int_0^\infty\int_{-\infty}^\infty G(x,t;\xi,\tau)f(\xi,\tau)\,\mathrm d\xi\mathrm d\tau\\
    &=\int_0^t\int_{-\infty}^\infty f(u,\tau)S_d(x-u,t-\tau)\,\mathrm du\mathrm d\tau
\end{align*}
which looks very familiar to our previous solution for the homogeneous case with initial conditions at $t=\tau$.
But this time, the initial condition of $\Theta(u,t)=f(u)$ at $t=\tau$ is replaced by the forcing term $f(u,\tau)$ in the equation, and the effect of this, as we see now, is just integrating the solution over $\tau$.
This phenomenon is called Duhamel's Principle which relates the solution of forced PDE with homogeneous boundary conditions to solutions of homogeneous PDEs with inhomogenoeus boundary conditions.
So in this philosophy, the forcing term is acting like an initial conditions for subsequent evolution, and the integral represents a superposition of all these initial-condition-like effects for $0<\tau<t$.
\subsection{Forced Wave Equation}
Consider the forced wave equation
$$\frac{\partial^2\phi}{\partial t^2}-c^2\frac{\partial^2\phi}{\partial x}=f(x,t)$$
subject to $\phi(x,0)=\phi_t(x,0)=0$.
The Green's function for this would satisfy
$$\frac{\partial^2\phi}{\partial t^2}-c^2\frac{\partial^2\phi}{\partial x}=\delta(x-\xi)\delta(t-\tau)$$
with $G=G_t=0$ at $t=0$.
Take FT wrt $x$ again,
$$\frac{\partial^2\tilde{G}}{\partial t^2}+c^2k^2\tilde{G}=e^{-ik\xi}\delta(t-\tau)$$
Solving it yields
$$\tilde{G}=\begin{cases}
    0\text{, if $t<\tau$}\\
    e^{-ik\xi}\sin(kc(t-\tau))/(kc)\text{, if $t>\tau$}
\end{cases}=e^{-ik\xi}\frac{\sin(kc(t-\tau))}{kc}$$
Inverting this gives
\begin{align*}
    G(x,t;\xi,\tau)&=\frac{H(t-\tau)}{2\pi c}\int_{-\infty}^\infty e^{ik(x-\xi)}\frac{\sin(kc(t-\tau))}{k}\,\mathrm dk\\
    &=\frac{H(t-\tau)}{2\pi c}\int_{-\infty}^\infty \frac{\cos(kA)\sin(kB)}{k}\,\mathrm dk,A=x-\xi,B=c(t-\tau)\\
    &=\frac{H(t-\tau)}{2\pi c}\int_{-\infty}^\infty \frac{\sin(k(A+B))-\sin(k(A-B))}{k}\,\mathrm dk\\
    &=\frac{H(t-\tau)}{2\pi c}(\operatorname{sgn}(A+B)-\operatorname{sgn}(A-B))\\
    &=\frac{1}{2c}H(c(t-\tau)-|x-\xi|)
\end{align*}
which is also called the causal fundamental solution since it hints the causal structure in $t$.
The general solution is then
\begin{align*}
    \phi(x,t)&=\int_0^\infty\int_{-\infty}^\infty f(\xi,t)G(x,t;\xi,\tau)\,\mathrm d\xi\mathrm d\tau\\
    &=\frac{1}{2c}\int_0^t\int_{x-c(t-\tau)}^{x+c(t-\tau)}f(\xi,\tau)\,\mathrm d\xi\mathrm d\tau
\end{align*}
which, by relating to our previous solution, can also be seen as an example of Dudamel's principle.
\subsection{Poisson's Equation}
We want to solve Poisson's equation, which is just a forced Laplace equation $\nabla^2\phi=-\rho$ on a domain $D$ subject to Dirichlet boundary conditions $\phi|_{\partial D}=0$.
To get its fundamental solution, note that we can get the notion of $\delta$ function on $\mathbb R^3$ analogously.
So the free space Green's function $G=G_{\rm FS}(\underline{r};\underline{r}')$ is naturally the solution to $\nabla^2G_{\rm FS}(\underline{r};\underline{r}')=\delta(\underline{r}-\underline{r}')$.
We can take this to be spherically symmetric $G(\underline{r};\underline{r}')=G(|\underline{r}-\underline{r}'|)=G(r)$.
Let $B$ be the ball with centre $\underline{r}'$ and radius $r$, then we have
$$1=\int_B\delta(\underline{r}-\underline{r}')\,\mathrm d^3\underline{r}=\int_B\nabla^2G_{\rm FS}\,\mathrm d^3r=\int_{\partial B}\nabla G_{\rm FS}\cdot\underline{n}\,\mathrm dS=4\pi r^2\frac{\partial G_{\rm FS}}{\partial r}$$
So $G_{\rm FS}=-1/(4\pi r)=-1/(4\pi|\underline{r}-\underline{r}'|)$ because of the boundary condition $G\to 0$ as $r\to\infty$.
Therefore we get the general solution
$$\Phi(\underline{r})=\frac{1}{4\pi}\int_D\frac{\rho(\underline{r}')}{|\underline{r}-\underline{r}'|}\,\mathrm d^3\underline{r}'$$
By the way, we can derive the Green's function in 2D in the similar way and get $G_2(\underline{r};\underline{r}')=-(2\pi)^{-1}\log(|\underline{r}-\underline{r}'|)+C_2$ where $C_2$ is a constant (which we almost always set to $0$).\\
We now turn to Green's Identities.
Consider two scalar functions $\phi,\psi$ twice differentiable on $D$.
Then
$$\int_D(\phi\nabla^2\psi+\nabla\phi\cdot\nabla\psi)\,\mathrm d^3\underline{r}=\int_D\nabla\cdot(\phi\nabla\psi)\,\mathrm d^3\underline{r}=\int_{\partial D}\phi\nabla\psi\cdot\underline{\hat{n}}\,\mathrm dS$$
by divergence theorem.
This is called Green's first identity.
Now switch $\psi$ and $\phi$ and substract from the first identity to get Green's second identity.
$$\int_{\partial D}\left( \phi\frac{\partial\psi}{\partial n}-\psi\frac{\partial\phi}{\partial n} \right)\,\mathrm dS=\int_D(\phi\nabla^2\psi-\psi\nabla^2\phi)\,\mathrm d^3\underline{r}$$
Now excise a small spherical ball $B_\epsilon$ around $\underline{r}'$ with radius $\epsilon$.
Now if $\phi$ is a solution to $\nabla^2\phi=-\rho$ and $\psi=G_{\rm FS}(\underline{r};\underline{r}')$, then the right hand side is
$$\int_{D-B_\epsilon}(\phi\nabla^2G_{\rm FS}-G_{\rm FS}\nabla^2\phi)\,\mathrm d^3r=\int_{D-B_\epsilon}G_{\rm FS}\rho\,\mathrm d^3\underline{r}$$
and the left hand side becomes
$$\int_{\partial D}\left( \phi\frac{\partial G_{\rm FS}}{\partial n}-G_{\rm FS}\frac{\partial\phi}{\partial n} \right)\,\mathrm dS+\int_{\partial B_\epsilon}\left( \phi\frac{\partial G_{\rm FS}}{\partial n}-G_{\rm FS}\frac{\partial\phi}{\partial n} \right)\,\mathrm dS$$
Taking $\epsilon\to 0$ gives
$$\int_{\partial B_\epsilon}\left( \phi\frac{\partial G_{\rm FS}}{\partial n}-G_{\rm FS}\frac{\partial\phi}{\partial n} \right)\,\mathrm dS\to-\phi(\underline{r}')$$
Therefore we conclude Green's third identity
$$\phi(\underline{r}')=\int_DG_{\rm FS}(\underline{r};\underline{r}')(-\rho(\underline{r}))\,\mathrm d^3\underline{r}+\int_{\partial D}\left( \phi(\underline{r})\frac{\partial G}{\partial n}(\underline{r};\underline{r}')-G_{\rm FS}(\underline{r};\underline{r}')\frac{\partial\phi}{\partial n}(\underline{r}) \right)\,\mathrm dS$$
Now we want to solve $\nabla^2\phi=-\rho$ on $D$ subject to inhomogeneous Dirichlet boundary conditions $\phi|_{\partial D}=h$.
In this case, we want the Green's function $G=G(\underline{r};\underline{r}')$ to satisfy:\\
(i) $\nabla^2G(\underline{r};\underline{r}')=0$ for any $\underline{r}\neq\underline{r}'$.\\
(ii) $G(\underline{r};\underline{r}')=0$ on $\partial D$.\\
(iii) $G(\underline{r},\underline{r}')=G_{\rm FS}(\underline{r};\underline{r}')+H(\underline{r},\underline{r}')$ for some $H$ such that $\nabla^2H=0$ in $D$.\\
Green's second identity with $\nabla^2\phi=-\rho$ and $\nabla^2H=0$ gives
$$\int_{\partial D}\left( \phi\frac{\partial H}{\partial n}-H\frac{\partial\phi}{\partial n} \right)\,\mathrm dS=\int_DH\rho\,\mathrm d^3\underline{r}$$
The Green's third identity simplifies to
\begin{align*}
    \phi(\underline{r}')&=\int_D(G-H)(-\rho)\,\mathrm d^3\underline{r}+\int_{\partial D}\left( \phi\frac{\partial (G-H)}{\partial n}-(G-H)\frac{\partial\phi}{\partial n} \right)\,\mathrm dS\\
    &=\int_DG(\underline{r};\underline{r}')(-\rho(\underline{r}))\,\mathrm d^3\underline{r}+\int_{\partial D}h(\underline{r})\frac{\partial G(\underline{r};\underline{r}')}{\partial n}\,\mathrm dS
\end{align*}
So this Green's function does give a way to obtain a particular solution.
Also $G(\underline{r};\underline{r}')=G(\underline{r};\underline{r}')$ by a simple use of Green's third identity again.\\
For Neumann boundary conditions with $\partial\phi/\partial n=k(\underline{r})$ on $\partial D$, we can do a similar thing and obtain
$$\phi(\underline{r}')=\int_DG(\underline{r};\underline{r}')(-\rho(\underline{r}))\,\mathrm d^3\underline{r}+\int_{\partial D}G(\underline{r};\underline{r}')(-k(\underline{r}))\,\mathrm dS$$
\subsection{Method of Images}
For symmetric domain $D$ we can construct Green's function with $G=0$ on $\partial D$ by cancelling the boundary potential with an opposite mirror image Green's function placed outside $D$.\\
Our first target is Laplace's equation on the half-plane $D=\{(x,y,z):z>0\}$ subject to $\phi(x,y,0)=h(x,y)$ and $\phi\to 0$ as $|\underline{r}|\to\infty$.
Now $G_{\rm FS}(\underline{r};\underline{r}')\to 0$ as $|\underline{r}|\to\infty$, but it is nonzero at $z=0$.
However, we can take $G(\underline{r};\underline{r}')=G(\underline{r};\underline{r}')-G(\underline{r},\underline{r}'')$ where $\underline{r}''=(x',y',-z')$ which totally works.
Also at $z=0$, $\partial G/\partial n=\partial G/\partial z=(2\pi)^{-1}z'((x-x')^2+(y-y')^2+z'^2)^{-3/2}$.
This means that the solution is
$$\Phi(x',y',z')=\frac{z'}{2\pi}\int_{-\infty}^\infty\int_{-\infty}^\infty((x-x')^2+(y-y')^2+z'^2)^{-3/2}h(x,t)\,\mathrm dx\mathrm dy$$
We now turn to the wave equation for $x>0$.
Consider $\ddot{\phi}-c^2\phi^{\prime\prime}=f$ with Dirichlet boundary conditions $\phi(0,t)=0$.
The same philosophy gives the Green's function
$$G(x,t;\xi,\tau)=\frac{1}{2c}H(c(t-\tau)-|x-\xi|)-\frac{1}{2c}H(c(t-\tau)-|x+\xi|)$$
So if $f=0$ and the initial condition is a Gaussian pause, then this solves to
$$\phi(x,t)=\exp((x-\xi+ct)^2)-\exp((-x-\xi+ct)^2)$$