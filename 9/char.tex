\section{Characteristics}
\subsection{Well-Posed Cauchy Problems}
Solving PDEs depends on the equation and the boundary/initial data.
A Cauchy problem is a PDF together with auxiliary data specified on a surface in 3D or a curve in 2D (known as the Cauchy data).
\begin{definition}
    A Cauchy problem is well-posed if:\\
    1. A solution exists.\\
    2. The solution is unique.\\
    3. The solution depends continuously on the auxiliary data.
\end{definition}
\subsection{Method of Characteristics}
Suppose we have a curve $C$ parameterised by $(x(s),y(s))$ in space and a tangent $v=(x^\prime(s),y^\prime(s))$ to the curve at some point $P$.
For a function $\phi(x,y)$, we can define a directional derivative
$$\left.\frac{\mathrm d\phi}{\mathrm ds}\right|_C=x^\prime(s)\frac{\partial\phi}{\partial x}+y^\prime(s)\frac{\partial\phi}{\partial y}=v\cdot \nabla\phi|_C$$
If $v\cdot\nabla\phi=0$, then $\mathrm d\phi/\mathrm ds=0$ and $\phi$ is constant along $C$.
Now suppose we have a vector field $u=(\alpha(x,y),\beta(x,y))$ with its family of integral curves (i.e. curves which tangent to the field at any point) non-intersecting and filling $\mathbb R^2$.
Now find a curve $B$ by $(x(t),y(t))$ tranverse to $u$ such that its tangent vector $w=(x^\prime(t),y^\prime(t))$ is never parallel to $u$.
Label each integral curve $C$ of $u$ using $t$ at the intersection point with $B$ and then use $s$ to parameterise along the curve (i.e. take $s=0$ at $B$).
Our integral curves then satisfy $x^\prime(s)=\alpha(x,y),y^\prime(s)=\beta(x,y)$, solving which gives a family of characteristic curves along whcih $t$ remains constant.
In some sense, we are creating a new coordinate $(s,t)$ at which the PDE is in a nice form.
\subsection{Characteristics of a First-Order PDE}
Consider the first-order PDE
$$\alpha(x,y)\frac{\partial\phi}{\partial x}+\beta(x,y)\frac{\partial\phi}{\partial y}=0$$
with specified Cauchy data on an initial curve $B$ parameterised by $(x(t),y(t))$.
Note that we immediately have $\alpha\phi_x+\beta\phi_y=u\cdot\nabla\phi=\phi^\prime(s)|_C$ where $C$ is an integral curve of $u=(\alpha,\beta)$.
These are called characteristic curves of the PDE.
The PDE then gives $\phi^\prime(s)=\alpha\phi_x+\beta\phi_y=0$, therefore $\phi$ is constant along the curve $C$.
Therefore the Cauchy data $f(t)$ defined on (sufficiently nice) $B$ at $s=0$ will be propagated constantly along $C$ to give the solution $\phi(s,t)=\phi(x(s,t),y(s,t))=f(t)$.
To get $\phi$, simply invert $s=s(x,y),t=t(x,y)$ (provided it has nonzero Jacobian) and we have $\phi(x,y)=f(t(x,y))$.
\begin{example}
    1. Consider the simple ODE $\partial\phi/\partial x=0$ with $\phi(0,y)=h(y)$ given on the $y$-axis.
    The family of curves we want are the integral curves $x^\prime(s)=\alpha=1,y^\prime(s)=\beta=0$.
    The $y$ axis is parameterised as $(x(t),y(t))=(0,t)$.
    Therefore at $s=0$ we need $(x,y)=(0,t)$, hence the family of curves $C$ are characterised as $x=s,y=t$ (simple indeed).
    Therefore $\phi(s,t)=h(t)$, and by inversion $\phi(x,y)=h(t)$.\\
    2. Let's do some example that are a bit less simple.
    We turn our attention to $e^x\phi_x+\phi_y=0$ with $\phi(x,0)=\cosh x$.
    The characteristic equation is $x^\prime(s)=e^x,y^\prime(s)=1$.
    The initial curve is parameterised as $x(t)=t,y(t)=0$ which shall apply when $s=0$.
    Solving these gives $e^{-x}=e^{-t}-s,y=s$.
    $\phi^\prime(s)=0$ gives $\phi(s,t)=\cosh t$.
    The inversion gives $s=y,t=-\log(y+e^{-x})$, so $\phi(x,y)=\cosh(-\log(y+e^{-x}))$.
\end{example}
So the homogeneous case is easy enough.
How about inhomogeneous ones?
Of course we can do the good old "guess and superposition" manoeuvre, but we can do better.\\
We want to solve $\alpha(x,y)\phi_x+\beta(x,y)\phi_y=\gamma(x,y)$ with specified Cauchy data $\phi(x(t),y(t))=f(t)$ on a curve $B$.
The characteristic curves $C$ satisfy the same system as usual, but there is a twist:
$\phi^\prime(s)|_C=u\cdot\nabla\phi=\gamma(x,y)$ instead of $0$.
So $f(t)$ is no longer propagating constantly and we must actually solve the ODE.
\begin{example}
    Consider $\phi_x+2\phi_y=ye^x$ with $\phi=\sin x$ along $y=x$.
    The characteristic equations are $x^\prime(s)=1,y^\prime(s)=2$.
    On $y=x$, we parameterise $(x(t),y(t))=(t,t)$, which then gives $x=s+t,y=2s+t$.
    Now we turn to $\phi^\prime(s)=\gamma=ye^x=(2s+t)e^{s+t}$ subject to $\phi=\sin t$ at $s=0$.
    By simple integration we obtain $\phi=(2s-2+t)e^{s+t}+C$ where $C$ is constant in $s$.
    The initial conditions then gives $\phi(s,t)=(2s-2+t)e^{s+t}+\sin t+(2-t)e^t$.
    With inversion $s=y-x,t=2x-y$ gives
    $$\phi(x,y)=(y-2)e^x+(y-2x+2)e^{2x-y}+\sin(2x-y)$$
\end{example}
\subsection{Classification of Second-Order Linear PDEs}
In $\mathbb R^2$, the general homogeneous second order linear PDE has the form
$$0=\mathcal L\phi=a\frac{\partial^2\phi}{\partial x^2}+2b\frac{\partial^2\phi}{\partial x\partial y}+c\frac{\partial^2\phi}{\partial y^2}+d\frac{\partial\phi}{\partial x}+e\frac{\partial\phi}{\partial y}+f\phi$$
The principal part of this ODE is defined as
$$\sigma_p(x,y,k_x,k_y)=k^\top Ak=\begin{pmatrix}
    k_x&k_y
\end{pmatrix}\begin{pmatrix}
    a(x,y)&b(x,y)\\
    b(x,y)&c(x,y)
\end{pmatrix}\begin{pmatrix}
    k_x\\
    k_y
\end{pmatrix}$$
The PDEs are classified by the properties of the eigenvalues of $A$.
If $b^2-ac<0$, then the eigenvalues have the same sign, so the equation is elliptic.
If $b^2-ac>0$, then the eignevalues have different signs and the equation is called hyperbolic.
If $b^2-ac=0$, then some eigenvalue is $0$, and the equation is called parabolic.
\begin{example}
    The wave equation is hyperbolic.
    The heat equation is parabolic.
    The Laplace equation is elliptic.
\end{example}
A curve defined by $f(x,y)$ being constant is a characteristic curve for this second order PDE if $(\nabla f)A(\nabla f)^\top =0$.
If the curve can be written as $y=y(x)$, then $f_x/f_y=\mathrm dy/\mathrm dx$, so a substitution gives $a(y^\prime)^2-2by^\prime+c=0$ and hence $y^\prime(x)=(b\pm\sqrt{b^2-ac})/a$.
Therefore this classification makes sense as the sign of $b^2-ac$ determines the behaviour of characteristics.
If the equation is hyperbolic, then $b^2-ac>0$ which gives two distinct solutions;
if it is parabolic, then we get exactly one solution;
but if it is elliptic, then there is simply no (real) characteristic curves in this form.\\
If we can transform these to characteristic coordinates $(u,v)$, the PDE will take the canonical form
$$0=\frac{\partial^2\phi}{\partial u\partial v}+\text{lower order terms}$$
\begin{example}
    Consider $-y\phi_{xx}+\phi_{yy}=0$ which is hyperbolic for $y>0$, elliptic for $y<0$ and parabolic for $y=0$.
    We are interested in the hyperbolic $y>0$ case.
    Then $\mathrm dy/\mathrm dx=\pm y^{-1/2}$ by quadratic formula.
    Integrating gives $(2/3)y^{3/2}\pm x=C_\pm$, $C_\pm$ constants.
    The characteristic coordinates are then set as $u=(2/3)y^{3/2}+ x,v=(2/3)y^{3/2}- x$.
    After a lot of calculations using chain rule, the equation can be transformed into
    $$\phi_{uv}+\frac{1}{6(u+v)}(\phi_u+\phi_v)=0$$
\end{example}
\subsection{General Solution for Wave Equation}
Going back to our old friend
$$\frac{1}{c^2}\frac{\partial^2\phi}{\partial t^2}-\frac{\partial^2\phi}{\partial x^2}=0$$
with initial conditions $\phi(x,0)=f(x),\phi_t(x,0)=g(x)$.
Then the characteristic equation is $\mathrm dx/\mathrm dt=\pm C$, $C$ constant.
Therefore we can do the change of coordinate $u=x-ct,v=x+ct$ which just gives
$$\frac{\partial^2\phi}{\partial u\partial v}=0\implies \phi=G(u)+H(v)=G(x-ct)+H(x+ct)$$
For differentiable $G,H$.
The initial conditions then give the eqautions
$$\begin{cases}
    f(x)=\phi(x,0)=G(x)+H(x)\\
    g(x)=\phi_t(x,0)=-cG^\prime(x)+cH^\prime(x)
\end{cases}$$
Combining them gives
$$H(x)=\frac{1}{2}(f(x)-f(0))+\frac{1}{2c}\int_0^xg(y)\,\mathrm dy,G(x)=\frac{1}{2}(f(x)+f(0))-\frac{1}{2c}\int_0^xg(y)\,\mathrm dy$$
So
$$\phi(x,t)=\frac{f(x-ct)+f(x+ct)}{2}+\frac{1}{2c}\int_{x-ct}^{x+ct}g(y)\,\mathrm dy$$
This means that the wave propagates at $v=c$ and $\phi$ is fully determined by the values of $f,g$ (which is the initial data at $t=0$) in the interval $[x-ct,x+ct]$.
This can be interpreted as the light cone, which gives the causal structure of relativity.
That is, data at $x=x_0$ only influence $[x_0-ct,x_0+ct]$ after time $t$.