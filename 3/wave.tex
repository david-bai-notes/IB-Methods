\section{The Wave Equation}
\subsection{Waves on an Elastic String}
Consider small displacements $y(x,t)$ on a stretched string with fixed ends at $x=0$ and $x=L$, that is with boundary conditions $y(0,t)=y(L,t)=0$.
We want to determine the string's motion subject to initial conditions
$$y(x,0)=p(x),\frac{\partial y}{\partial t}(x,0)=q(x)$$
We want to derive its equation of motion.
We try to obtain a differential equation by balancing the forces on string segment $x,x+\delta x$ and taking $\delta x\to 0$.
By resolving in $x$ direction we get that the tension $T$ on the string is independent of $x$.
By resolving in $y$ direction we arrive at
$$(\mu\delta x)\frac{\partial^2y}{\partial t^2}=T\frac{\partial^2y}{\partial x^2}\delta x-g\mu\delta x$$
where $\mu$ is the mass per unit length (aka linear mass density).
Write $c=\sqrt{T/\mu}$ the wave speed and assume the acceleration due to gravity is negligible, then this equation is just
$$\frac{1}{c^2}\frac{\partial^2y}{\partial t^2}=\frac{\partial^2y}{\partial x^2}$$
which is called the one-dimensional wave equation.
\subsection{Seperation of Variables}
Our first attempt at a solution is to guess a solution of seperable form, that is $y(x,t)=X(x)T(t)$, then substitution gives
$$\frac{1}{c^2}X\ddot{T}=X^{\prime\prime}T\implies \frac{1}{c^2}\frac{\ddot{T}}{T}=\frac{X^{\prime\prime}}{X}$$
The left hand side depends only on $t$ and the right hand side depends only on $x$, so they can only equal if both sides equal a constant $-\lambda$ (called the seperation constant), then we get
$$\begin{cases}
    X^{\prime\prime}+\lambda X=0\\
    \ddot{T}+\lambda c^2T=0
\end{cases}$$
Such nice things don't always happen, but we are glad it happened in this particular case.
\footnote{Whether we can do it depends on the existence of symmetry in the boundary conditions, which is probably not going to be discussed here.}
So we reduced a PDE to two independent ODEs, which we know how to solve.
\subsection{Boundary Conditions and Normal Modes}
There are a few possibilities depending on the sign of $\lambda$.\\
If $\lambda<0$, then taking $\chi^2=-\lambda$ gives the general solution
$$X(x)=Ae^{\chi x}+Be^{-\chi x}=\tilde{A}\cosh(\chi x)+\tilde{B}\sinh(\chi x)$$
where $A,B,\tilde{A},\tilde{B}$ are constants.
But the boundary conditions $X(0)=X(L)=0$ would yield $X=0$ everywhere, so this is just the trivial solution.\\
Now if $\lambda=0$, then $X(x)=Ax+B$ where $A,B$ are constants, but again we must have $A=B=0$ as $X(0)=X(L)=0$.
Trivial solution again.\\
What is left is $\lambda>0$, so $X(x)=A\cos(\sqrt{\lambda}x)+B\sin(\sqrt{\lambda}x)$, then the boundary condition $X(0)=X(L)=0$ gives the family of solutions
$$\lambda_n=\left( \frac{n\pi}{L} \right)^2,X_n(x)=B_n\sin\frac{n\pi x}{L}$$
where $B_n$ are constants.
It is not hard to observe that they are just our familiar Fourier eigenvalues and eigenfunctions.
These are called the normal modes of the system since its spacial shape in $x$ does not change in time but the amplitudes may vary.
\footnote{Well, duh!}
The case $n=1$ is called the fundamental mode.
A plot shows the modes are simply just the patterns of simple vibrations we expect.
\subsection{Initial Conditions and Temporal Solutions}
Substituting $\lambda_n=(n\pi/L)^2$ into the time ODE gives
$$T_n(t)=C_n\cos\frac{n\pi ct}{L}+D_n\sin\frac{n\pi ct}{L}$$
where $C_n,D_n$ are constants.
So we obtain the family of solutions
$$y_n(x,t)=T_n(t)X_n(x)=\left( C_n\cos\frac{n\pi ct}{L}+D_n\sin\frac{n\pi ct}{L} \right)\sin\frac{n\pi x}{L}$$
As the system we are trying to deal with is homogeneous and linear, we have the superposition principle, so the general solution is
$$y(x,t)=\sum_{n=1}^\infty\left( C_n\cos\frac{n\pi ct}{L}+D_n\sin\frac{n\pi ct}{L} \right)\sin\frac{n\pi x}{L}$$
assuming it converges sufficiently well.
This satisfies the boundary conditions as $X_n$ does.
Substituting into the initial condition, we get
$$p(x)=\sum_{n=1}^\infty C_n\sin\frac{n\pi x}{L},q(x)=\sum_{n=1}^\infty\frac{n\pi c}{L}D_n\sin\frac{n\pi x}{L}$$
which allows us to find $C_n,D_n$ by expanding $p,q$ as Fourier sine series.
In particular,
$$C_n=\frac{2}{L}\int_0^Lp(x)\sin\frac{n\pi x}{L}\,\mathrm dx,D_n=\frac{2}{n\pi c}\int_0^Lq(x)\sin\frac{n\pi x}{L}\,\mathrm dx$$
This is possible (well, assuming everything), so we have found a particular solution to the system.
\begin{example}
    We pluck string at $x=\xi$, which requires
    $$y(x,0)=p(x)=\begin{cases}
        x(1-\xi)\text{, for $0\le x\le \xi$}\\
        \xi(1-x)\text{, for $\xi\le x\le 1$}
    \end{cases},\frac{\partial y}{\partial t}(x,0)=q(x)=0$$
    Then with our formulas we obtain
    $$C_n=\frac{2\sin(n\pi\xi)}{(n\pi)^2},D_n=0\implies y(x,t)=\sum_{n=1}^\infty\frac{2}{(n\pi)^2}\sin(n\pi\xi)\sin(n\pi x)\cos(n\pi ct)$$
    Of course, this case happens to be the way of making sound on a string instrument, where guitar has $\xi\in [1/4,1/3]$ and violin has $\xi\approx 1/7$.
\end{example}
By the usual trigonometric identities, the general solution we found earlier becomes $y(x,t)=f(x-ct)+g(x+ct)$ where
$$f(x-ct)=\frac{1}{2}\sum_{n=1}^\infty\left(C_n\sin\frac{n\pi(x-ct)}{L}+D_n\cos\frac{n\pi(x-ct)}{L}\right)$$
and
$$g(x+ct)=\frac{1}{2}\sum_{n=1}^\infty\left(C_n\sin\frac{n\pi(x+ct)}{L}-D_n\cos\frac{n\pi(x+ct)}{L}\right)$$
So the standing wave solution can be interpreted as a superposition of a right-moving wave (along $x-ct=\eta$, $\eta$ constant) and a left-moving wave (along $x+ct=\xi$, $\xi$ constant).
We can generalise this idea later.
\begin{example}
    In the special case where $q=0$, we have $f=g=p/2$, so
    $$y(x,t)=\frac{p(x-ct)+p(x+ct)}{2}$$
\end{example}
\subsection{Oscillation Energy}
A vibrating string has kinetic energy due to the motion of the particles in the string.
This is given by $mv^2/2$.
So the total kinetic energy on the string would be the integral
$$\operatorname{KE}=\frac{1}{2}\int_0^L\left(\frac{\partial y}{\partial t}\right)^2\mu\,\mathrm dx$$
where $\mu$ is the mass per unit length.
The (elastic) potential energy due to streching $\Delta x$ is then
$$\operatorname{PE}=T\int_0^L\left( \sqrt{1+\left( \frac{\partial y}{\partial x} \right)^2} -1\right)\,\mathrm dx\approx\frac{1}{2}T\int_0^L\left( \frac{\partial y}{\partial x} \right)^2\,\mathrm dx$$
for small $|\partial y/\partial x|$.
So the total summed energy of the string is then, via $c^2=T/\mu$,
$$E=\frac{1}{2}\mu\int_0^L\left( \left( \frac{\partial y}{\partial t} \right)^2+c^2\left(\frac{\partial y}{\partial x}\right)^2 \right)\,\mathrm dx$$
So by substituting our generalisation and using orthogonality,
$$E=\frac{1}{2}\mu\sum_{n=1}^\infty (A_n+B_n)$$
where
$$A_n=\int_0^L \left( \frac{n\pi c}{L}C_n\sin\frac{n\pi ct}{L}+ \frac{n\pi c}{L}D_n\cos\frac{n\pi ct}{L}\right)^2\sin^2\frac{n\pi x}{L}\,\mathrm dx$$
and
$$B_n=\int_0^Lc^2\left( C_n\cos\frac{n\pi ct}{L}+ D_n\sin\frac{n\pi ct}{L}\right)^2\frac{n^2\pi^2}{L^2}\cos^2\frac{n\pi x}{L}\,\mathrm dx$$
Simplifying this mess gives
$$E=\frac{1}{4}\mu\sum_{n=1}^\infty\frac{n^2\pi^2c^2}{L}(C_n^2+D_n^2)$$
which can be interpreted as the sum of the energy of all normal modes.
Also, this is constant, so it is conserved in time.
\subsection{Wave Reflection and Transmission}
Recall the travelling wave solution along the $x\pm ct$ directions.
We want to further develop this idea
\begin{definition}
    A simple harmonic travelling wave is defined as
    $$t=\operatorname{Re}(Ae^{i\omega (t-x/c)})=|A|\cos(\omega(t-x/c)+\phi)$$
    where $\phi=\arg A$ is the phase and $2\pi c/\omega$ is the wavelength.
\end{definition}
Sometimes we will just assume the $\operatorname{Re}$ is there without writing it out explicitly.\\
Consider a density discontinuity on the string at $x=0$ with
$$\mu=\begin{cases}
    \mu_-\text{, for $x<0$}\implies c_-=\sqrt{T/\mu_-}\\
    \mu_+\text{, for $x>0$}\implies c_+=\sqrt{T/\mu_+}
\end{cases}$$
So a wave $Ae^{i\omega (t-x/c_-)}$ approaching $0$ from left will break down to two parts:
The reflected wave $Be^{i\omega (t+x/c_-)}$ and the transmitted wave $De^{i\omega (t-x/c_+)}$.
The continuity condition on $y$ gives $A+B=D$ and by balancing the forces
$$T\left.\frac{\partial y}{\partial x}\right|_{x=0_-}=T\left.\frac{\partial y}{\partial x}\right|_{x=0_+}$$
(which is basically the continuity condition on $\partial y/\partial x$) it gives $2A=D(c_++c_-)/c_+$.
Combining them all gives
$$D=\frac{2c_+}{c_-+c_+},B=\frac{c_+-c_-}{c_-+c_+}A$$
In general, it is possible to have different phase shifts $\phi$.\\
Now, if $c_+=c_-$, then $D=A$ and $B=0$, so there is no reflection, which is intuitive.
If we have the Dirichlet boundary conditions $\mu_+/\mu_-\to\infty$ (interpreted as a fixed end $y=0$ at $x=0$), then $c_+/c_-\to 0$, so $D=0$ and $B=-A$ where we get total reflection (with opposite phase $\phi=\pi$).
This is also what is expected.
If we have the Neumann boundary conditions $\mu_+/\mu_-\to 0$ (interpreted as an extremely light string in $x>0$), then $c_+/c_-\to\infty$, so $D=2A$ and $B=A$, so we get total reflection with same phase $\phi=0$.
\subsection{Wave Equation in the Plane Polar Coordinates}
The wave equation in two dimensions is
$$\frac{1}{c^2}\frac{\partial^2u}{\partial t^2}=\nabla^2u$$
Under plane polar coordinates $u=u(r,\theta,t)$, we impose the boundary condition $u(1,\theta,t)=0$ for any $\theta,t$ (interpreted as a fixed rim) and initial conditions
$$u(r,\theta,0)=\phi(r,\theta),\frac{\partial u}{\partial t}(r,\theta,0)=\psi(r,\theta)$$
We use seperation of variables again.
If we substitute $u(r,\theta,t)=T(t)V(r,\theta)$, then we obtain the decoupled system
$$\begin{cases}
    \ddot{T}+\lambda c^2T=0\\
    \nabla^2V+\lambda V=0
\end{cases}$$
where $\lambda$ is a seperation constant.
Note that in plane polar,
$$\nabla^2V=\frac{\partial^2V}{\partial r^2}+\frac{1}{r}\frac{\partial V}{\partial r}+\frac{1}{r^2}\frac{\partial^2V}{\partial\theta^2}+\lambda V=0$$
We seperate the variables further by writing $V(r,\theta)=R(r)\Theta(\theta)$.
This gives
$$\begin{cases}
    \Theta^{\prime\prime}+\mu\Theta=0\\
    r^2R^{\prime\prime}+rR^\prime+(\lambda r^2-\mu)R=0
\end{cases}$$
where $\mu$ is again a seperation constant.
Assuming $\mu>0$.
Solving for $\Theta$ subject to $\Theta(0)=\Theta(\pi)$ gives our old friend
$$\Theta_m(\theta)=A_m\cos(m\theta)+B_m\sin(m\theta),m\in\mathbb Z_{>0}$$
with $\mu_m=m^2$.
For $R$, we divide the equation by $r$ and transform it into Sturm-Liouville form
$$\frac{\mathrm d}{\mathrm dr}(rR^\prime)-\frac{m^2}{r}=-\lambda rR$$
Note that the boundary condition means we only care about $r\in[0,1]$.
The boundary condition are self-adjoint, which is convenient.
\subsection{Bessel's Equation}
Substitute $z=\sqrt{\lambda}r$ gives
$$z^2\frac{\mathrm d^2R}{\mathrm dz^2}+z\frac{\mathrm dR}{\mathrm dz}+(z^2-m^2)R=0\iff (zR^\prime)^\prime+\left( z-\frac{m^2}{z} \right)R=0$$
Apparently $0$ is a regular singular point, so we substitute the power series
$$R=z^p\sum_{n=0}^\infty a_nz^n$$
which gives
$$\sum_{n=0}^\infty (a_n(n+p)(n+p-1)z^{n+p}+(n+p)z^{n+p}+z^{n+p+2}+m^2z^{n+p})=0$$
The indivial equation is then $p^2-m^2=0$, so $p=\pm m$.
For $p=m$, it is called the regular solution (otherwise you get a singular point at $0$).
Here we have the recurrence
$$(n+m)^2a_n+a_{n-2}-m^2a_n=0\implies a_n=-\frac{1}{n(n+2m)}a_{n-2}$$
This gives the even series with solutions
$$a_{2n}=a_0\frac{(-1)^n}{2^{2n}n!(n+m)(n+m-1)\cdots (m+1)}$$
Conveniently we set $a_0=1/(2^mm!)$ which gives the Bessel function
$$J_m(z)=\left( \frac{z}{2} \right)^m\sum_{n=0}^\infty\frac{(-1)^n}{n!(n+m)!}\left( \frac{z}{2} \right)^{2n}$$
Actually, if we set $y=\sqrt{z}R$, then we will obtain
$$y^{\prime\prime}+y\left( 1+\frac{1}{4z}-\frac{m^2}{z^2} \right)=0$$
which, as $z\to\infty$, gives the approximation $y^{\prime\prime}\approx -y$ which has (approximated) solutions $R\approx z^{-1/2}(A\cos z+B\sin z)$ for $A,B$ constants.\\
Back to Bessel's function.
In fact, when $m=\nu\notin\mathbb Z$, this power solution also works but replace $(n+m)!$ by $\Gamma(n+\nu+1)$.
The second solution with $p=-m$ is known as the Neumann functions (or Bessel functions of second kind) which satisfies
$$Y_m(z)=\lim_{\nu\to m}\frac{J_\nu(z)\cos(\nu\pi)-J_{-\nu}(z)}{\sin(\nu\pi)}$$
As one can verify, there are a number of identities associated with Bessel functions, e.g. we have $(z^mJ_m(z))^\prime=z^mJ_{m-1}(z)$.
This also implies
$$\begin{cases}
    J_m^\prime(z)+mJ_m(z)/z=J_{m-1}(z)\\
    J_{m-1}(z)+J_{m+1}(z)=2mJ_m(z)/z\\
    2J_m^\prime(z)=J_{m-1}(z)-J_{m+1}(z)
\end{cases}$$
Naturally, we want to study the aymptopic behaviour of $J_m$ and $Y_m$.
As $z\to 0$, easily $J_0(z)\to 1$, $J_m(z)\sim (z/2)^m/m!$ and $Y_0(z)\to 2\log(z/2)/\pi$, $Y_m(z)=-(m-1)!(2/z)^m/\pi$.
So $0$ is a singularity of $Y_m$.\\
For large $z\to\infty$, $J_m,Y_m$ converges to $0$ oscillatorily, i.e.
$$J_m(z)\sim\sqrt{\frac{2}{\pi z}}\cos\left( z-\frac{m\pi}{2}-\frac{\pi}{4} \right),Y_m(z)\sim\sqrt{\frac{2}{\pi z}}\sin\left( z-\frac{m\pi}{2}-\frac{\pi}{4} \right)$$
This hints that we might want to take a look at the infinitely many zeros in $\mathbb R_{>0}$ of Bessel function.
Let $j_{m,n}$ to be the $n^{th}$ positive zero of $J_m$.
So the asymptopic formula above shows approximately $j_{m,n}\approx \tilde{j}_{m,n}=n\pi +m\pi/2-\pi/4$ with accuracy $|(j_{m,n}-\tilde{j}_{m,n})/j_{m,n}|<1/(10n)$ for $n>m^2/2$.
A few of the actual values are below:
$$j_{0,1}\approx 2.405,j_{0,2}\approx 5.520,j_{0,3}\approx 8.653$$
It is a fun activity for the reader to try and draw $J_m$.
\subsection{A Vibrating Drum}
So the radial solutions become
$$R_m(z)=R_m(\sqrt{\lambda}r)=AJ_m(\sqrt{\lambda}r)+BY_m(\sqrt{\lambda} r)$$
For $A,B$ constants.
But we want the solution to be bounded near zero which would mean $B=0$.
We also need $R(1)=0$, therefore $\lambda_{m,n}=j_{m,n}^2$ are the eigenvalues.
Therefore the spacial solutions are
$$V_{m,n}(r,\theta)=\Theta_m(\theta)R_{m,n}(\sqrt{\lambda_{m,n}}r)=(A_{m,n}\cos(m\theta)+B_{m,n}\sin(m\theta))J_m(j_{m,n}r)$$
Putting these eigenvalues to the temporal equation then shows $T_{m,n}$ are just linear combinations of $\cos(j_{m,n}ct)$ and $\sin(j_{m,n}ct)$.
Putting everything together we get $u(r,\theta,t)=A+B+C$ where
$$A=\sum_{n=1}^\infty J_0(j_{0n})r(A_{0n}\cos(j_{0n}ct)+C_{0n}\sin(j_{0n}ct))$$
$$B=\sum_{m=1}^\infty\sum_{n=1}^\infty J_m(j_{m,n}r)(A_{m,n}\cos(m\theta)+B_{m,n}\sin(m\theta))\cos(j_{m,n}ct)$$
$$C=\sum_{m=1}^\infty\sum_{n=1}^\infty J_m(j_{m,n}r)(C_{m,n}\cos(m\theta)+D_{m,n}\sin(m\theta))\sin(j_{m,n}ct)$$
We still have to impose the initial conditions.
$$\phi(r,\theta)=u(r,\theta,0)=\sum_{m=0}^\infty\sum_{n=1}^\infty J_m(j_{m,n}r)(A_{m,n}\cos(m\theta)+B_{m,n}\sin(m\theta))$$
$$\psi(r,\theta)=\frac{\partial u}{\partial t}(r,\theta,0)=\sum_{m=0}^\infty\sum_{n=1}^\infty j_{mn}cJ_m(j_{m,n}r)(C_{m,n}\cos(m\theta)+D_{m,n}\sin(m\theta))$$
As usual, we find the coefficient by orthogonality of these eigenfunctions.
We do already know that they are orthogonal, what we really need is the normalisation constant.
Some calculation then reveals
$$\int_0^1J_m(j_{m,n}r)J_m(j_{m,k}r)r\,\mathrm dr=\frac{1}{2}(J_m^\prime(j_{m,n}))\delta_{nk}=\frac{1}{2}J_{m+1}(j_{m,n})^2\delta_{n,k}$$
Therefore for $p>0$,
$$A_{pq}=\left(\int_0^{2\pi}\int_0^1\cos(p\theta)J_p(j_{pq}r)\phi(r,\theta)r\,\mathrm dr\mathrm d\theta\right)\left( \frac{\pi+\delta_{0p}\pi}{2}J_{p+1}(j_{pq})^2 \right)^{-1}$$
We can obtain $B_{m,n},C_{m,n},D_{m,n}$ in similar ways.
\begin{example}
    Consider $\phi=1-r^2$, so we have $\forall m,B_{m,n}=0$ and $\forall m\neq 0,A_{m,n}=0$.
    We also set $\psi=0$, which means $C_{m,n}=D_{m,n}=0$ for all $m,n$.
    By calculations,
    $$A_{0,n}=\frac{2}{J_1(j_{0,n})^2}\frac{J_2(j_{0,n})}{j_{0,n}^2}\approx\frac{J_2(j_{0,n})}{n}$$
    for large $n$.
    So the solution is
    $$u(r,\theta,t)=\sum_{n=1}^\infty\frac{2}{J_1(j_{0,n})^2}\frac{J_2(j_{0,n})}{j_{0,n}^2}J_0(j_{0,n}r)\cos(j_{0,n}ct)$$
    So the fundamental frequency is $\omega=j_{0,1}c(2/d)\approx 4.8c/d$ which is higher than the value for a string, whose value is approximately $77\%$ of the figure for the drum.\\
    A sketch of the nodal lines will then show that the solution is pretty close to our intuition.
\end{example}